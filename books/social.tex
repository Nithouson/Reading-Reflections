
\section{社会主义}

\subsection*{共产党宣言}
\addcontentsline{toc}{subsection}{共产党宣言}
\par \emph{马克思 \ 恩格斯 / 人民出版社 2014年版}
\par 这本小册子是我学习马克思主义读的第一本书,在一次长途高铁上看完。《宣言》宣告了资本主义制度的必然灭亡,为共产主义运动提供了理论指导。对于前一点,《宣言》论述得比较简单,此时《资本论》尚未问世,要想透彻理解还需进一步阅读。对于后一点,书中注释和附录给出了当时历史事件的不少细节,不熟悉共产主义运动史的我只能看个大概,认识到马克思、恩格斯并非学究式的理论家,而是积极投身运动前线;基于对资本主义内在矛盾的认识,马克思、恩格斯清楚地认识到两次危机之间、资本主义蓬勃发展的时期不适合发动革命,这就体现了理论的指导作用。《宣言》的历史意义远远超过了其理论阐述本身,在共产主义运动中具有旗帜性的象征意义,不少文句在相关影视作品中被反复引用。
\par \rightline{2022年8月31日}

\section{科学社会学}

\subsection*{有了博士学位还不够:学术生涯指南}
\addcontentsline{toc}{subsection}{有了博士学位还不够:学术生涯指南}
\par \textbf{A PhD Is not Enough: A Guide to Survival in Science}
\par \emph{Peter J. Feibelman / 张婷婷译} 
\par 问:朋友博士毕业,送什么礼物合适?
\par 答:《有了博士学位还不够》。
\par 此书是学界职业生涯的经验之谈,从标题中的“Survival”便可看出作者强调其中的困难不应被我们这样的年轻人低估。作者写到应当了解所做研究的背景和意义,而非只关注技术细节的解决,我思量这一点已经固化在现今期刊论文的篇章结构里了,难怪读上世纪的论文感觉他们写作更自由些。书中不乏有价值的职业经验,却也显出些“成功学”味道和内卷气息。在我看来,职业的成功和人生的成功不能说是正交的,至少也是有夹角的吧。
\par \rightline{2021年12月19日}
