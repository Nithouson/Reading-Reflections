\section{人工智能}

\subsection*{科学之路}
\addcontentsline{toc}{subsection}{科学之路}
\par \textbf{Quand la machine apprend}
\par \emph{Yann LeCun / 李皓、马跃译}

\par 杨立昆在中国演讲时宣布中文名的照片让我印象深刻,这样一位深度学习宗师、图灵奖得主还是免不了取了个大俗名。作者在书中回顾了自己的学术生涯,分享了自己对人工智能的思考,中间几章顺带科普了反向传播和卷积网络。
\par 令我印象深刻的是作者在神经网络的低谷期的坚持并非毫无依据的执念,他认为“人类智能如此复杂,必须建立一个具有自我学习能力和经验学习能力的自组织才能复制它”。当时Geoffrey Hinton则相对悲观,他在40岁生日时说“我的职业生涯到头了,什么也做不成了。”书中杨立昆说神经网络loss存在多个极小值“从来都不是问题”,似乎过于乐观。有趣的是,在近期GPT-4发展带来的争论中,Hinton从Google离职以警告AI的风险,而杨立昆则旗帜鲜明地反对Yoshua Bengio等人暂停大模型研究的倡议。
\par 关于学术研究,有一段话写得精彩:“我贪婪地阅读,我熟知前人的所有工作。在探索之路上我们并非孤身一人,时机到来之时,那些已经存在但尚未提出的理念会一个接一个地涌入许多人的头脑中。”
\par 最后吐槽一下中文翻译,毫不夸张地说,此书的译者和校对者没有一个具备基本的机器学习知识,不然不会把双曲正切函数译成“双曲线切线”,把神经网络层的堆叠译成“堆栈”,把图的最短路译成“图像最短路径”。类似讹误还有不少,着实令人哭笑不得。
\par \rightline{2023年6月28日}
