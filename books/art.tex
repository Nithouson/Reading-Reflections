\section{现代艺术}

\subsection*{数学\ 艺术}
\addcontentsline{toc}{subsection}{数学\ 艺术}
\par \textbf{Math Art: Truth, Beauty, and Equations}
\par \emph{Stephen Ornes / 杨大地译} 

\par 这本书介绍的并非数学与艺术结合的经典议题,如黄金比例、透视法等,而是十几篇介绍当代艺术家作品和创作理念的小品文。此书译笔流畅,附有不少优美的艺术作品插图,适合作为休闲读物。书中介绍到David Bachman尝试用曲线方程描述自然事物;Melinda Green利用分形集绘出佛陀图案。这让我忆起大一刚接触MathStudio的时候,我和舍友热衷于用它绘出各种空间曲面和分形造型。不过Green的想法更进一步,Buddabrot渲染的是复平面上一部分点的迭代轨迹,而非分形集合本身。Robert Bosch的TSP艺术也十分有趣,他把货郎担问题的求解视为一种作画的方式。每篇后还有相关数学背景知识的介绍,但对比较了解数学文化的读者来说,大多数内容已并不新鲜。其中三周期极小曲面是一个我不甚理解而感兴趣的话题。
\par \rightline{2022年6月17日}

\section{图像小说、绘本}

\subsection*{不属于我的城市}
\addcontentsline{toc}{subsection}{不属于我的城市}
\par \emph{寂地}

\par 这是寂地2023年的新书,发售不久我就购入了签绘版。后来作者来五道口PAGEONE与读者见面,若不是入场券捆绑一本新书可能我就去了。本来我是不为绘本写阅读闲谈的,但此书是绘本故事(图像小说)与短篇小说的穿插,因此提笔写来。(其实她的《My Way》是我最喜欢的系列绘本作品,也有更多内容值得分析和讨论。)
\par 正如书名所说,本书讲述的是大城市中渺小个体的生存处境,前一半的故事涉及过劳死、抑郁、分居状态下脆弱的感情等议题(说这本书是治愈系的,确定不是致郁吗)。《躺在你身旁》一篇中,灯光师的话令人心酸,让人联想到很多底层劳动者的生存状态:“我以前希望能赚很多钱,好买一套大房子。现在只希望每天都睡饱、睡够。”后三篇故事中“左手”和“影子”的设定有对应之处,都提倡为自己而非别人而生活。同事影子尖刻的议论读来令人沮丧,人心真的如此险恶不堪吗?作者的评论颇有见地:“人们常说人心复杂,我却越来越觉得人心其实很简单,无非就是一个个迷失的痛苦灵魂,焦虑不安地争夺着不存在的阵地。”主角后来喜欢上了户外徒步,她说“世界很广阔,足以包容所有狭隘的人心。”
\par 寂地的绘本故事富有幻想色彩,她善于把精神世界用隐喻的方式具象化,阅读时会自然地试图解读这种投射,而解读的结果又往往引人深思。比如乙粒粒身高的增长实际指社会阶层的变化,音符、影子则具有多重指向,气泡的含义我不甚理解。这一点在《My Way》的一些故事中更为突出。
\par \rightline{2023年11月15日}
