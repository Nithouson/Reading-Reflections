\section{复杂性}

\subsection*{规模}
\addcontentsline{toc}{subsection}{规模}
\par \textbf{Scale: The Universal Law of Growth, Innovation, Sustainability, and the Pace of Life in Organisms, Cities, Economies and Companies}
\par \emph{Geoffrey West / 张培译} 

\par 还记得读本科时某门课课间,一位陌生的同学在读此书,注意到我探询的目光,便热情地向我推荐。到了升学读研之时,从事城市复杂性研究也是我的选项之一(我所在的团队有一个小组关注这个领域);开始读这本书后,我对复杂性科学的好感却下降了。
\par 这实在是科学界的成功范例,很难想象其它什么研究主题能仅用取对数回归这样简单的分析方法在顶级科学期刊上连续发表文章。一些随规模幂律增长的统计规律被冠以“生物、城市、公司的复杂性大统一理论”这样野心勃勃的名号,加之对幂指数不无牵强的机理解释。对于复杂系统而言,尽管其行为难以用定量规律精确预测,其大致趋势和共性的存在并不值得惊叹。发展心理学中有人毕生发展历程的大致描述,反映了一种平均的生长历程。那是否也可以说是“人类的大统一理论”呢?我无意贬低类似研究范式的价值(比如对参数提供了一种“基准”),但同样地,它们的价值不应被夸大。我并不反对“大道至简”的信条,但这样一些粗糙的统计规律,或许称不上什么“大道”。
\par \rightline{2022年3月5日}

\par 补注:现在看来我当时对这本书的评价有些偏颇,这一定程度上源于我对相关工作缺乏深入了解。但不可否认的是,现有理论能够给出可靠解释的现象仍是较少的一部分,离“城市大一统理论”的远景还有很长的路要走。
\par \rightline{2023年1月1日}

\subsection*{随椋鸟飞行:复杂系统的奇境}
\addcontentsline{toc}{subsection}{随椋鸟飞行:复杂系统的奇境}
\par \textbf{In un Volo di Storni: Le Meraviglie dei Systemi Complessi}
\par \emph{Giorgio Parisi / 文铮译} 

\par 作为新晋诺贝尔物理学奖得主,帕里西这部小书并未试图系统地向公众介绍自己的研究成果,而是用散文化的笔法将科学研究中的一些经历和思考娓娓道来,行文自由而不显造作。第一篇《与椋鸟齐飞》称得上其中富有诗意的佳作,讲述了通过拍摄鸟群三维轨迹研究其运动规律的经历。后面对隐喻和无意识思维的讨论也引起了我的共鸣,体现了作者对科学研究本身的反思。至于科普二级相变和自旋玻璃的篇目,我倒觉得略叙大意终究只能是蜻蜓点水,不足以让人有深刻的领会。最后,有一句话特别让我印象深刻:“科学研究就像诗歌创作一样,没有任何迹象表明创作过程的艰辛,以及与之相伴的怀疑与彷徨。”
\par \rightline{2022年9月25日}

