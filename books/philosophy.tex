\section{西方哲学}

\subsection*{裴洞篇}
\addcontentsline{toc}{subsection}{裴洞篇}
\par \emph{柏拉图 / 王太庆译} 
\par 遥想本科时选修的通识课“哲学导论”,课上介绍的中外哲学观点现在大多都不记得了。但我还清楚地记得第一节课教授讲解什么是哲学:“哲学是追求终极智慧的学问,但和宗教不同,它是以逻辑论证的方式探求终极智慧。”加上我本人受数学和自然科学影响颇深,读哲学著作不免会当做数学理论那样严格的逻辑推演体系来读,其结果就是发现所谓“论证”只能说是一种“说明”。当时读《会饮篇》如此,三年后读《裴洞篇》仍是如此。顺便提一句,当时我的课程小论文写的便是批驳《会饮篇》中阿里斯托芬的“同体神话”,以及探讨爱的本义。可惜文稿遗失,现在我还颇想知道那是我是怎么思考爱的。
\par 《裴洞篇》写苏格拉底临刑前和门人探讨死亡,试图论证灵魂不朽。首要的问题自然是“什么是灵魂”。对话中没有专门的讨论,但从书中的意思理解,灵魂与肉体相对,大致是指人的精神性的部分,并且是灵魂让躯体具有了生命。对话中苏格拉底宣扬的灵魂不朽、投胎转世,此生潜心研习哲学、来世与神灵共处,读来颇有教主之风。至于如何论证灵魂不朽,类比推理是主要的手段:“火伴随热而不容纳冷;三伴随奇(指“奇数”这一性质)而不容纳偶;所以灵魂伴随生命而不容纳死亡”。不死的,就是不灭的,所以灵魂不朽。这种类比推理当然不百分百可靠,不然就可以说“奇数不圆满,不容纳完美,所以奇完全数必然不存在”了。
\par 后来我又忆起一个哲学范畴叫“超验”,意思是超出人类认知的范围,“只可信,不可证”。或许我应该换一种读法,即哲学结论并不是要严格地证明对与错,而是将所谓“论证”看成一种说明,说明不必蕴含结论,但可以让结论更可靠:让人觉得足够可靠的观点,也会让人接受。让我这样的唯物主义者、无神论者接受“灵魂不朽”这样的观点自然是困难的,但书中苏格拉底在生命的最后时刻讨论死亡与灵魂,呼吁门人摆脱形体的欲望和快乐,关注人的理性和精神,其本身就具有十足的浪漫色彩。我想一本哲学书引发了读者对终极问题的思考,这书就没有白读。
\par \rightline{2021年8月28日}

\section{中国哲学}

