
\section{数学文化与通俗读物}
\subsection*{无穷的画廊——数学家如何思考无穷}
\addcontentsline{toc}{subsection}{无穷的画廊——数学家如何思考无穷}
\par \textbf{Gallery of the Infinite}
\par \emph{Richard Evan Schwartz} 
\par 这本书是在四川省图书馆外文区看到的,已经做好了进原版书的准备,结果双十一欣喜地发现出了中译本。书中用漫画形式阐释了朴素集合论中的一些内容,特别是Cantor开创的无穷集合论。这一理论的结果可称令人惊叹,比如“有理数和整数一样多”、“实数集是比有理数更高的无穷”,正应了《逍遥游》中所言:“汤问棘曰:‘上下四方有极乎?’ 棘曰:‘无极之外,复无极也。’” 我几乎是一口气翻完了全书,不时为书中有趣的讲法而捧腹大笑。此书的绘画也十分精彩,有着浓郁的现代气息。
\par \rightline{2020年11月21日}
