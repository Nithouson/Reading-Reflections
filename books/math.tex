
\section{数学文化与通俗读物}
\subsection*{无穷的画廊——数学家如何思考无穷}
\addcontentsline{toc}{subsection}{无穷的画廊——数学家如何思考无穷}
\par \textbf{Gallery of the Infinite}
\par \emph{Richard Evan Schwartz} 
\par 这本书是在四川省图书馆外文区看到的,已经做好了进原版书的准备,结果双十一欣喜地发现出了中译本。书中用漫画形式阐释了朴素集合论中的一些内容,特别是Cantor开创的无穷集合论。这一理论的结果可称令人惊叹,比如“有理数和整数一样多”、“实数集是比有理数更高的无穷”,正应了《逍遥游》中所言:“汤问棘曰:‘上下四方有极乎?’ 棘曰:‘无极之外,复无极也。’” 我几乎是一口气翻完了全书,不时为书中有趣的讲法而捧腹大笑。此书的绘画也十分精彩,有着浓郁的现代气息。
\par \rightline{2020年11月21日}

\subsection*{数学之旅:数学的抽象与心智的荣耀}
\addcontentsline{toc}{subsection}{数学之旅:数学的抽象与心智的荣耀}
\par \emph{王维克} 

\par 在大学里,大多数学生接受的数学教育止步于“高等数学”“线性代数”“概率统计”。作为其它学科的工具或许差不多够用了,但其实还有不少 “遗珠”,未曾领略则殊为遗憾。这本书试图为非数学专业学生介绍一些更深刻理论的思想,如康托尔对无穷集合基数的讨论、Banach空间与Hilbert空间的定义、混沌与分形等,我认为是一种有益的尝试。当然对于已系统学习过相关理论的读者,这本书信息量不大;书中有些通俗化的解说也略显勉强。
\par 此书开篇介绍说“数学的目的是理解和揭示自然”,这一点我不敢苟同。我以为数学的发展已远远超过了描述我们所在世界的范畴,而是触及其它种种可能世界甚至不可能世界。数学可以在人类思维和创造力的推动下独立发展,不依赖于其理论对于其它学科的应用价值;它既显示人类精神的成就,也是人类精神的食粮。或许这才是副标题 “数学的抽象与心智的荣耀”所蕴含的精神。
\par \rightline{2022年10月15日}

\subsection*{数学 艺术}
\addcontentsline{toc}{subsection}{数学 艺术}
\par \textbf{Math Art: Truth, Beauty, and Equations}
\par \emph{Stephen Ornes / 杨大地译} 

\par 这本书介绍的并非数学与艺术结合的经典议题,如黄金比例、透视法等,而是十几篇介绍当代艺术家作品和创作理念的小品文。此书译笔流畅,附有不少优美的艺术作品插图,适合作为休闲读物。书中介绍到David Bachman尝试用曲线方程描述自然事物;Melinda Green利用分形集绘出佛陀图案。这让我忆起大一刚接触MathStudio的时候,我和舍友热衷于用它绘出各种空间曲面和分形造型。不过Green的想法更进一步,Buddabrot渲染的是复平面上一部分点的迭代轨迹,而非分形集合本身。Robert Bosch的TSP艺术也十分有趣,他把货郎担问题的求解视为一种作画的方式。每篇后还有相关数学背景知识的介绍,但对比较了解数学文化的读者来说,大多数内容已并不新鲜。其中三周期极小曲面是一个我不甚理解而感兴趣的话题。
\par \rightline{2022年6月17日}

