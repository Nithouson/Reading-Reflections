
\section{数学文化与通俗读物}

\subsection*{素数之恋}
\addcontentsline{toc}{subsection}{素数之恋}
\par \textbf{Prime Obsession: Bernhard Riemann and the Greatest Unsolved Problem in Mathematics}
\par \emph{John Derbyshire}

\par 这本以黎曼猜想为主题的科普书采用数学知识与数学史内容交错的章节结构。或许由于我对数学史兴趣有限,后者一些时代背景、人物经历的叙述读来觉得有些琐碎,但也不失为读数学知识的一种调剂。对于数学知识本身的讲述,这本书超出了我对科普读物的预期:作者不仅介绍了什么是黎曼猜想,还尽可能说明了黎曼Zeta函数为何与素数关联(黎曼1859年的论文指出,借助J函数这一中间工具,素数计数函数π(x)可由黎曼Zeta函数的非平凡零点表出),黎曼猜想为什么重要(素数定理指出π(x)可用Li(x)估计,黎曼猜想成立将给出估计偏差的一个精确上界估计),以及其它丰富的研究面向,尽管很多技术细节被不可避免地略去。如果你愿意花十几个小时的阅读时间了解黎曼猜想(或者一半的时间,跳过那一半数学史章节),我相信本书是个不错的选择,而且书的主干内容你几乎肯定能读懂。
\par \rightline{2023年12月30日}


\subsection*{千年难题}
\addcontentsline{toc}{subsection}{千年难题}
\par \textbf{The Millennium Problems}
\par \emph{Keith Devlin}

\par 在我看来,介绍数理前沿进展的高级科普多少面临这样的困境:读者可以满足一些好奇心,获得令人羡慕的谈资,却无法真正理解所讲的内容。比如市面上讲相对论、量子力学的科普书多如牛毛,可若是仅凭茶余饭后看本科普书就能理解,物理系的学生也不必花几个学期学习这些理论了。读罢此书,我对七个问题的理解程度仍取决于先前的知识储备:P对NP问题是唯一比较清楚的,这得益于听刘田老师理论计算机科学的网课;Yang-Mills和霍奇猜想仍是云里雾里;剩下四个问题多少接触过相关的数学,算是知道个大概。
\par 话说回来,这本书算得上不错的数学科普和数学文化读物。书的假想读者是高中毕业生,介绍每个问题都作足了铺垫(BSD和霍奇猜想除外),因而大部分内容读来比较轻松。此书的翻译和编校也值得肯定,术语翻译过关,译者还加了些必要的脚注补充说明。另一方面,作者对当代数学的发展状态作了很好的描述:“(数学)抽象的程度和速度可能已经达到了只有专家才能跟上的阶段”;而要理解前沿的概念,“唯一的途径也只能是循着通向它们的由一个个抽象环节连成的整个链条。”关于这一点作者还有一个有趣的说法:“作为对你有勇气研究不可见之物的回报,我(数学之神)将使数学研究变得更简单。”而数学家的努力,源于对真理与美的追求,对自我的超越。对于前者,设立克雷数学促进会的Landon Clay的话是很好的说明:“求知欲是人类的本性之一。遗憾的是,已确立的各种宗教不再提供令人满意的答案,这就转变成对确定性和真理的一种需求。这就是数学为什么而运作,为什么人们为之奉献终身。它是对真理的渴望,是对驱动着数学家的数学之美妙与优雅的回应。”对于后者,作者将数学家与登山家类比,世界上顶尖的数学家自然会去研究千年难题,“因为它就在那里。”作者说,相比于一般数学家和业余爱好者,他们只是投入了更多的奉献和激情。
\par “外行看热闹,内行看门道。”继Poincaré猜想之后,此生能见证谁解决第二个千年难题,也是一件值得期待的事。当然我不想满足于浮光掠影地看数学,椭圆曲线理论中Mordell, Nagell-Lutz, Mazur等的结果就令我十分好奇。有时间了解一下椭圆曲线有理点的理论,想来是极好的。
\par \rightline{2023年11月28日}



\subsection*{无穷的画廊——数学家如何思考无穷}
\addcontentsline{toc}{subsection}{无穷的画廊——数学家如何思考无穷}
\par \textbf{Gallery of the Infinite}
\par \emph{Richard Evan Schwartz} 
\par 这本书是在四川省图书馆外文区看到的,已经做好了进原版书的准备,结果双十一欣喜地发现出了中译本。书中用漫画形式阐释了朴素集合论中的一些内容,特别是Cantor开创的无穷集合论。这一理论的结果可称令人惊叹,比如“有理数和整数一样多”、“实数集是比有理数更高的无穷”,正应了《逍遥游》中所言:“汤问棘曰:‘上下四方有极乎?’ 棘曰:‘无极之外,复无极也。’” 我几乎是一口气翻完了全书,不时为书中有趣的讲法而捧腹大笑。此书的绘画也十分精彩,有着浓郁的现代气息。
\par \rightline{2020年11月21日}

\subsection*{无穷农场上的生活}
\addcontentsline{toc}{subsection}{无穷农场上的生活}
\par \textbf{Life on the Infinite Farm}
\par \emph{Richard Evan Schwartz} 

\par 这是《无穷的画廊》作者的又一部数学科普绘本,目前尚无中译本。这部书科普的主要内容有二:无穷集合的一些特殊性质,即“希尔伯特大饭店”;无穷包含在有限之中的思想,如双曲几何中的Poincaré圆盘模型。感觉第二部分用火山口或者湖泊类比有些乏力,大众读者依然不容易完全理解。值得赞许的是作者插入了一些无关数学但增添趣味性的设定,如无穷动物如何通过彼此,可以称为无穷农场的“世界观”;加之别具一格的画风,展现了数学思想产生的艺术魅力。
\par \rightline{2023年4月17日}

\subsection*{真正巨大的数}
\addcontentsline{toc}{subsection}{真正巨大的数}
\par \textbf{Really Big Numbers}
\par \emph{Richard Evan Schwartz}

\par 又找到《无穷的画廊》作者的一部数学科普绘本,同样尚无中译本。这部书科普的主要内容有二:通过实际例子说明不同数量级的概念;幂塔运算等定义更大数的方法。令人欣喜的是,作者没有照本宣科讲述Knuth箭头符号等现成的概念,而是通过定义方块、多边形、圆圈箭头等形象化的符号表现了攀登数字之梯的更多可能(用Knuth箭头符号表达,书中定义的方块n、五边形n依次为10↑↑n,10↑↑↑(n+1);五边形n定义为10套n-1层方块(10↑↑↑n)似乎更合适,这样六边形、七边形依次为10↑↑↑↑n,10↑↑↑↑↑n,而圆圈箭头定义是不同级箭头符号的叠加)。书中插图同样饶有趣味,值得一观。
\par \rightline{2023年12月15日}


\subsection*{数学之旅:数学的抽象与心智的荣耀}
\addcontentsline{toc}{subsection}{数学之旅:数学的抽象与心智的荣耀}
\par \emph{王维克} 

\par 在大学里,大多数学生接受的数学教育止步于“高等数学”“线性代数”“概率统计”。作为其它学科的工具或许差不多够用了,但其实还有不少 “遗珠”,未曾领略则殊为遗憾。这本书试图为非数学专业学生介绍一些更深刻理论的思想,如康托尔对无穷集合基数的讨论、Banach空间与Hilbert空间的定义、混沌与分形等,我认为是一种有益的尝试。当然对于已系统学习过相关理论的读者,这本书信息量不大;书中有些通俗化的解说也略显勉强。
\par 此书开篇介绍说“数学的目的是理解和揭示自然”,这一点我不敢苟同。我以为数学的发展已远远超过了描述我们所在世界的范畴,而是触及其它种种可能世界甚至不可能世界。数学可以在人类思维和创造力的推动下独立发展,不依赖于其理论对于其它学科的应用价值;它既显示人类精神的成就,也是人类精神的食粮。或许这才是副标题 “数学的抽象与心智的荣耀”所蕴含的精神。
\par \rightline{2022年10月15日}


