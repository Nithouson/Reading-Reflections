
\section{中国古代文学}

\subsection*{三国演义}
\addcontentsline{toc}{subsection}{三国演义}
\par \emph{罗贯中} 
\par 我初中时读了《西游记》和《水浒传》;《三国演义》读了一半左右就上了高中,便搁置下来。六年多过去,这才又拿起来读;由于间隔时间太久,干脆从头读起。22岁读《三国》显得晚了些,甚至说起来有些令人惭愧,但晚读也有晚读的好处:初中时我每天强撑着完成读三回的任务,现在倒觉得其中的权谋计策有些意思,不必规定任务就津津有味的读下去了。初中时只觉得陈琳讨曹操的檄文“畅快淋漓”,如今其中影响最大的精彩片段当然是“诸葛亮骂死王司徒”了,哈哈。
\par 读此书似看戏。在历史舞台上,主公、忠臣、反贼、内奸,你方唱罢我登场。对于没有实权、说被废就被废的君主,处于弱势或战败的诸侯、武将,常常面临“生”与“义”的抉择。我常叹献帝不能掌控命运,被权臣肆意欺凌,还不如效仿高贵乡公曹髦以卵击石,有尊严地死去。至于臣子,有的宁死不降,有的“择良木而栖”,倒都可理解,毕竟内部矛盾不涉及民族大义。最惨的是“生”“义”二者皆失,如荆州刘琮,拱手投降,被诛于赴任路上。而在残酷的政治斗争中,一些小人物难以左右大局,却一样做了牺牲品。比如那些送信的使节就令人同情:有时还不知道信里写的是啥,就被对方“怒斩来使”。
\par “均势”是我的另一点体会。魏国最强,因而孙刘联盟,魏伐蜀则吴伐魏,魏伐吴则蜀伐魏,这才得以成鼎足之势。这种“均势”在现当代国际政治中也屡见不鲜。在一国之内,君臣之间也存在着一种“均势”:曹操、刘备、孙权能统领手下文臣武将,而他们的后继者有的没有那样的雄才,以致手下功高盖主,大权在握。掌握重权的臣子,若没有诸葛亮那样的尽忠之心,行废立、谋篡位便是常见的事了。
\par 小说作为封建时代的产物,竭力渲染“天命”“神鬼”:大将身死之前,必有“将星坠于野”;孔明、关羽死后显灵;甚至出现了管辂教人求神仙改寿数的情节。要是真有一个数据库保存着每个人的寿数,那岂不是会有大量程序员毕生致力于入侵那个数据库?
\par \rightline{2020年7月6日}

\subsection*{红楼梦}
\addcontentsline{toc}{subsection}{红楼梦}
\par \emph{曹雪芹 / 无名氏续\  程伟元、高鹗整理} 
\par 《红楼梦》被誉为中国古典小说的高峰,因其复杂性、残缺性引发了后人极为丰富的讨论和探索。红学家如俞平伯、周汝昌、蔡义江、梁归智,文人学者如张爱玲、王蒙、叶朗、王博,通俗讲者如刘心武、蒋勋、欧丽娟,还有众多关心和参与其讨论的读者和红学爱好者……一部小说受到如此广泛和深入的关注,我想在世界文学史上都是绝无仅有的。而进入这些讨论的钥匙,正是《红楼梦》原著。
\par 《红楼梦》的人物塑造、语言功力之高,也是我读原著之前就知道的。在读原著之前,我就知道“金陵十二钗”,特别是林黛玉的形象已成为日常用典;高中语文老师便让我们背“白玉为床金做马”;初中时候一位女同学谈到作文文笔优美的秘诀,“你们去读《红楼梦》吧,那实在是好词好句的宝库。”我读到宝玉的酒令“滴不尽的相思血泪抛红豆”,想起这正是她作文里曾整段引用的话。
\par 去年夏天去北京植物园,走到曹雪芹纪念馆,我道:“不会有人还没看过《红楼梦》吧?不会吧不会吧?哦,原来这人是我自己。” 终于下决心读四大名著中剩下的最后一部,是在今年春天;到11月20日读完,期间虽也看些别的书,但周末大段的读书时光,都在这里面了。
\par 读书之初最深的感受是叙述之细致:第七回写周瑞家的给姑娘们送宫花,作者花不少笔墨依次写来,区区小事花了近半回笔墨。有人说作者正是从细处着手刻画人物性格;而在我看来,作者能把荣宁府、大观园里每个人的一言一行写得如此细致,这一切在读者看来也变得真实可感了。读着读着,我似乎觉得自己在见证少年宝玉的生活;提到黛玉、湘云、宝琴,竟仿佛现实中朋友的名字一般。我印象很深是一个情节是第二十三回林黛玉听《牡丹亭》,闻“如花美眷,似水流年”而醉心落泪;时隔数百年,这里面也有我们青春的影子。
\par 我曾和父亲交流读《红楼梦》的感受,殊不知我读来最厌烦的对服饰器物的铺叙正是他最感兴趣的;而他一概跳过的诗词曲,却是我最为钟爱、反复咀嚼的。我到书店里翻欧丽娟的书,她提到对雅文化的赞美是作品的主题之一,对此我是赞同的。书中多次写到宝玉和姐妹们结社,赋诗联句,我读来作者本意并不是所谓“封建统治阶级文学的空虚”,的确是展现了作诗的技艺、巧思和乐趣;宝玉题大观园匾额对联、香菱向黛玉学诗、宝玉作姽婳词几章也都十分精彩。穿插在公子闺秀们的风雅游戏之间的,却是现实世界的种种不堪:从家塾学生的打斗,到家族中的勾心斗角、聚赌通奸……再看作诗联句,芦雪广即景联诗何其热闹,到了凹晶馆则唯有黛玉湘云二人,这不正是贾家由盛转衰的象征吗?
\par 要说诗歌艺术,我最喜欢的还是第五回宝玉梦游太虚幻境,金陵十二钗的判词和红楼梦十二支曲,其中暗暗伏下整个故事的走向和每个主要角色的结局。伏笔照应也是小说艺术中很吸引我的一种,因而红学诸多流派,我对探佚学最有兴趣。虽说续书作者总引用前八十回的情节来表明自己承接前八十回,但看了第五回的“剧透”,就知道续书的故事走向完全错了。续书中虽写到元妃薨逝、王子腾病亡,贾家被抄,但最后结局居然是“沐皇恩贾家延世泽”,贾赦、贾珍被赦,贾政官复原职,死刑犯薛蟠获释,香菱还成了正室;虽说宝玉离家出走,宝钗怀孕留下子嗣,贾兰也得中举人。好一个大团圆结局!可是这恰恰是最不应该大团圆的小说。读到前八十回末尾,我已感到明显的衰败走向;按照由盛而衰的总基调,有理由相信故事会一路下行,走向一个个悲惨的结局:富贵温柔,只是尘世一梦。我甚至觉得按照前八十回下行的惯性,可能全书一百回就结束了,达不到刘心武的108回,张之的110回,更没有程高本的120回。传说曹雪芹写作《红楼梦》泪尽而逝,想来竟是有可能的——读者看来红楼闺秀们尚且真切鲜活,作者更会觉得她们如同活在自己的生活中一样;那么每想一次她们的结局,便为之心痛一次。作家写作一个故事,其中情节不知要在脑中“排演”多少次。由此想来,如何不令人悲恸!
\par 宝钗是封建礼教的代言者,加之对金钏儿之死的冷漠(高中时读到周先慎《琐碎中有无限烟波》的分析),我并无好感。黛玉伶牙俐齿,才华值得欣赏。人民文学出版社出了一套黛玉怼人语录书签,我买了其中两款:“咱们只管乐咱们的”、“我就不能一目十行吗”。我觉得黛玉教香菱学诗说的话更有意思:“什么难事,也值得去学!我虽不通,大略还教得起你。”相比之下我更喜欢湘云、宝琴,除才华外,还有些活泼率真的性格。
\par \rightline{2022年11月26日}

\section{中国现当代文学}

\subsection*{野草}
\addcontentsline{toc}{subsection}{野草}
\par \emph{鲁迅} 
\par 鲁迅大概是中小学课本选入篇目最多的作家之一了。当时学《从百草园到三味书屋》《社戏》《雪》《藤野先生》,以至后来的《孔乙己》《祝福》,并未觉有何格外伟大之处。直到大学国文读《铸剑》里的“哈哈爱兮歌”,始觉这奇文远非常人所能为;又读令人毛骨悚然的《墓碣文》:“待我成尘时,你将见我的微笑”,只得叹服——终究是第一流的文字。
\par 这学期旁听现代文学史课程,讲鲁迅时,教授从《野草》出发解读鲁迅的人生哲学和思想,讲演对书中多数篇目都有涉及。《野草》的作品大多篇幅短小,一些篇目我随堂即上网查找读完,于是干脆找来全书通读。不少篇目都有浓厚的象征意味,而意象的格调总体是阴暗、残酷的(甚至不乏血腥、恐怖);语言则不乏深刻、有力的句子,如饱含激情的演讲。当然也有篇目如《立论》《我的失恋》,让我了解到鲁迅创作的多样性,读到时忍不住当“笑话”转发给朋友。
\par 除之前读过的《墓碣文》,读来觉得格外欣赏的篇目还有《影的告别》《过客》《这样的战士》《淡淡的血痕中》。《影的告别》关注了“影子”这一意象,“黑暗会吞并我,光明又会是我消失”,很是巧妙。《这样的战士》则具有讽刺意味和震撼人心的力量。《过客》采用剧本体裁,当老人谈及前方的坟地,小孩子关注的则是野百合、野蔷薇,让我很有感触。《淡淡的血痕中》放到现在依然很有警醒意义:网络环境下的公众会震惊、愤慨,但似乎是健忘的;他们会追逐新的热点,而真的猛士“记得一切深广和久远的苦痛”。
\par \rightline{2021年8月12日}

\subsection*{围城}
\addcontentsline{toc}{subsection}{围城}
\par \emph{钱钟书} 
\par 《围城》称得上是一部奇书。书中故事是知识阶层青年最为平常的工作、恋爱、家庭、社交经历,也并无什么新奇的想象和夸张,却写得极为细致和真实(特别是人物的心理活动)。此书的另一特点是其连串的俏皮话,不时令人捧腹。比如李梅亭在三闾大学讲授“先秦小说史”,此课可谓内容充实。
\par 初读此书,认为不过是一部幽默小说,作者借以讽刺种种想讽刺的人和事,包括新诗人、旧体诗人、炫才者、贪图小利者、爱慕虚荣者…… 不想越读越觉得沉重,到最后一章达到顶峰:三闾大学中的勾心斗角尚可平心而观,一对知识阶层的青年夫妇,相遇尚存些许美好,终于在两个大家庭的夹缝中失和。这不由得引发我对人性的思考。
\par 大学的一位先生曾在课上引述书中的话:“你是个好人,但没用。”(赵辛楣评价方鸿渐的原话是“你不讨厌,可是全无用处。”)以此告诫我们。当时我不以为然,正所谓“无用为大用”。后来渐渐觉得,这“用处”或许是指人在社会网络中的所谓“势力”,即财力、社会关系以及个人才能的总和。没有谁要求一个人“有用”,但“有用”可以少些依附、嘲讽,生活得舒服些;或者需要自己和亲人少些攀比之心、虚荣之心,“安贫乐道”。
\par 我以为方鸿渐最严重的弱点在于没主意。游学漫无目的,职业只等亲友相助,追求唐晓芙或许是唯一的例外。
\par \rightline{2021年8月18日}

\subsection*{南行记}
\addcontentsline{toc}{subsection}{南行记}
\par \emph{艾芜} 
\par 2021年底读了这本不足十万字的短篇小说集,算是最后冲了一下KPI。八篇小说以上世纪二十年代滇缅一带的漂泊生活为背景,都用第一人称而少有外部视角的评述,读来如记述亲身经历的非虚构作品。得知此书缘于一本短篇小说选集中的《山峡中》,现在看仍是集子中艺术成就最高的一篇,奇险的自然环境和特殊的一群人构成的场域,令人印象深刻。《洋官与鸡》描绘洋官搜刮民脂民膏的丑恶嘴脸,也十分生动。主人公虽是漂泊于社会底层的流浪者,却表现出可贵的正义感(《我诅咒你那么一笑》)、反思性与革命性(《松岭上》)、坚韧精神(《人生哲学的一课》),传递着昂扬向上的精神力量。
\par \rightline{2022年1月1日}

\subsection*{俗世奇人(二、三)}
\addcontentsline{toc}{subsection}{俗世奇人(二、三)}
\par \emph{冯骥才} 
\par 中学时读冯骥才的小说集《俗世奇人》,缘于课本所选《刷子李》等篇,书中也的确有不少精彩篇目。近年来冯骥才连推续作,却未能摆脱“一部不如一部”的魔咒,让人读完印象深刻的篇目减少了。个人觉得写市井奇人绝技比较出色的有第二册的《四十八样》,以及第三册的《白四爷说小说》。此外第二册《张果老》讲了一个精心设计的骗局,揭示了收藏者追求收藏品完整(集齐一套)的弱点,很是有趣。第三册《十三不靠》中的汪无奇,则称得上是一个让人拍案叫绝的人物。
\par \rightline{2020年12月8日}


\section{外国文学:中世纪及以前}

\section{外国文学:文艺复兴至十八世纪}

\subsection*{哈姆雷特}
\addcontentsline{toc}{subsection}{哈姆雷特}
\par \textbf{Hamlet}
\par \emph{William Shakespeare/朱生豪译}
\par 把一个涉世未深的青年推向政治斗争的漩涡,这命运是颇为残酷的;更何况丹麦王子哈姆雷特所要面对的是心狠手辣的僭主和不得不报的深仇。能勇敢地面对残酷的命运和黑暗的现实,已经十分不易;而斗争本身又是难以预知的棋局,稍有不慎便会身死人手,亦无法指望胜利者书写历史的公正。倘若真的遇到了这样的命运,也只有动用全部的胆识和智谋,向着漩涡而去了。在吕氏的威权下韬光养晦的汉文帝刘恒,继位时只有23岁;古罗马的屋大维登上政治舞台,和安东尼、雷必达争夺权力,也不过二十出头。命运会成就英雄,失败者也留下了为后人感叹的故事。哈姆雷特假充疯癫掩饰内心的愤怒,又在被押往英国的路上勇敢逃脱,都值得称赞;演戏影射之举虽有打草惊蛇之嫌,却也借此看清了国王的真面目。最后的同归于尽结局,更是可悲可叹。
\par 宗教对来世的描述影响着剧中人对生死的看法,从无神论者视角看来,国王祈祷时动手再好不过了。
\par \rightline{2022年2月16日}

\subsection*{罗密欧与朱丽叶}
\addcontentsline{toc}{subsection}{罗密欧与朱丽叶}
\par \textbf{Romeo and Juliet}
\par \emph{William Shakespeare / 朱生豪译} 
\par 这是我读的第一部莎士比亚剧作。2017年11月我曾在百讲看过TNT剧院演这部剧。记得出场后,我便问看过原著的同伴:那种可以装死四十二小时的药,原著真是这样写的吗?
\par 看书的时候,我笑着跟老爸讲,Capulet虽为贵族家长,还是会在宴会上说出那样的粗鄙之语。Romeo和Juliet的对白无疑是戏剧化的,若恋人真的要这样聊天,那不得累死。不过最深的感触和看演出时是一致的:那个时候的贵族衣食无忧,又不必有自己的志业。他们谈论爱情时,很轻易地谈到生死,并说出“爱情价更高”。相比之下,现代人的生活就丰富得多了。
\par \rightline{2020年2月23日}


\section{外国文学:十九世纪}

\subsection*{欧也妮·葛朗台}
\addcontentsline{toc}{subsection}{欧也妮·葛朗台}
\par \textbf{Eugénie Grandet}
\par \emph{Honoré de Balzac / 李恒基译} 
\par 小说的故事情节并无特别的新意——对金钱和地位的狂热追逐,穿插在“痴情女子负心汉”的感情线中。类似题材的作品中,男女主角地位交换的情况在文学作品中也有出现,如皮兰德娄的《西西里柠檬》,可与之对读。
\par 值得一提的是,作者常常中断情节的叙述,置身事外对故事中人物做一番评论和分析,而这些评论多有引人共鸣之处。作者不仅构造故事,还探讨社会的现象、人的精神的现象,并试图分析其原因,从中可以看出其风俗(社会现状)——哲理(原因)——分析(原则)三步框架的影子。我想这也是本书得以成为名著的重要原因。
\par 小说中的葛朗台老爹被刻画为一个受金钱异化到无以复加地步的典型形象。人类文明发展进步了近两百年,面对金钱的诱惑,我们的表现却似乎并没有多少长进。他们说看透了这世界,其实常常只是看到了表象之上的谎言:因为金钱是手段而非目的,并且很多东西如果一定要标个价,那会是阿列夫零。
\par \rightline{2021年4月24日}

\section{外国文学:二十世纪至今}

\subsection*{二十世纪外国短篇小说精选}
\addcontentsline{toc}{subsection}{二十世纪外国短篇小说精选}
\par \emph{王向远选编}

\par 这是人民文学版语文必读书目中的一本,一直以来被我当作火车上打发时间的读物,断断续续看了数年之久。
\par 本书所选篇目中,我最欣赏的是马赛尔·埃梅的《穿墙记》,其构思之精彩、叙述之生动堪称典范,作为穿墙者的小人物煊赫一时,被困墙中的形象则留下悲凉的余味。其次是泰戈尔的《喀布尔人》,其语言亲切温暖,是一曲跨越种族的人性之美的高歌。
\par 一篇小说如果没有奇妙的构思、细致的刻画,其主题再深刻,也只是一个空架子。卡夫卡的《判决》对心理的刻画细致入微,卡尔维诺的《阿根廷蚂蚁》中对各种灭蚁装置的设想更是让我惊叹,体现了虚构场景下讲出逼真故事的高超能力。我想即使不知道作者,读这两篇文本也能看出大家的功力。
\par 亨利·巴比塞的《十字勋章》和萧伯纳的《皇帝与小姑娘》同为反战题材,前者基调沉重,后者则以幽默、讽刺的笔触叙事,都具有强烈的感染力。博尔赫斯的《小径分岔的花园》文字带些意识流色彩,将时间分岔(平行宇宙)的设想嵌入一个构思奇巧的谍战故事中。此外库尔特·冯内古特的《今天我演什么角色》,杜鲁门·卡波特的《圣诞节忆旧》,川端康成的《伊豆的舞女》,志贺直哉的《清兵卫与葫芦》,都是值得一读的佳作。
\par \rightline{2022年8月25日}

\subsection*{老人与海}
\addcontentsline{toc}{subsection}{老人与海}
\par \textbf{The Old Man and the Sea}
\par \emph{Ernest Hemingway / 孙致礼译} 
\par 这篇小说情节算不上扣人心弦,但能把老人一段出海打鱼的经历写得这么长,实属不易。读到最后人们称道鱼骨架之大时,我才感到一丝悲壮:用“硬汉”的气概与命运抗争的人,纵使功业未成,依然令人肃然起敬。这一点让我想起《布兰诗歌》里的“O Fortuna”。
\par \rightline{2020年2月16日}

\subsection*{诺贝尔的囚徒}
\addcontentsline{toc}{subsection}{诺贝尔的囚徒}
\par \textbf{Cantor’s Dilemma}
\par \emph{Carl Djerassi / 黄群译} 
\par 这是王骏教授在《自然辩证法》课上提到的一本书。作者本人是一位知名的化学教授,退休后致力于用文学作品帮助公众了解“科学江湖”的文化。的确,科研团队的生态较少成为文学作品表达的对象,因此我读此书颇有兴致。事实上,此书涉及的主要是生物学或化学领域的实验室生态,它与我所在领域的情形有一定差异。小说引入了一些与主题关联较弱的人物情感元素,使我读来有些网文的感觉。情节方面,康托和杰里的研究在缺乏重复实验验证的情况下短时间内获得诺贝尔奖,应该说是一个较为牵强的地方。
\par 作者在后记中说:“发表论文、优先权、作者的排序、杂志的选择、大学的终身教职、资助的申请、诺贝尔奖……这些是当代科学的灵魂和包袱。”小说对科学文化中类似议题做了十分细致的刻画,如杰里在领取诺贝尔奖时如何表明自己的贡献。
\par 科学发现的可重复性,是科学界的重要议题之一。前不久国内学界发生了一场围绕G蛋白偶联受体相关研究可重复性的论争,可谓是让人大开眼界。本书中实验的可重复性构成了康托最大的忧虑,也成为克劳斯要挟康托的手段之一。相比之下,我所在的偏数据科学的领域,重复实验就是跑一跑作者公开的代码那样简单,这看来是一件好事。
\par 以R.K.默顿为代表的科学社会学理论是《自然辩证法》课的主要内容之一,其中一个核心观点是“科学家的工作是为了获得同行的承认”,包括奖章、奖金、Fellow头衔在内的各种科学奖励都是同行承认的表现形式。这部小说也在试图宣传这一观点。而在我看来,如果把论文篇数、引用数以及各类科学奖励视为从事科学研究的全部动力,那无疑是另一种异化。
\par \rightline{2021年2月14日}

\subsection*{雪国}
\addcontentsline{toc}{subsection}{雪国}
\par \emph{川端康成/ 叶渭渠译}
\par 这是一部“弱情节”的小说,作者的着力点在于日常场景的叙述和情境的刻画。从开篇车窗玻璃内外风景与人的叠加,到结尾火光之上的银河,作者精心营造的意境,读来的确让人印象深刻。不过从个人角度,这类“散文化”的小说相比于情节清晰紧凑的小说,并不受到我的偏爱。小说写岛村与艺伎的朦胧恋情,从“美的徒劳”写到美的毁灭,这其中是否包含着对人世的感伤?
\par \rightline{2021年12月4日}

\subsection*{情书}
\addcontentsline{toc}{subsection}{情书}
\par \textbf{Love Letter}
\par \emph{岩井俊二 /  穆晓芳译}
\par 今年520那天和女朋友看了重映的同名电影。电影简单而动人的情节打动了我,也给那晚的约会增添了一些淡淡的忧伤。我很快找到了这本书买来看,用了和看电影差不多的时间就读完了。人物对话占了小说的大部分篇幅,使人感觉小说只是在以类似电影剧本的方式叙事;这也使我觉得电影对这个故事的表达更为成功。
\par 在情节构思方面,同名同姓的设定自有其巧妙之处。而在主题方面,这个故事可以说是爱和生死的交织。从中学时代朦胧的感情,到青年时期热烈的爱恋;图书管理员、玻璃工匠、油画学徒……故事的主人公们并非什么成功人士,但正是众多平凡人物的生活中,存在着有关“爱”的动人故事啊。长眠雪山的登山者,让我想到白银百公里越野赛意外逝去的跑者们,令人叹惋。面对高烧四十度的藤井树(女),在家等救护车还是冒着风雪护送病人,又该是何等艰难的抉择啊。在山间小屋的火锅旁,当生者追忆逝者的往事,幸存者为了更多人的生命而坚守,影片的配乐深沉却又澄澈,大概是我最喜欢的片段;一句“你好吗?我很好”,又似乎用爱情为交织的种种作了简单的解答:爱可以跨越有机界与无机界,独立于时间箭头的延伸,乃至“超越永恒”。
\par \rightline{2021年7月11日}

\section{科幻、奇幻文学}

\subsection*{平面国}
\addcontentsline{toc}{subsection}{平面国}
\par \textbf{Flatland: A Romance of Many Dimensions}
\par \emph{Edwin Abbott Abbott /  陈凤洁译} 
\par 终于读完高中时同学推荐的这部“数学幻想小说”。其中的数学内涵现在看来是基本的,高维立方体和高维球已经是大一的议题了。不过小说中提到,从三维空间能看到平面国物体的内部,由此类比,在四维空间中能看到我们日常所见的三维物体的内部,这一点我之前没有想到过。
\par 小说有一段振奋人心的献词,鼓舞人们探求更高维空间的奥秘。我想起一位专长于几何的数院同学告诉我的话:“对于三维空间中的几何学,你可以看《曲线与曲面的微分几何》;但你要是想了解更高维空间发生的事情,就要先学拓扑学作为基础。”因而我读到这一段时,竟感到心潮澎湃。
\par \rightline{2020年2月12日}

\subsection*{哈利波特与凤凰社}
\addcontentsline{toc}{subsection}{哈利波特与凤凰社}
\par \textbf{Harry Potter and the Order of Phoenix}
\par \emph{J.K.Rowling / 马爱农、马爱新译} 
\par 在2017年暑假终于读完《纳尼亚传奇》后,我开始读《哈利波特》系列。比起同龄人,20岁开始读这七本书显得有些晚。小学阅读课上,曾有一名同学借给我《哈利波特与魔法石》看,我看了十几页似乎觉得并没什么意思?而我渐渐发现罗琳构建的这个魔法世界对我们这一代人,实在有着广泛的影响。初入大学,一位同学在自己的创意画上写下“Accio GPA”的宣言;我也曾见到有人对着宿舍的门大喝一声“Alohomora”。高中时一位热心哈迷向大家介绍了Pottermore网站(现在叫Wizarding World),其中有一个“分院测试”项目。我最喜欢的学院是Ravenclaw,但分院测试的结果是Hufflepuff,想来那的确更符合自己的特质。
\par 言归正传。七部小说的故事情节,逐渐从校园生活转变为残酷的战争;《哈利波特与火焰杯》中伏地魔的复活,可谓是一个转折点。在《哈利波特与凤凰社》的神秘事务司之战中,哈利和同学们与食死徒展开真正的对抗,而D.A.是他们得以这样做的基础。我觉得D.A.的创建和活动,实在是这部书的Highlight,正如现实生活中那些能定期开展活动、成员自发参与的学生组织,常常给参与者带来真挚的友谊和难忘的回忆。
\par Fred和George Weasley,通过一个“飞向自由”的情节,把一直以来表现出的性格和志向发挥到了极致。新出场的Luna Lovegood也是我比较喜欢的一个角色,她有着鲜明的不同寻常的特质,令人印象深刻。
\par \rightline{2020年3月13日}

\subsection*{哈利波特与“混血王子”}
\addcontentsline{toc}{subsection}{哈利波特与“混血王子”}
\par \textbf{Harry Potter and the Half-Blood Prince}
\par \emph{J.K.Rowling / 马爱农、马爱新译} 
\par 田先生在他的课上说,《哈利波特》绝不仅仅是一部魔幻小说,它有着深刻的现实意义。快要读到系列尾声的时候,我想起了这句话。与伏地魔的两次战争被称为“第一次巫师战争”和“第二次巫师战争”;目前人类历史上被冠之以“世界大战”的战争,也恰好是两次。伏地魔和食死徒,邓布利多和凤凰社,这两个阵营有什么鲜明的差异呢?伏地魔利用恐怖统治他人,这一点在德拉科身上体现得很突出:他虽声称成功杀死邓布利多将在食死徒中获得无上的荣耀,却同样在天文塔上说出伏地魔以杀死全家相逼。此外,伏地魔及其党羽宣扬纯血统巫师的优越,这在人类历史上也似曾相识;而邓布利多一方不仅对麻瓜出身的巫师没有偏见,对于家养小精灵等其它的生命也能表示出尊重。
\par 这部小说的主线之一是邓布利多向哈利展示伏地魔的过去,并带他走上寻找伏地魔魂器的道路。他没有把这份重任托付给学校里其他资深的巫师,或是法术高强的傲罗,而是选择了哈利和他的朋友们。这一点是耐人寻味的。从寻找挂坠的过程看,有不少环节是哈利一个人难以过关的(邓布利多本人也说“跟我的力量相比,你的力量恐怕可以忽略不计”)。但从哈利在随邓布利多出发前对罗恩、赫敏的交代中,我看出他的从容镇静,这代表着他正走向成熟。“分一点给金妮,替我向她说声再见。”这部小说在校园恋情上着墨不少,而我觉得其动人之处则在于大敌当前的危险、斗争的重任给恋情带上的悲壮意味,这是一般的校园恋情所没有的。
\par 参加天文塔之战的学生,恰恰是一年前在神秘事务司抵抗食死徒的那六人。有的地方管他们叫“D.A.六杰”。为什么其他人没有来并肩作战呢?诚然,在严峻的形势下,明智的人不难明白要学习防御本领用于自卫。而主动加入与恶势力的抗争,则需要真正的勇气。读到小说结尾哈利与金妮的对话时,我想起这样的诗句:“弃身锋刃端,性命安可怀?父母且不顾,何言子与妻。名编壮士籍,不得中顾私。捐躯赴国难,视死忽如归。”
\par \rightline{2020年4月3日}

\subsection*{哈利波特与死亡圣器}
\addcontentsline{toc}{subsection}{哈利波特与死亡圣器}
\par \textbf{Harry Potter and the Deathly Hallows}
\par \emph{J.K.Rowling / 马爱农、马爱新译} 
\par 小说的前六部都写到“校园生活”,有上课、考试、魁地奇比赛这样的插曲;相比之下,最后一部的情节显得紧凑得多。第六部写完,至少有四个魂器需要处理;罗琳还嫌内容不够充实,又加入了“死亡圣器”。应该说“找齐几件东西”是幻想故事中拉长篇幅的常见技法,比如“虹猫蓝兔”的七剑合璧、《福娃》先找如意碎片,再找“精神力量”等等。这种方法容易把本来完整的情节分段化,一旦处理不当,就有长篇小说退化为短篇小说集之嫌。不过在我看来,罗琳的“找齐魂器”还是比较成功的:里德尔的日记给了第二部剧情一个更明白的解释;哈利等人摧毁的四个魂器,其方法各不相同,而且在本书中情节的节奏逐渐加快:寻找和摧毁挂坠盒已经用了一半篇幅,金杯、冠冕和蛇则是在霍格沃兹之战的硝烟中被接连摧毁的。
\par 斯内普的反转可谓是作者为读者设的一个圈套——作为双面间谍,你最后说他真正效忠哪一边都是合理的;特别是斯内普是一个“高超的大脑封闭术师”,邓布利多和伏地魔信任错了人都是可能的。斯内普在亲手杀死邓布利多后,又击伤乔治,读者怎么可能保留着他是正面人物的猜测呢?
\par 通过斯内普的故事,作者似乎在传递这样的认识:我们常常谈论选择、价值观乃至信仰,但行动时常是被情感驱使的。斯内普在自身的成长中并没有选择邓布利多一方的博爱,而是走向了伏地魔的“巫师至上”;对莉莉的爱却使他最终转变了阵营。马尔福夫妇大战中的“一切为了儿子”,不再关心他们的“主人”是不是胜利,同样出于此。现实生活中是否当真如此呢?我只知道,自己也会因为一个人而对一个城市、一门学科心向往之,再说不出它们的坏话。
\par “不要低估爱的力量。”在作者的笔下,哈利在成长中不断体会亲情、友情、师生情以及爱情的。我想,邓布利多和伏地魔的不同,不是简单的正义与邪恶、光明与黑暗。伏地魔不懂得爱,甚至随意屠戮自己的追随者,削弱自己的力量。如果巫师真的谋求统治麻瓜,当魔法遇上巫师瞧不起的先进科技,谁能获胜还是个未知数;但不去这样做的出发点是共情。作者似乎希望告诉孩子们,真正的爱将引导一个人走上正确的道路。
\par 在“第二次巫师战争”中,邓布利多无疑是那个幕后的统帅。事情的发展远远超出了“神机妙算”可以把握的范围,想来很多时候画像里的他也是见机行事吧。至于死去的人如何做到在画像里活着,难道就像模仿人思维的神经网络可以在人死后继续写博客?
\par \rightline{2020年5月15日}

\subsection*{三体:黑暗森林}
\addcontentsline{toc}{subsection}{三体:黑暗森林}
\par \emph{刘慈欣} 
\par 今年暑假开始读这本书之前,不止一位朋友和我谈到《三体》三部曲。我便称赞道:“我只读过第一部,它并不以细致的描写取胜,情节设计却真是巧妙!称得上‘高潮迭起’,从头到尾都很精彩。”朋友便说,快去看看后两部吧——相比于整个系列展示的宏大图景,第一部不过是个引子罢了。读来果然令人欲罢不能。
\par “黑暗森林”无疑是这部小说的核心。当三体入侵的危机袭来,巨大的技术差距之下,直接对抗的手段一一失败;面壁者罗辑在叶文洁的指引下悟出“黑暗森林”法则,才拥有了谈判的筹码。故事情节则是戏剧化的大起大落:当人类经过大低谷的洗礼,技术突飞猛进,两千艘威力强大的星际战舰让人类达到了乐观的顶峰;谁料一个小小的水滴让太阳系成了又一个威海卫,其惨烈情状无可言说。当其他面壁者紧锣密鼓地展开计划之时,罗辑居于北欧的世外桃源,上演一出“梦中情人走向现实”的恋爱戏,触到人内心的柔软。行文中亦不乏新奇的观点和设想,读来令人称妙。
\par 先说“黑暗森林”理论。一部小说的深刻性常来源于所涉及的有关社会、人生以及人类精神的重大问题。这部小说试图处理“宇宙社会学”这一课题,可谓格局宏大;不论是将人类世界的社会学推向宇宙,还是在科幻中引入社会学思想,都具有相当的启发性。至于“黑暗森林”理论本身,网络上有不少从两条公理、两个基本概念(猜疑链和技术爆炸)出发的反驳意见,但这些讨论不完全是纯粹、非功利的;毕竟要使发言引人注意、显示“水平”,大家不懂的东西要说好;大家能懂且说好的东西要说不行。在我看来,这一理论有其合理之处,却不能说可靠。
\par 尽管作者把星舰地球的生存死局和宇宙中的文明之争都作为黑暗森林理论的实例,在我看来它们有着本质的区别,区别在于竞争中的一个单元是否有能力在遭到反击之前彻底消灭对方。在星舰地球的场景中,次声武器可以彻底消灭一艘星际战舰上的全部人员,一旦成功,再不会受到威胁。这种情形类似于几个人被困在封闭环境(如荒岛、洞穴),生存资源有限的场景,可以设想类似于《饥饿游戏》的恐怖事件会发生,只不过这里的竞争单元由战舰换成了个体。彼得·萨伯讨论的法哲学公案《洞穴奇案》情境与之类似,不过几个人没有直接诉诸暴力,而是试图进行协商。而宇宙中的文明情形就不同了。我们是否可以把文明与单一星球或者太阳系这样的系统等同起来呢?答案是否定的,就连面临三体危机的地球都保留了星舰地球这样的分支;具备星际远航能力的三体文明乃至更高文明,建立多星球的太空帝国完全是可能的。如果是这样,我们还能遵循黑暗森林的原理,对探测到的文明直接“消灭之”吗?如果是袭击了一个庞大帝国的珍珠港,进而招致大规模的反击,这样的行动能称得上符合“生存是文明的第一需要”吗?一个类似的情景是后核武器时代地球上的国家之争,就算将来有一天出现资源危机,要攻击一个战略纵深较广的有核国家,恐怕也要掂量掂量可能的反击吧。所以,真正的宇宙文明图景,很可能不像刘慈欣设想得那么简单。
\par 再说“面壁计划”。破壁人从面壁者的行为中推测战略意图,犹如一场格局宏大的解谜游戏,作为一个推理解谜爱好者,读来也是十分过瘾。不过我觉得至少对于泰勒、雷迪亚兹而言,他们的计划于事无益却消耗大量资源,由此看来ETO的破壁也是帮了地球人的忙。
\par 除罗辑外,章北海算得上是另一个成功的面壁者,尽管他保全的“自然选择”号、以及他本人都毁灭于黑暗战役,地球文明的火种的确在他的精心策划下实现了逃亡。抛开他的战略不谈,读到“看不透”的人,显示着坚定信念的目光,似乎觉得有种特别的魅力。
\par \rightline{2021年12月19日}

\subsection*{三体:死神永生}
\addcontentsline{toc}{subsection}{三体:死神永生}
\par \emph{刘慈欣} 
\par 从我翻开这《地球往事》终章的那一天起,晚上回到宿舍继续阅读就成了我每天的期待。短短十二天时间我就读完了全书。在我看来,《黑暗森林》的艺术成就和《三体》相当,我甚至更欣赏《三体》中现实与游戏交织的结构;而《死神永生》以更为宏大的时空跨度、更为惊人的奇妙想象超越了前两部,可以说是我读过的最杰出的科幻小说(当然这个评价没什么分量,因为读过的还不多)。
\par 这部小说分为六部,是一部地球文明的史诗。前半部分承接《黑暗森林》,交代了黑暗森林威慑和“蓝色空间”号的结局;后半部分则是人类防备黑暗森林打击失败,太阳系毁灭的悲剧。不同于一些叙事视角仅限于若干主角或一个群体的小说,作者能出色地驾驭整个地球文明的宏大视角,能对国际社会面对各种事态的反应做出合理可信的设想;对主角心理、行为的刻画,又不失细致入微。
\par 《三体》三部曲各有一个核心的“科学概念”,作者分别在小说中展现了自己对它们的思考和想象。第一部是“三体问题”,第二部是“宇宙社会学”,第三部则是“维度”。在这部小说中,作者继承并发扬了Edwin Abbott Abbott《平面国》的精神:开篇女魔法师摘取大脑的情节,让人联想到对球的无所不知大为惊异的正方形。 对“蓝色空间”号乘员进入四维空间的描写尽管谈不上精妙,却也符合二、三维空间之间关系的类推。小说进一步设想,高维空间可以在低维空间中以碎块的形式存在;低维生物甚至可以进入高维空间并保持生存。这一点初看让人难以接受,三维空间怎么能和四维空间衔接呢?三维生物由三维基本粒子构成,如何能在四维空间的粒子中安然存在呢?细细想来,又无法断然否定:考虑到维度可以卷曲在微观中,谁知道在穿越维度分界的时候究竟会发生什么呢?小说还设想生物可以主动降维,从而可以对原本同一维度的文明进行维度攻击。这又是有数学基础的:平面上[0,1]和[0,1]×[0,1]基数是一样的,也就是降维可以不损失信息。
\par 地球系统的研究中有所谓“盖娅假说”(Gaia Hypothesis),即生命对地球环境有巨大影响,且朝着对自己有利的方向进行。这部小说通过杨冬生前的思考、关一帆与程心关于星际战争的对话,提出了“生命有意识地改变宇宙规律”这一惊人假说,或可称为“宙斯假说”(Zeus Hypothesis),从而为宇宙的不和谐(光速有限、多个维度卷曲在微观,关一帆称为“三与三十万的综合征”)给出了大胆的解释。这当然是未经证实的猜测(我本人倾向其不成立),但作者没有把光速有限且恒定等现代物理的基本图景视为理所当然,让自己的想象力在宇宙规律上驰骋,这份气魄令人赞叹。至于说数学规律被改造,那不会。正如与“魔戒”的交流中,素数序列可以作为智慧生命的证明,正是由于数学规律天然的时空恒定性。
\par 在故事中嵌入诗歌等其它文体,可为小说增添艺术性,这一点做到极致的是《红楼梦》。这部小说在这方面同样值得赞赏,与四维空间中“魔戒”的对话,云天明的三个童话,歌者吟唱的古老歌谣,语言含蓄,令人回味。隐喻(metaphor)一直是文学艺术中最令我着迷的手法,这些穿插的片段都展现了隐喻的迷人魅力。
\par 继第二部的章北海、罗辑之后,又有褚岩、云天明等令人敬佩的杰出人物,为地球文明延续做出了关键性的贡献。而作者笔下的大众,则自私、软弱、短视、愚昧,与之形成鲜明对照。当引力波威慑被摧毁,众多人争当“球奸”,加入效忠于三体的“治安军”;启动引力波广播的“蓝色空间”号最初被视为英雄,三体入侵的阴云散去之后,又成了带来毁灭厄运的魔鬼;面对黑暗森林打击的威胁,大批民众诉诸宗教,当智子透露可能向宇宙发出安全声明之后, 甚至三体文明都成为了朝拜的对象,读来令人唏嘘。教会和神学家花了一个多世纪重新解释经典,使三体人的存在能够符合教义,这是何其辛辣的讽刺!
\par 歌者向太阳系抛出一片小小的二向箔,整个太阳系随之毁灭,成为一幅悲壮的巨画。这是地球文明的终曲,一幕带有浪漫色彩的悲剧;而这浪漫是黑色的,正如茫茫空宇。按照作者的设想,这巨画也难以在宇宙中保留;整个太阳系将成为令当今天文学界困惑不解的暗物质。最后一代人苦心经营掩体工程,可二向箔的维度打击又有谁能预知?
\par 在太阳系灭亡的大背景下,作者重新追问生命的意义。默斯肯岛的老杰森说:“一切都将逝去,只有死神永生。”天文学家威纳尔这样评论被毁灭的三体世界:“没有真正毁掉什么,更没有灭掉什么,物质总量一点不少都还在,……,只是组合方式变了变,像一副扑克牌,仅仅重洗而已……可生命是一手同花顺,一洗什么都没了。”这个比喻很好地解释了生命的精巧和脆弱:其存在本身就是宇宙中的奇迹。更进一步地,歌者发挥了Erwin Schrödinger的思想:在熵增的宇宙中唯有低熵体(生命体)在变得更有序,“这是最高层的意义”;要维持这种意义,生命体就必须存在和延续,这是比出于好奇心探索宇宙更高层的意义。注意到这与《朝闻道》传递的观念截然相反。在我看来,正如生命的存在降低宇宙的无序度,通过探索未知发现世界的秩序,可以降低我们认识中世界的不确定度,这便是学人求索的直接意义;此外研究对象的性质和研究过程本身可以为我们提供审美价值。但我不赞同将这种探索上升到高于存在的意义层面,特别是针对整个文明而言(如《朝闻道》中的星云生物)。
\par 除探索未知之外,还有什么可能置于比存在更高的意义层面呢?孟子说“舍生取义”,这关乎伦理道德,同样是这部小说引人思考的主题。星舰地球的生存困境带来了类似于《洞穴奇案》的道德困境;不管是危机纪元还是后来的掩体纪元,平等的价值观一定程度上成为了人类飞向外太空、广播文明火种的包袱;而黑暗森林打击、维度攻击更是以“生存”的名义完全抛弃了道德。有趣的是,小说结尾重提“责任的阶梯”,宇宙的未来让关一帆、程心做出牺牲重返太空。
\par 在《黑暗森林》的闲谈中,我提出了对黑暗森林理论的质疑。这部小说使我的思考更加清晰。像故事中的三体文明和银河系人类一样,很可能多数文明都不会局限在一个恒星系内,在掌握光速飞船技术的条件下,即使遭到打击也能部分逃脱。这意味着“清理”的作用极为有限,很可能只会暂时削弱一个文明;由于最为重要的科技思想得到保留,文明有能力在很短时间内再度产生威胁。“清理”成立还有一个前提,即来源无法追踪,这消除了反击的可能性。但在我看来,这种追踪能力并非不可能实现;一旦被掌握,“清理”伴随着巨大的潜在风险,宇宙文明将是完全不同的图景。
\par 云天明对程心的单恋故事凄婉感人;他在三体世界得以生存,又精心设计为人类传递情报,堪称可歌可泣。程心自愧于两次为人类做出了错误的选择,我以为她竞选执剑人且未能启动广播是个错误,但制止星环城战争不应苛责。两人赴约于“我们的星星”;艾AA和关一帆亦有情愫,如果一切顺利,想来是美好的结局。只可惜意外使他们的命运发生了交错,两对恋人“交叉互换”。程心“终于看清,使自己这里沙尘飘飞的,是怎样的天风;把自己这片小叶送向远方的,是怎样的大河”。在苍茫的宇宙面前,生存尚且是一种奢望,何况爱情。好在关一帆和程心小世界中的学者生活依然恬静温馨。
\par 程心著述《时间之外的往事》之时依然年轻美丽,可她已见证了地球文明从危机纪元到掩体失败四百年的波澜历史。读完三部曲的读者同样是如此,百感交集之时,程心的一段话令人震撼:
\par “每个文明的历程都是这样:从一个狭小的摇篮世界中觉醒,蹒跚地走出去,飞起来,越飞越快,越飞越远,最后与宇宙的命运融为一体。对于智慧文明来说,它们最后总变得和自己的思想一样大。”
\par 我想,这段话应当能在许许多多和我一样,以及更年轻的读者心中种下一颗种子,我们会一同仰望星空,了解天体物理和宇宙学,随着人类深空探索的前哨(如JWST)看宇宙,思索宇宙中文明的关系、宇宙的前世和未来。
\par “心事浩茫连广宇,于无声处听惊雷。”是以为记。
\par \rightline{2022年11月5日}

\section{侦探、推理、悬疑}

\subsection*{东方快车谋杀案}
\addcontentsline{toc}{subsection}{东方快车谋杀案}
\par \textbf{Murder on the Orient Express}
\par \emph{Agatha Christie / 郑桥译} 
\par 这是我读的第一本阿加莎推理小说,阅读时间在七八个小时。书的前两部平铺直叙,节奏不紧不慢。波洛对不同的乘客采用不同的讯问方式;频频用诈,在不经意间得到需要的信息,展示出高超的技巧。他与布克先生、康斯坦汀医生的对话也不时带着些反讽和冷幽默。推理进程在最后六分之一的篇幅突然加快,直到谜底揭晓,颠覆读者的认知。证词中埋下的伏笔在波洛的分析中一点点揭开;直到最后波洛阐述结论,仍有我没有注意到的问讯细节成为破案的线索,令人拍案叫绝。
\par 这本书之所以能成为阿加莎最有影响力的名作之一,首先在于其颠覆性的设计。阅读证词时,我思考的前提假设是:大部分证词是可靠的(除了一两个可能的凶手)。而作者恰恰推翻了这个前提假设:车厢上的十五位乘客和列车员,除去侦探和死者,都是案件的策划者。他们的行动和证词只是给侦探演了一个难解的故事,以此为基础推理什么时间线、不在场证明,都是推了个寂寞!而谨慎、细致的波洛从护照上的油渍入手,挖掘出了乘客隐藏的身份和作案动机。以此为基础,死者伤口的疑点就得到了解释;再回头看来,印在精装版封面上的麦奎因先生的话“这趟列车可真是座无虚席啊!”竟是整个故事的大伏笔。此外,侦探小说中的故事常常是悲剧的,其揭露的犯罪显示出人性阴暗、邪恶的一面;而这本书的谋杀却是正义的伸张,我想这也是读者愿意看到的。
\par 波洛对金钱的态度值得赞赏,在拒绝雷切特的重金委托时,他说:“我在事业上很走运,所赚的钱完全可以满足我的现实需要和各种任性的想法。我现在只接受感兴趣的案子。”
\par \rightline{2022年11月13日}

\subsection*{尼罗河上的惨案}
\addcontentsline{toc}{subsection}{尼罗河上的惨案}
\par \textbf{Death on the Nile}
\par \emph{Agatha Christie / 张乐敏译} 
\par 读完《东方快车谋杀案》后,我很快又在一个周末分两天读完了《尼罗河上的惨案》。第一天睡前担心睡不好觉去看了剧透,但即使知道凶手,读波洛的推理过程依然十分过瘾。
\par 这部小说的案情虽没有《东方快车谋杀案》那样颠覆性的构思,却也十分巧妙:西蒙和杰奎琳互相制造不在场证明,相当程度上成功转移了大家的怀疑。穿插其中的珠宝盗窃、国际危险分子,也为案件增添了不少复杂性。但波洛说得好:“清除外表的杂质,以便发现真相。”通过对每个人言行、心理细致的体察,首先解释一些无关的疑点,剩下的将指明真相。波洛用考古发掘的例子说明这个道理,我想或许是受了阿加莎考古学家丈夫的影响。杰奎琳的话同样令人回味:“每个人都得追随自己的星星,不管它引导我们走向何处。”这一无奈之语,道出了盲目、失去原则的爱引发的悲剧。她接着说“那是一颗坏星星,那颗星星会掉下来”,分别暗示了西蒙的作案蓄谋与失败结局,也是十分高明的伏笔。
\par \rightline{2022年11月20日}

\section{寓言、神话}

\subsection*{拉封丹寓言}
\addcontentsline{toc}{subsection}{拉封丹寓言}
\par \textbf{Fables}
\par \emph{La Fontaine / 苏迪译} 
\par 书中的寓言故事篇幅短小,大多只有半页纸。很多故事采用拟人手法,饶有趣味。不少生活中听到的故事在本书找到了出处。值得一提的是,我读的本子收录了Percy J. Billinghurst精美的版画插图,绝大多数都可以直接拿去印明信片,单凭这些插图也值回了买书的钱。
\par 简要总结一些令我印象深刻而不那么著名的篇目。《死神与伐木工》一篇写世人相较于死亡宁愿生而受苦,与周国平人生寓言中《落难的王子》有相通之处。《老鼠的议会》讽刺议会讨论积极却无人执行,令人捧腹。《狐狸和山羊》讨论了合作者的背叛,让我想起玩踩气球游戏的经历。(有人说,当自己的盟友突然踩向自己气球的时候,那是怎样的心理创伤啊。反观古罗马史中的政治同盟,还不是要决个你死我活吗。)在《狼和牧羊人》中,一只因残杀羊受到恶名的狼发现“自诩为羊群保护神的人类也在吃烤羊肉”,最后得出结论“狼的唯一错误在于它不是所有生命的主宰”,角度独特。《橡子和南瓜》十分滑稽。《两只鸽子》写远行的鸽子途中遭遇各种危险,使我思量自己周游世界的计划还是限于相对安全的地方为好。《要求有个国王的青蛙》道出了君主制的弊端,即难以保证君主的贤明。为数不多的几篇体现了作者对于一些议题的立场,如《掉进井里的占星家》批评了占星术的虚伪,还有一两篇意在说明“动物也会思考”。《寓言的威力》一篇在全书中有纲领性的意义,表明了创作寓言的动机:“有人说,世界衰老了;而我却认为,人们如同孩子,他们渴望那些浅显易懂、充满快乐的故事。”

\par \rightline{2021年2月6日}

\section{诗歌}

\subsection*{仓央嘉措圣歌集}
\addcontentsline{toc}{subsection}{仓央嘉措圣歌集}
\par \emph{龙冬译} 
\par 一本颇费些周折才淘到的书。对这本书的兴趣多半来自歌曲《在那东山顶上》,最初听的是谭晶《看见》的版本,后来找到玛吉阿米藏音的版本,觉得这两位藏族歌手演绎得更有韵味。译者的翻译以忠实原文为原则,比如第24首“若要附和美人心愿,恐将失去此生佛缘;若是隐居山里修行,又会背离女子芳心。”由此观之,所谓“世间安得双全法,不负如来不负卿”一定程度上是一种再创作。这本诗集并非简单的情歌,有几首作品明显表达了敌对势力对自己的恶毒中伤。作者甚至认为诗中很多情语也只是一种隐喻,对此我对背景了解不深,尚难以判断。
\par \rightline{2020年7月23日}

\section{散文、随笔}

\section{纪实、传记}

\subsection*{假如给我三天光明}
\addcontentsline{toc}{subsection}{假如给我三天光明}
\par \textbf{Three Days to See}
\par \emph{Helen Keller / 林海岑译} 
\par 《假如给我三天光明》是散文名篇,作者告诫读者“像明天将要失明那样去使用你的眼睛”,这样将看到之前从未看见过的东西。作者描述了三天光明时光的计划,观察的对象大致可归为亲友、自然、艺术、社会四个方面。我想对这个问题的回答,几乎就是回答“什么事物值得为之付出时间”,亦即“如何生活”。
\par 作者23岁时写就的自传《我生命的故事》占了全书绝大部分篇幅,介绍了自己从接受启蒙教育直至读大学的经历。印象深刻的一点是“首位成功者”的榜样作用:海伦得知挪威的盲聋女孩朗希尔顿·卡达学会了说话,立即下定学会说话的决心。而第一位以盲聋人身份获文学学士的海伦本人,也成为其他人的榜样。“首位成功者”用实际行动使可能性得到证实,其带来的希望和激励作用是巨大的。此外,海伦的成就固然离不开本人的坚强意志和刻苦努力,也离不开外部条件的支持。海伦的父母能为她聘请专业的盲人教师,海伦的朋友帮助她把需要阅读的书籍和资料制成盲文书,这不是每一个家庭都能做到的。海伦13岁时就在著名发明家贝尔的陪同下参观了世界博览会,这简直令人惊叹。海伦对大学教育节奏过快的评论引人思考,她认为“我们应该把教育视为一次乡间散步,从容不迫,敞开思想,尽情接纳天地万物。”但大学面对着在给定年限内完成人才培养、输送到社会的压力,更不必说被state-of-the-art裹挟着前进的学术界;额。海伦的观点也许只能作为终身学习者的理想。
\par 在《三论乐观》中,海伦阐释了她的乐观主义立场,她似乎认为乐观的态度对事业的成功、社会的进步是必要的,因而“有害的”悲观思想不应被传播。但一个观点是否正确和是否有利是两个问题。成年以来,我逐渐反思这种无充分依据的乐观:既然复杂系统的行为难以预测(特别是影响重大的偶发事件),真的可以说“明天会更好”吗?海伦说要相信人性,“成千上万的人一直在努力维持这世界的美好”。而我似乎觉得需要一套让个体利益与群体利益趋于一致的社会制度;个人生活得到改善,当且仅当他为维护社会美好、推动社会进步做出贡献。似乎有时候,绝大多数人“努力维持世界的美好”都是不够的。
\par \rightline{2023年1月1日}

\section{儿童文学}

\subsection*{小公主}
\addcontentsline{toc}{subsection}{小公主}
\par \textbf{A Little Princess}
\par \emph{Frances H. Burnett / 梅静译} 
\par 伯内特夫人的书,少年时曾读过《秘密花园》,很是喜爱。《小公主》接触过书虫系列的改写本,如今找到全译本来读,仍然觉得是个温暖感人的故事。女主人公萨拉是个典范的儿童形象:善良、礼貌、酷爱读书、多才多艺、富于想象;小说着力刻画的则是她不论处于富贵还是贫贱,都努力像公主一样行事,保持着一份美德和尊严。或许是读书使然(十一岁的女孩居然对法国大革命感兴趣),她有着与年龄不相称的成熟。她对学校的洗碗女仆说:“我和你一样都是小女孩,而我不是你,你不是我,只是个意外而已。”不久父亲破产、去世,她从饱受宠爱的富家女变成了一文不名的女佣,就住在洗碗女仆隔壁。势利的女校长对她极尽剥削、虐待,却没有想到一时失势的人还有东山再起的一天。阁楼宴会一幕一波三折,从现实破灭到美梦成真,亦是整个情节的缩影,具有温暖的浪漫气息。
\par \rightline{2022年12月14日}

\subsection*{窗边的小豆豆}
\addcontentsline{toc}{subsection}{窗边的小豆豆}
\par \emph{黑柳彻子 / 赵玉皎译} 
\par 好久没读儿童文学作品了,果然感觉阅读压力小了很多,可以一口气读几十页(大概是最近读论文的缘故)。这本书的版权页将其归类为“儿童文学——长篇小说”,但既然写的是真人真事,归入纪实散文或许更合适。
\par 书中的内容并不限于巴学园,但小林宗作校长的教育理念和方法应当说是全书的主线之一。我相信日常的自由活动时间、亲近自然或是接触生产实践的集体出游对小学生有益处,但对小学生能否主要通过自学奠定知识的基础抱有怀疑。巴学园的小班教学实验当然很难适应竞争激烈的当下社会,但如果高中三年完全以大学为目的,初中三年完全以高中为目的,小学六年完全以初中为目的,我想那也不是最理想的情况。
\par \rightline{2020年7月23日}

\subsection*{天蓝色的彼岸}
\addcontentsline{toc}{subsection}{天蓝色的彼岸}
\par \textbf{The Great Blue Yonder}
\par \emph{Alex Shearer / 吕良忠译} 

\par 这是一部试图阐释“死亡”这一主题的儿童文学作品,国庆假期期间约四小时读完。因车祸意外身亡的小学生哈里以幽灵的形式回到人间,先是因看到没有自己的世界如常前行而沮丧,之后从师友的纪念、亲人的悲伤中得到感动。哈里重见家人的一段,读来感人至深,让人更懂得珍视生活中的点滴,特别是和亲人朋友共度的美好时光。
\par 对于人死后的路径,书中的设定似乎是先到一个叫“他乡”的地方,之后自愿前往“天蓝色的彼岸”,代表个体意识的终结,有机体解体、回归自然。在他乡滞留的时间可以任意长,甚至能以幽灵身份重返人间(尽管不合规);这样做多是有未竟的心愿,如寻找失散的亲人。若真如此,没有挂念也何妨在人间再游荡个千百年。可赏美景,却不可品美食;可观棋,却不能下棋;能思考,却不能著述。想来难受;却也凑合。
\par \rightline{2022年10月3日}


\section{文学研究、文学史}

\subsection*{人间词话}
\addcontentsline{toc}{subsection}{人间词话}
\par \emph{王国维/ 徐调孚校注} 

\par 此书少数章节是对词的一般讨论,其余则似读词的批注合集,对一人、一篇乃至一句而发。可以看出作者读词十分广泛,除作者认可的名家李煜、冯延巳、欧阳修、苏轼、秦观、辛弃疾,书中还论及不少我看到字、号一头雾水,查出名字仍十分陌生的词人。我至今对词缺乏系统的阅读,如今读此书也只能观其大略。
\par 还记得《红楼梦》中林黛玉论诗词,说立意最为重要,词句新奇次之,格律相比之下并不要紧。 “质重于文”的观点我是赞同的,但这些讨论并未体现出词的美学内蕴。本书开篇即说“词以境界为最上”,但作者未给出“境界”概念界定,我几乎一直是按“意境”来理解“境界”的。附录中叶嘉莹的论文《〈人间词话〉之基本理论——境界说》认为本书用“境界”一词意在强调词作对感官和心理体验的真切重现。我认可这一解读,不过我以为在意境(或所谓境界)中重要的是其组成部分(景物、事物和精神体验)间的联系和交互,即“整体大于部分之和”。
\par 书中的很多其它概念也未给出明确的界定,如第三则写有我之境、无我之境,我读来就不甚理解。叶嘉莹通过考察叔本华美学对王国维的影响,辨析了有我与无我、造境与写境、主观与客观三组对立概念的差异,解得清晰。有朋友推崇模糊语言“只可意会不可言传”的丰富意蕴,认为这种概念的准确界定“化神奇为腐朽”。我却觉得十分必要,或许是思维方式的差异使然。
\par 此外各章,也有不少精妙之论。第五则说虚构之境“材料必求于自然,构造亦必从自然之法则”,正与我虚构写作“虚实相生”的理念相合。第三十四则批评替代词的使用,“意足则不暇代,语妙则不必代”。第四十四则言东坡词旷,稼轩词豪,“无二人之胸襟而学其词”有如东施效颦。第五十四则论一种文体兴盛期过后难出新意,故有主流文体的转变,我想这一观点对美术史上的流派、思潮之变同样成立。第六十则说“诗人对宇宙人生,须入乎其内,又须出乎其外;入乎其内,故能写之;出乎其外,故能观之。”读来深以为然。
\par 当然最为著名的还是被选入小学语文课本的一节,“古今成大事业、大学问之三种境界”。我想前两境界确是必经的;至于第三境界,可能有一个灵光乍现、豁然开朗的戏剧化的时刻(解决数学难题可能是很恰当的例子);但也可能是日积月累的精进,最后外人所见的成功只是水到渠成的展现。王国维在这则词话中写尽了其中的孤独与坚守、豪情与快意,引用名句却又都是“断章取义”,实在是古典诗词意涵丰富的绝好说明。
\par 王国维作《人间词》,友人“山阴樊志厚”作序大加称赞。谁知据考证这序就出自王国维本人之手,令人大开眼界。
\par \rightline{2022年12月25日}
