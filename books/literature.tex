
\section{中国古代文学}

\subsection*{三国演义}
\addcontentsline{toc}{subsection}{三国演义}
\par \emph{罗贯中} 
\par 我初中时读了《西游记》和《水浒传》;《三国演义》读了一半左右就上了高中,便搁置下来。六年多过去,这才又拿起来读;由于间隔时间太久,干脆从头读起。22岁读《三国》显得晚了些,甚至说起来有些令人惭愧,但晚读也有晚读的好处:初中时我每天强撑着完成读三回的任务,现在倒觉得其中的权谋计策有些意思,不必规定任务就津津有味的读下去了。初中时只觉得陈琳讨曹操的檄文“畅快淋漓”,如今其中影响最大的精彩片段当然是“诸葛亮骂死王司徒”了,哈哈。
\par 读此书似看戏。在历史舞台上,主公、忠臣、反贼、内奸,你方唱罢我登场。对于没有实权、说被废就被废的君主,处于弱势或战败的诸侯、武将,常常面临“生”与“义”的抉择。我常叹献帝不能掌控命运,被权臣肆意欺凌,还不如效仿高贵乡公曹髦以卵击石,有尊严地死去。至于臣子,有的宁死不降,有的“择良木而栖”,倒都可理解,毕竟内部矛盾不涉及民族大义。最惨的是“生”“义”二者皆失,如荆州刘琮,拱手投降,被诛于赴任路上。而在残酷的政治斗争中,一些小人物难以左右大局,却一样做了牺牲品。比如那些送信的使节就令人同情:有时还不知道信里写的是啥,就被对方“怒斩来使”。
\par “均势”是我的另一点体会。魏国最强,因而孙刘联盟,魏伐蜀则吴伐魏,魏伐吴则蜀伐魏,这才得以成鼎足之势。这种“均势”在现当代国际政治中也屡见不鲜。在一国之内,君臣之间也存在着一种“均势”:曹操、刘备、孙权能统领手下文臣武将,而他们的后继者有的没有那样的雄才,以致手下功高盖主,大权在握。掌握重权的臣子,若没有诸葛亮那样的尽忠之心,行废立、谋篡位便是常见的事了。
\par 小说作为封建时代的产物,竭力渲染“天命”“神鬼”:大将身死之前,必有“将星坠于野”;孔明、关羽死后显灵;甚至出现了管辂教人求神仙改寿数的情节。要是真有一个数据库保存着每个人的寿数,那岂不是会有大量程序员毕生致力于入侵那个数据库?
\par \rightline{2020年7月6日}

\section{中国现当代文学}

\subsection*{野草}
\addcontentsline{toc}{subsection}{野草}
\par \emph{鲁迅} 
\par 鲁迅大概是中小学课本选入篇目最多的作家之一了。当时学《从百草园到三味书屋》《社戏》《雪》《藤野先生》,以至后来的《孔乙己》《祝福》,并未觉有何格外伟大之处。直到大学国文读《铸剑》里的“哈哈爱兮歌”,始觉这奇文远非常人所能为;又读令人毛骨悚然的《墓碣文》:“待我成尘时,你将见我的微笑”,只得叹服——终究是第一流的文字。
\par 这学期旁听现代文学史课程,讲鲁迅时,教授从《野草》出发解读鲁迅的人生哲学和思想,讲演对书中多数篇目都有涉及。《野草》的作品大多篇幅短小,一些篇目我随堂即上网查找读完,于是干脆找来全书通读。不少篇目都有浓厚的象征意味,而意象的格调总体是阴暗、残酷的(甚至不乏血腥、恐怖);语言则不乏深刻、有力的句子,如饱含激情的演讲。当然也有篇目如《立论》《我的失恋》,让我了解到鲁迅创作的多样性,读到时忍不住当“笑话”转发给朋友。
\par 除之前读过的《墓碣文》,读来觉得格外欣赏的篇目还有《影的告别》《过客》《这样的战士》《淡淡的血痕中》。《影的告别》关注了“影子”这一意象,“黑暗会吞并我,光明又会是我消失”,很是巧妙。《这样的战士》则具有讽刺意味和震撼人心的力量。《过客》采用剧本体裁,当老人谈及前方的坟地,小孩子关注的则是野百合、野蔷薇,让我很有感触。《淡淡的血痕中》放到现在依然很有警醒意义:网络环境下的公众会震惊、愤慨,但似乎是健忘的;他们会追逐新的热点,而真的猛士“记得一切深广和久远的苦痛”。
\par \rightline{2021年8月12日}

\subsection*{围城}
\addcontentsline{toc}{subsection}{围城}
\par \emph{钱钟书} 
\par 《围城》称得上是一部奇书。书中故事是知识阶层青年最为平常的工作、恋爱、家庭、社交经历,也并无什么新奇的想象和夸张,却写得极为细致和真实(特别是人物的心理活动)。此书的另一特点是其连串的俏皮话,不时令人捧腹。比如李梅亭在三闾大学讲授“先秦小说史”,此课可谓内容充实。
\par 初读此书,认为不过是一部幽默小说,作者借以讽刺种种想讽刺的人和事,包括新诗人、旧体诗人、炫才者、贪图小利者、爱慕虚荣者…… 不想越读越觉得沉重,到最后一章达到顶峰:三闾大学中的勾心斗角尚可平心而观,一对知识阶层的青年夫妇,相遇尚存些许美好,终于在两个大家庭的夹缝中失和。这不由得引发我对人性的思考。
\par 大学的一位先生曾在课上引述书中的话:“你是个好人,但没用。”(赵辛楣评价方鸿渐的原话是“你不讨厌,可是全无用处。”)以此告诫我们。当时我不以为然,正所谓“无用为大用”。后来渐渐觉得,这“用处”或许是指人在社会网络中的所谓“势力”,即财力、社会关系以及个人才能的总和。没有谁要求一个人“有用”,但“有用”可以少些依附、嘲讽,生活得舒服些;或者需要自己和亲人少些攀比之心、虚荣之心,“安贫乐道”。
\par 我以为方鸿渐最严重的弱点在于没主意。游学漫无目的,职业只等亲友相助,追求唐晓芙或许是唯一的例外。
\par \rightline{2021年8月18日}

\subsection*{南行记}
\addcontentsline{toc}{subsection}{南行记}
\par \emph{艾芜} 
\par 2021年底读了这本不足十万字的短篇小说集,算是最后冲了一下KPI。八篇小说以上世纪二十年代滇缅一带的漂泊生活为背景,都用第一人称而少有外部视角的评述,读来如记述亲身经历的非虚构作品。得知此书缘于一本短篇小说选集中的《山峡中》,现在看仍是集子中艺术成就最高的一篇,奇险的自然环境和特殊的一群人构成的场域,令人印象深刻。《洋官与鸡》描绘洋官搜刮民脂民膏的丑恶嘴脸,也十分生动。主人公虽是漂泊于社会底层的流浪者,却表现出可贵的正义感(《我诅咒你那么一笑》)、反思性与革命性(《松岭上》)、坚韧精神(《人生哲学的一课》),传递着昂扬向上的精神力量。
\par \rightline{2022年1月1日}

\subsection*{俗世奇人(二、三)}
\addcontentsline{toc}{subsection}{俗世奇人(二、三)}
\par \emph{冯骥才} 
\par 中学时读冯骥才的小说集《俗世奇人》,缘于课本所选《刷子李》等篇,书中也的确有不少精彩篇目。近年来冯骥才连推续作,却未能摆脱“一部不如一部”的魔咒,让人读完印象深刻的篇目减少了。个人觉得写市井奇人绝技比较出色的有第二册的《四十八样》,以及第三册的《白四爷说小说》。此外第二册《张果老》讲了一个精心设计的骗局,揭示了收藏者追求收藏品完整(集齐一套)的弱点,很是有趣。第三册《十三不靠》中的汪无奇,则称得上是一个让人拍案叫绝的人物。
\par \rightline{2020年12月8日}


\section{外国文学:中世纪及以前}

\section{外国文学:文艺复兴至十八世纪}

\subsection*{罗密欧与朱丽叶}
\addcontentsline{toc}{subsection}{罗密欧与朱丽叶}
\par \textbf{Romeo and Juliet}
\par \emph{William Shakespeare / 朱生豪译} 
\par 这是我读的第一部莎士比亚剧作。2017年11月我曾在百讲看过TNT剧院演这部剧。记得出场后,我便问看过原著的同伴:那种可以装死四十二小时的药,原著真是这样写的吗?
\par 看书的时候,我笑着跟老爸讲,Capulet虽为贵族家长,还是会在宴会上说出那样的粗鄙之语。Romeo和Juliet的对白无疑是戏剧化的,若恋人真的要这样聊天,那不得累死。不过最深的感触和看演出时是一致的:那个时候的贵族衣食无忧,又不必有自己的志业。他们谈论爱情时,很轻易地谈到生死,并说出“爱情价更高”。相比之下,现代人的生活就丰富得多了。
\par \rightline{2020年2月23日}


\section{外国文学:十九世纪}

\subsection*{欧也妮·葛朗台}
\addcontentsline{toc}{subsection}{欧也妮·葛朗台}
\par \textbf{Eugénie Grandet}
\par \emph{Honoré de Balzac / 李恒基译} 
\par 小说的故事情节并无特别的新意——对金钱和地位的狂热追逐,穿插在“痴情女子负心汉”的感情线中。类似题材的作品中,男女主角地位交换的情况在文学作品中也有出现,如皮兰德娄的《西西里柠檬》,可与之对读。
\par 值得一提的是,作者常常中断情节的叙述,置身事外对故事中人物做一番评论和分析,而这些评论多有引人共鸣之处。作者不仅构造故事,还探讨社会的现象、人的精神的现象,并试图分析其原因,从中可以看出其风俗(社会现状)——哲理(原因)——分析(原则)三步框架的影子。我想这也是本书得以成为名著的重要原因。
\par 小说中的葛朗台老爹被刻画为一个受金钱异化到无以复加地步的典型形象。人类文明发展进步了近两百年,面对金钱的诱惑,我们的表现却似乎并没有多少长进。他们说看透了这世界,其实常常只是看到了表象之上的谎言:因为金钱是手段而非目的,并且很多东西如果一定要标个价,那会是阿列夫零。
\par \rightline{2021年4月24日}

\section{外国文学:二十世纪至今}

\subsection*{老人与海}
\addcontentsline{toc}{subsection}{老人与海}
\par \textbf{The Old Man and the Sea}
\par \emph{Ernest Hemingway / 孙致礼译} 
\par 这篇小说情节算不上扣人心弦,但能把老人一段出海打鱼的经历写得这么长,实属不易。读到最后人们称道鱼骨架之大时,我才感到一丝悲壮:用“硬汉”的气概与命运抗争的人,纵使功业未成,依然令人肃然起敬。这一点让我想起《布兰诗歌》里的“O Fortuna”。
\par \rightline{2020年2月16日}

\subsection*{诺贝尔的囚徒}
\addcontentsline{toc}{subsection}{诺贝尔的囚徒}
\par \textbf{Cantor’s Dilemma}
\par \emph{Carl Djerassi / 黄群译} 
\par 这是王骏教授在《自然辩证法》课上提到的一本书。作者本人是一位知名的化学教授,退休后致力于用文学作品帮助公众了解“科学江湖”的文化。的确,科研团队的生态较少成为文学作品表达的对象,因此我读此书颇有兴致。事实上,此书涉及的主要是生物学或化学领域的实验室生态,它与我所在领域的情形有一定差异。小说引入了一些与主题关联较弱的人物情感元素,使我读来有些网文的感觉。情节方面,康托和杰里的研究在缺乏重复实验验证的情况下短时间内获得诺贝尔奖,应该说是一个较为牵强的地方。
\par 作者在后记中说:“发表论文、优先权、作者的排序、杂志的选择、大学的终身教职、资助的申请、诺贝尔奖……这些是当代科学的灵魂和包袱。”小说对科学文化中类似议题做了十分细致的刻画,如杰里在领取诺贝尔奖时如何表明自己的贡献。
\par 科学发现的可重复性,是科学界的重要议题之一。前不久国内学界发生了一场围绕G蛋白偶联受体相关研究可重复性的论争,可谓是让人大开眼界。本书中实验的可重复性构成了康托最大的忧虑,也成为克劳斯要挟康托的手段之一。相比之下,我所在的偏数据科学的领域,重复实验就是跑一跑作者公开的代码那样简单,这看来是一件好事。
\par 以R.K.默顿为代表的科学社会学理论是《自然辩证法》课的主要内容之一,其中一个核心观点是“科学家的工作是为了获得同行的承认”,包括奖章、奖金、Fellow头衔在内的各种科学奖励都是同行承认的表现形式。这部小说也在试图宣传这一观点。而在我看来,如果把论文篇数、引用数以及各类科学奖励视为从事科学研究的全部动力,那无疑是另一种异化。
\par \rightline{2021年2月14日}

\subsection*{雪国}
\addcontentsline{toc}{subsection}{雪国}
\par \emph{川端康成/ 叶渭渠译}
\par 这是一部“弱情节”的小说,作者的着力点在于日常场景的叙述和情境的刻画。从开篇车窗玻璃内外风景与人的叠加,到结尾火光之上的银河,作者精心营造的意境,读来的确让人印象深刻。不过从个人角度,这类“散文化”的小说相比于情节清晰紧凑的小说,并不受到我的偏爱。小说写岛村与艺伎的朦胧恋情,从“美的徒劳”写到美的毁灭,这其中是否包含着对人世的感伤?
\par \rightline{2021年12月4日}

\subsection*{情书}
\addcontentsline{toc}{subsection}{情书}
\par \textbf{Love Letter}
\par \emph{岩井俊二 /  穆晓芳译}
\par 今年520那天和女朋友看了重映的同名电影。电影简单而动人的情节打动了我,也给那晚的约会增添了一些淡淡的忧伤。我很快找到了这本书买来看,用了和看电影差不多的时间就读完了。人物对话占了小说的大部分篇幅,使人感觉小说只是在以类似电影剧本的方式叙事;这也使我觉得电影对这个故事的表达更为成功。
\par 在情节构思方面,同名同姓的设定自有其巧妙之处。而在主题方面,这个故事可以说是爱和生死的交织。从中学时代朦胧的感情,到青年时期热烈的爱恋;图书管理员、玻璃工匠、油画学徒……故事的主人公们并非什么成功人士,但正是众多平凡人物的生活中,存在着有关“爱”的动人故事啊。长眠雪山的登山者,让我想到白银百公里越野赛意外逝去的跑者们,令人叹惋。面对高烧四十度的藤井树(女),在家等救护车还是冒着风雪护送病人,又该是何等艰难的抉择啊。在山间小屋的火锅旁,当生者追忆逝者的往事,幸存者为了更多人的生命而坚守,影片的配乐深沉却又澄澈,大概是我最喜欢的片段;一句“你好吗?我很好”,又似乎用爱情为交织的种种作了简单的解答:爱可以跨越有机界与无机界,独立于时间箭头的延伸,乃至“超越永恒”。
\par \rightline{2021年7月11日}

\section{科幻、奇幻文学}

\subsection*{平面国}
\addcontentsline{toc}{subsection}{平面国}
\par \textbf{Flatland: A Romance of Many Dimensions}
\par \emph{Edwin Abbott Abbott /  陈凤洁译} 
\par 终于读完高中时同学推荐的这部“数学幻想小说”。其中的数学内涵现在看来是基本的,高维立方体和高维球已经是大一的议题了。不过小说中提到,从三维空间能看到平面国物体的内部,由此类比,在四维空间中能看到我们日常所见的三维物体的内部,这一点我之前没有想到过。
\par 小说有一段振奋人心的献词,鼓舞人们探求更高维空间的奥秘。我想起一位专长于几何的数院同学告诉我的话:“对于三维空间中的几何学,你可以看《曲线与曲面的微分几何》;但你要是想了解更高维空间发生的事情,就要先学拓扑学作为基础。”因而我读到这一段时,竟感到心潮澎湃。
\par \rightline{2020年2月12日}

\subsection*{哈利波特与凤凰社}
\addcontentsline{toc}{subsection}{哈利波特与凤凰社}
\par \textbf{Harry Potter and the Order of Phoenix}
\par \emph{J.K.Rowling / 马爱农、马爱新译} 
\par 在2017年暑假终于读完《纳尼亚传奇》后,我开始读《哈利波特》系列。比起同龄人,20岁开始读这七本书显得有些晚。小学阅读课上,曾有一名同学借给我《哈利波特与魔法石》看,我看了十几页似乎觉得并没什么意思?而我渐渐发现罗琳构建的这个魔法世界对我们这一代人,实在有着广泛的影响。初入大学,一位同学在自己的创意画上写下“Accio GPA”的宣言;我也曾见到有人对着宿舍的门大喝一声“Alohomora”。高中时一位热心哈迷向大家介绍了Pottermore网站(现在叫Wizarding World),其中有一个“分院测试”项目。我最喜欢的学院是Ravenclaw,但分院测试的结果是Hufflepuff,想来那的确更符合自己的特质。
\par 言归正传。七部小说的故事情节,逐渐从校园生活转变为残酷的战争;《哈利波特与火焰杯》中伏地魔的复活,可谓是一个转折点。在《哈利波特与凤凰社》的神秘事务司之战中,哈利和同学们与食死徒展开真正的对抗,而D.A.是他们得以这样做的基础。我觉得D.A.的创建和活动,实在是这部书的Highlight,正如现实生活中那些能定期开展活动、成员自发参与的学生组织,常常给参与者带来真挚的友谊和难忘的回忆。
\par Fred和George Weasley,通过一个“飞向自由”的情节,把一直以来表现出的性格和志向发挥到了极致。新出场的Luna Lovegood也是我比较喜欢的一个角色,她有着鲜明的不同寻常的特质,令人印象深刻。
\par \rightline{2020年3月13日}

\subsection*{哈利波特与“混血王子”}
\addcontentsline{toc}{subsection}{哈利波特与“混血王子”}
\par \textbf{Harry Potter and the Half-Blood Prince}
\par \emph{J.K.Rowling / 马爱农、马爱新译} 
\par 田先生在他的课上说,《哈利波特》绝不仅仅是一部魔幻小说,它有着深刻的现实意义。快要读到系列尾声的时候,我想起了这句话。与伏地魔的两次战争被称为“第一次巫师战争”和“第二次巫师战争”;目前人类历史上被冠之以“世界大战”的战争,也恰好是两次。伏地魔和食死徒,邓布利多和凤凰社,这两个阵营有什么鲜明的差异呢?伏地魔利用恐怖统治他人,这一点在德拉科身上体现得很突出:他虽声称成功杀死邓布利多将在食死徒中获得无上的荣耀,却同样在天文塔上说出伏地魔以杀死全家相逼。此外,伏地魔及其党羽宣扬纯血统巫师的优越,这在人类历史上也似曾相识;而邓布利多一方不仅对麻瓜出身的巫师没有偏见,对于家养小精灵等其它的生命也能表示出尊重。
\par 这部小说的主线之一是邓布利多向哈利展示伏地魔的过去,并带他走上寻找伏地魔魂器的道路。他没有把这份重任托付给学校里其他资深的巫师,或是法术高强的傲罗,而是选择了哈利和他的朋友们。这一点是耐人寻味的。从寻找挂坠的过程看,有不少环节是哈利一个人难以过关的(邓布利多本人也说“跟我的力量相比,你的力量恐怕可以忽略不计”)。但从哈利在随邓布利多出发前对罗恩、赫敏的交代中,我看出他的从容镇静,这代表着他正走向成熟。“分一点给金妮,替我向她说声再见。”这部小说在校园恋情上着墨不少,而我觉得其动人之处则在于大敌当前的危险、斗争的重任给恋情带上的悲壮意味,这是一般的校园恋情所没有的。
\par 参加天文塔之战的学生,恰恰是一年前在神秘事务司抵抗食死徒的那六人。有的地方管他们叫“D.A.六杰”。为什么其他人没有来并肩作战呢?诚然,在严峻的形势下,明智的人不难明白要学习防御本领用于自卫。而主动加入与恶势力的抗争,则需要真正的勇气。读到小说结尾哈利与金妮的对话时,我想起这样的诗句:“弃身锋刃端,性命安可怀?父母且不顾,何言子与妻。名编壮士籍,不得中顾私。捐躯赴国难,视死忽如归。”
\par \rightline{2020年4月3日}

\subsection*{哈利波特与死亡圣器}
\addcontentsline{toc}{subsection}{哈利波特与死亡圣器}
\par \textbf{Harry Potter and the Deathly Hallows}
\par \emph{J.K.Rowling / 马爱农、马爱新译} 
\par 小说的前六部都写到“校园生活”,有上课、考试、魁地奇比赛这样的插曲;相比之下,最后一部的情节显得紧凑得多。第六部写完,至少有四个魂器需要处理;罗琳还嫌内容不够充实,又加入了“死亡圣器”。应该说“找齐几件东西”是幻想故事中拉长篇幅的常见技法,比如“虹猫蓝兔”的七剑合璧、《福娃》先找如意碎片,再找“精神力量”等等。这种方法容易把本来完整的情节分段化,一旦处理不当,就有长篇小说退化为短篇小说集之嫌。不过在我看来,罗琳的“找齐魂器”还是比较成功的:里德尔的日记给了第二部剧情一个更明白的解释;哈利等人摧毁的四个魂器,其方法各不相同,而且在本书中情节的节奏逐渐加快:寻找和摧毁挂坠盒已经用了一半篇幅,金杯、冠冕和蛇则是在霍格沃兹之战的硝烟中被接连摧毁的。
\par 斯内普的反转可谓是作者为读者设的一个圈套——作为双面间谍,你最后说他真正效忠哪一边都是合理的;特别是斯内普是一个“高超的大脑封闭术师”,邓布利多和伏地魔信任错了人都是可能的。斯内普在亲手杀死邓布利多后,又击伤乔治,读者怎么可能保留着他是正面人物的猜测呢?
\par 通过斯内普的故事,作者似乎在传递这样的认识:我们常常谈论选择、价值观乃至信仰,但行动时常是被情感驱使的。斯内普在自身的成长中并没有选择邓布利多一方的博爱,而是走向了伏地魔的“巫师至上”;对莉莉的爱却使他最终转变了阵营。马尔福夫妇大战中的“一切为了儿子”,不再关心他们的“主人”是不是胜利,同样出于此。现实生活中是否当真如此呢?我只知道,自己也会因为一个人而对一个城市、一门学科心向往之,再说不出它们的坏话。
\par “不要低估爱的力量。”在作者的笔下,哈利在成长中不断体会亲情、友情、师生情以及爱情的。我想,邓布利多和伏地魔的不同,不是简单的正义与邪恶、光明与黑暗。伏地魔不懂得爱,甚至随意屠戮自己的追随者,削弱自己的力量。如果巫师真的谋求统治麻瓜,当魔法遇上巫师瞧不起的先进科技,谁能获胜还是个未知数;但不去这样做的出发点是共情。作者似乎希望告诉孩子们,真正的爱将引导一个人走上正确的道路。
\par 在“第二次巫师战争”中,邓布利多无疑是那个幕后的统帅。事情的发展远远超出了“神机妙算”可以把握的范围,想来很多时候画像里的他也是见机行事吧。至于死去的人如何做到在画像里活着,难道就像模仿人思维的神经网络可以在人死后继续写博客?
\par \rightline{2020年5月15日}

\subsection*{三体:黑暗森林}
\addcontentsline{toc}{subsection}{三体:黑暗森林}
\par \emph{刘慈欣} 
\par 今年暑假开始读这本书之前,不止一位朋友和我谈到《三体》三部曲。我便称赞道:“我只读过第一部,它并不以细致的描写取胜,情节设计却真是巧妙!称得上‘高潮迭起’,从头到尾都很精彩。”朋友便说,快去看看后两部吧——相比于整个系列展示的宏大图景,第一部不过是个引子罢了。读来果然令人欲罢不能。
\par “黑暗森林”无疑是这部小说的核心。当三体入侵的危机袭来,巨大的技术差距之下,直接对抗的手段一一失败;面壁者罗辑在叶文洁的指引下悟出“黑暗森林”法则,才拥有了谈判的筹码。故事情节则是戏剧化的大起大落:当人类经过大低谷的洗礼,技术突飞猛进,两千艘威力强大的星际战舰让人类达到了乐观的顶峰;谁料一个小小的水滴让太阳系成了又一个威海卫,其惨烈情状无可言说。当其他面壁者紧锣密鼓地展开计划之时,罗辑居于北欧的世外桃源,上演一出“梦中情人走向现实”的恋爱戏,触到人内心的柔软。行文中亦不乏新奇的观点和设想,读来令人称妙。
\par 先说“黑暗森林”理论。一部小说的深刻性常来源于所涉及的有关社会、人生以及人类精神的重大问题。这部小说试图处理“宇宙社会学”这一课题,可谓格局宏大;不论是将人类世界的社会学推向宇宙,还是在科幻中引入社会学思想,都具有相当的启发性。至于“黑暗森林”理论本身,网络上有不少从两条公理、两个基本概念(猜疑链和技术爆炸)出发的反驳意见,但这些讨论不完全是纯粹、非功利的;毕竟要使发言引人注意、显示“水平”,大家不懂的东西要说好;大家能懂且说好的东西要说不行。在我看来,这一理论有其合理之处,却不能说可靠。
\par 尽管作者把星舰地球的生存死局和宇宙中的文明之争都作为黑暗森林理论的实例,在我看来它们有着本质的区别,区别在于竞争中的一个单元是否有能力在遭到反击之前彻底消灭对方。在星舰地球的场景中,次声武器可以彻底消灭一艘星际战舰上的全部人员,一旦成功,再不会受到威胁。这种情形类似于几个人被困在封闭环境(如荒岛、洞穴),生存资源有限的场景,可以设想类似于《饥饿游戏》的恐怖事件会发生,只不过这里的竞争单元由战舰换成了个体。彼得·萨伯讨论的法哲学公案《洞穴奇案》情境与之类似,不过几个人没有直接诉诸暴力,而是试图进行协商。而宇宙中的文明情形就不同了。我们是否可以把文明与单一星球或者太阳系这样的系统等同起来呢?答案是否定的,就连面临三体危机的地球都保留了星舰地球这样的分支;具备星际远航能力的三体文明乃至更高文明,建立多星球的太空帝国完全是可能的。如果是这样,我们还能遵循黑暗森林的原理,对探测到的文明直接“消灭之”吗?如果是袭击了一个庞大帝国的珍珠港,进而招致大规模的反击,这样的行动能称得上符合“生存是文明的第一需要”吗?一个类似的情景是后核武器时代地球上的国家之争,就算将来有一天出现资源危机,要攻击一个战略纵深较广的有核国家,恐怕也要掂量掂量可能的反击吧。所以,真正的宇宙文明图景,很可能不像刘慈欣设想得那么简单。
\par 再说“面壁计划”。破壁人从面壁者的行为中推测战略意图,犹如一场格局宏大的解谜游戏,作为一个推理解谜爱好者,读来也是十分过瘾。不过我觉得至少对于泰勒、雷迪亚兹而言,他们的计划于事无益却消耗大量资源,由此看来ETO的破壁也是帮了地球人的忙。
\par 除罗辑外,章北海算得上是另一个成功的面壁者,尽管他保全的“自然选择”号、以及他本人都毁灭于黑暗战役,地球文明的火种的确在他的精心策划下实现了逃亡。抛开他的战略不谈,读到“看不透”的人,显示着坚定信念的目光,似乎觉得有种特别的魅力。
\par \rightline{2021年12月19日}

\section{侦探、推理、悬疑}

\section{寓言、神话}

\subsection*{拉封丹寓言}
\addcontentsline{toc}{subsection}{拉封丹寓言}
\par \textbf{Fables}
\par \emph{La Fontaine / 苏迪译} 
\par 书中的寓言故事篇幅短小,大多只有半页纸。很多故事采用拟人手法,饶有趣味。不少生活中听到的故事在本书找到了出处。值得一提的是,我读的本子收录了Percy J. Billinghurst精美的版画插图,绝大多数都可以直接拿去印明信片,单凭这些插图也值回了买书的钱。
\par 简要总结一些令我印象深刻而不那么著名的篇目。《死神与伐木工》一篇写世人相较于死亡宁愿生而受苦,与周国平人生寓言中《落难的王子》有相通之处。《老鼠的议会》讽刺议会讨论积极却无人执行,令人捧腹。《狐狸和山羊》讨论了合作者的背叛,让我想起玩踩气球游戏的经历。(有人说,当自己的盟友突然踩向自己气球的时候,那是怎样的心理创伤啊。反观古罗马史中的政治同盟,还不是要决个你死我活吗。)在《狼和牧羊人》中,一只因残杀羊受到恶名的狼发现“自诩为羊群保护神的人类也在吃烤羊肉”,最后得出结论“狼的唯一错误在于它不是所有生命的主宰”,角度独特。《橡子和南瓜》十分滑稽。《两只鸽子》写远行的鸽子途中遭遇各种危险,使我思量自己周游世界的计划还是限于相对安全的地方为好。《要求有个国王的青蛙》道出了君主制的弊端,即难以保证君主的贤明。为数不多的几篇体现了作者对于一些议题的立场,如《掉进井里的占星家》批评了占星术的虚伪,还有一两篇意在说明“动物也会思考”。《寓言的威力》一篇在全书中有纲领性的意义,表明了创作寓言的动机:“有人说,世界衰老了;而我却认为,人们如同孩子,他们渴望那些浅显易懂、充满快乐的故事。”

\par \rightline{2021年2月6日}

\section{诗歌}

\subsection*{仓央嘉措圣歌集}
\addcontentsline{toc}{subsection}{仓央嘉措圣歌集}
\par \emph{龙冬译} 
\par 一本颇费些周折才淘到的书。对这本书的兴趣多半来自歌曲《在那东山顶上》,最初听的是谭晶《看见》的版本,后来找到玛吉阿米藏音的版本,觉得这两位藏族歌手演绎得更有韵味。译者的翻译以忠实原文为原则,比如第24首“若要附和美人心愿,恐将失去此生佛缘;若是隐居山里修行,又会背离女子芳心。”由此观之,所谓“世间安得双全法,不负如来不负卿”一定程度上是一种再创作。这本诗集并非简单的情歌,有几首作品明显表达了敌对势力对自己的恶毒中伤。作者甚至认为诗中很多情语也只是一种隐喻,对此我对背景了解不深,尚难以判断。
\par \rightline{2020年7月23日}

\section{散文、随笔}

\section{纪实、传记}

\section{儿童文学}

\subsection*{窗边的小豆豆}
\addcontentsline{toc}{subsection}{窗边的小豆豆}
\par \emph{黑柳彻子 / 赵玉皎译} 
\par 好久没读儿童文学作品了,果然感觉阅读压力小了很多,可以一口气读几十页(大概是最近读论文的缘故)。这本书的版权页将其归类为“儿童文学——长篇小说”,但既然写的是真人真事,归入纪实散文或许更合适。
\par 书中的内容并不限于巴学园,但小林宗作校长的教育理念和方法应当说是全书的主线之一。我相信日常的自由活动时间、亲近自然或是接触生产实践的集体出游对小学生有益处,但对小学生能否主要通过自学奠定知识的基础抱有怀疑。巴学园的小班教学实验当然很难适应竞争激烈的当下社会,但如果高中三年完全以大学为目的,初中三年完全以高中为目的,小学六年完全以初中为目的,我想那也不是最理想的情况。
\par \rightline{2020年7月23日}

\section{文学研究、文学史}

