
\section{中国古代文学}

\subsection*{三国演义}
\addcontentsline{toc}{subsection}{三国演义}
\par \emph{罗贯中} 
\par 我初中时读了《西游记》和《水浒传》;《三国演义》读了一半左右就上了高中,便搁置下来。六年多过去,这才又拿起来读;由于间隔时间太久,干脆从头读起。22岁读《三国》显得晚了些,甚至说起来有些令人惭愧,但晚读也有晚读的好处:初中时我每天强撑着完成读三回的任务,现在倒觉得其中的权谋计策有些意思,不必规定任务就津津有味的读下去了。初中时只觉得陈琳讨曹操的檄文“畅快淋漓”,如今其中影响最大的精彩片段当然是“诸葛亮骂死王司徒”了,哈哈。
\par 读此书似看戏。在历史舞台上,主公、忠臣、反贼、内奸,你方唱罢我登场。对于没有实权、说被废就被废的君主,处于弱势或战败的诸侯、武将,常常面临“生”与“义”的抉择。我常叹献帝不能掌控命运,被权臣肆意欺凌,还不如效仿高贵乡公曹髦以卵击石,有尊严地死去。至于臣子,有的宁死不降,有的“择良木而栖”,倒都可理解,毕竟内部矛盾不涉及民族大义。最惨的是“生”“义”二者皆失,如荆州刘琮,拱手投降,被诛于赴任路上。而在残酷的政治斗争中,一些小人物难以左右大局,却一样做了牺牲品。比如那些送信的使节就令人同情:有时还不知道信里写的是啥,就被对方“怒斩来使”。
\par “均势”是我的另一点体会。魏国最强,因而孙刘联盟,魏伐蜀则吴伐魏,魏伐吴则蜀伐魏,这才得以成鼎足之势。这种“均势”在现当代国际政治中也屡见不鲜。在一国之内,君臣之间也存在着一种“均势”:曹操、刘备、孙权能统领手下文臣武将,而他们的后继者有的没有那样的雄才,以致手下功高盖主,大权在握。掌握重权的臣子,若没有诸葛亮那样的尽忠之心,行废立、谋篡位便是常见的事了。
\par 小说作为封建时代的产物,竭力渲染“天命”“神鬼”:大将身死之前,必有“将星坠于野”;孔明、关羽死后显灵;甚至出现了管辂教人求神仙改寿数的情节。要是真有一个数据库保存着每个人的寿数,那岂不是会有大量程序员毕生致力于入侵那个数据库?
\par \rightline{2020年7月6日}

\subsection*{红楼梦}
\addcontentsline{toc}{subsection}{红楼梦}
\par \emph{曹雪芹 / 无名氏续\  程伟元、高鹗整理} 
\par 《红楼梦》被誉为中国古典小说的高峰,因其复杂性、残缺性引发了后人极为丰富的讨论和探索。红学家如俞平伯、周汝昌、蔡义江、梁归智,文人学者如张爱玲、王蒙、叶朗、王博,通俗讲者如刘心武、蒋勋、欧丽娟,还有众多关心和参与其讨论的读者和红学爱好者……一部小说受到如此广泛和深入的关注,我想在世界文学史上都是绝无仅有的。而进入这些讨论的钥匙,正是《红楼梦》原著。
\par 《红楼梦》的人物塑造、语言功力之高,也是我读原著之前就知道的。在读原著之前,我就知道“金陵十二钗”,特别是林黛玉的形象已成为日常用典;高中语文老师便让我们背“白玉为床金做马”;初中时候一位女同学谈到作文文笔优美的秘诀,“你们去读《红楼梦》吧,那实在是好词好句的宝库。”我读到宝玉的酒令“滴不尽的相思血泪抛红豆”,想起这正是她作文里曾整段引用的话。
\par 去年夏天去北京植物园,走到曹雪芹纪念馆,我道:“不会有人还没看过《红楼梦》吧?不会吧不会吧?哦,原来这人是我自己。” 终于下决心读四大名著中剩下的最后一部,是在今年春天;到11月20日读完,期间虽也看些别的书,但周末大段的读书时光,都在这里面了。
\par 读书之初最深的感受是叙述之细致:第七回写周瑞家的给姑娘们送宫花,作者花不少笔墨依次写来,区区小事花了近半回笔墨。有人说作者正是从细处着手刻画人物性格;而在我看来,作者能把荣宁府、大观园里每个人的一言一行写得如此细致,这一切在读者看来也变得真实可感了。读着读着,我似乎觉得自己在见证少年宝玉的生活;提到黛玉、湘云、宝琴,竟仿佛现实中朋友的名字一般。我印象很深是一个情节是第二十三回林黛玉听《牡丹亭》,闻“如花美眷,似水流年”而醉心落泪;时隔数百年,这里面也有我们青春的影子。
\par 我曾和父亲交流读《红楼梦》的感受,殊不知我读来最厌烦的对服饰器物的铺叙正是他最感兴趣的;而他一概跳过的诗词曲,却是我最为钟爱、反复咀嚼的。我到书店里翻欧丽娟的书,她提到对雅文化的赞美是作品的主题之一,对此我是赞同的。书中多次写到宝玉和姐妹们结社,赋诗联句,我读来作者本意并不是所谓“封建统治阶级文学的空虚”,的确是展现了作诗的技艺、巧思和乐趣;宝玉题大观园匾额对联、香菱向黛玉学诗、宝玉作{\CJKfontspec{simsun.ttc}姽婳}词几章也都十分精彩。穿插在公子闺秀们的风雅游戏之间的,却是现实世界的种种不堪:从家塾学生的打斗,到家族中的勾心斗角、聚赌通奸……再看作诗联句,芦雪广即景联诗何其热闹,到了凹晶馆则唯有黛玉湘云二人,这不正是贾家由盛转衰的象征吗?
\par 要说诗歌艺术,我最喜欢的还是第五回宝玉梦游太虚幻境,金陵十二钗的判词和红楼梦十二支曲,其中暗暗伏下整个故事的走向和每个主要角色的结局。伏笔照应也是小说艺术中很吸引我的一种,因而红学诸多流派,我对探佚学最有兴趣。虽说续书作者总引用前八十回的情节来表明自己承接前八十回,但看了第五回的“剧透”,就知道续书的故事走向完全错了。续书中虽写到元妃薨逝、王子腾病亡,贾家被抄,但最后结局居然是“沐皇恩贾家延世泽”,贾赦、贾珍被赦,贾政官复原职,死刑犯薛蟠获释,香菱还成了正室;虽说宝玉离家出走,宝钗怀孕留下子嗣,贾兰也得中举人。好一个大团圆结局!可是这恰恰是最不应该大团圆的小说。读到前八十回末尾,我已感到明显的衰败走向;按照由盛而衰的总基调,有理由相信故事会一路下行,走向一个个悲惨的结局:富贵温柔,只是尘世一梦。我甚至觉得按照前八十回下行的惯性,可能全书一百回就结束了,达不到刘心武的108回,张之的110回,更没有程高本的120回。传说曹雪芹写作《红楼梦》泪尽而逝,想来竟是有可能的——读者看来红楼闺秀们尚且真切鲜活,作者更会觉得她们如同活在自己的生活中一样;那么每想一次她们的结局,便为之心痛一次。作家写作一个故事,其中情节不知要在脑中“排演”多少次。由此想来,如何不令人悲恸!
\par 宝钗是封建礼教的代言者,加之对金钏儿之死的冷漠(高中时读到周先慎《琐碎中有无限烟波》的分析),我并无好感。黛玉伶牙俐齿,才华值得欣赏。人民文学出版社出了一套黛玉怼人语录书签,我买了其中两款:“咱们只管乐咱们的”、“我就不能一目十行吗”。我觉得黛玉教香菱学诗说的话更有意思:“什么难事,也值得去学!我虽不通,大略还教得起你。”相比之下我更喜欢湘云、宝琴,除才华外,还有些活泼率真的性格。
\par \rightline{2022年11月26日}

\section{中国现当代小说}


\subsection*{围城}
\addcontentsline{toc}{subsection}{围城}
\par \emph{钱钟书} 
\par 《围城》称得上是一部奇书。书中故事是知识阶层青年最为平常的工作、恋爱、家庭、社交经历,也并无什么新奇的想象和夸张,却写得极为细致和真实(特别是人物的心理活动)。此书的另一特点是其连串的俏皮话,不时令人捧腹。比如李梅亭在三闾大学讲授“先秦小说史”,此课可谓内容充实。
\par 初读此书,认为不过是一部幽默小说,作者借以讽刺种种想讽刺的人和事,包括新诗人、旧体诗人、炫才者、贪图小利者、爱慕虚荣者…… 不想越读越觉得沉重,到最后一章达到顶峰:三闾大学中的勾心斗角尚可平心而观,一对知识阶层的青年夫妇,相遇尚存些许美好,终于在两个大家庭的夹缝中失和。这不由得引发我对人性的思考。
\par 大学的一位先生曾在课上引述书中的话:“你是个好人,但没用。”(赵辛楣评价方鸿渐的原话是“你不讨厌,可是全无用处。”)以此告诫我们。当时我不以为然,正所谓“无用为大用”。后来渐渐觉得,这“用处”或许是指人在社会网络中的所谓“势力”,即财力、社会关系以及个人才能的总和。没有谁要求一个人“有用”,但“有用”可以少些依附、嘲讽,生活得舒服些;或者需要自己和亲人少些攀比之心、虚荣之心,“安贫乐道”。
\par 我以为方鸿渐最严重的弱点在于没主意。游学漫无目的,职业只等亲友相助,追求唐晓芙或许是唯一的例外。
\par \rightline{2021年8月18日}

\subsection*{南行记}
\addcontentsline{toc}{subsection}{南行记}
\par \emph{艾芜} 
\par 2021年底读了这本不足十万字的短篇小说集,算是最后冲了一下KPI。八篇小说以上世纪二十年代滇缅一带的漂泊生活为背景,都用第一人称而少有外部视角的评述,读来如记述亲身经历的非虚构作品。得知此书缘于一本短篇小说选集中的《山峡中》,现在看仍是集子中艺术成就最高的一篇,奇险的自然环境和特殊的一群人构成的场域,令人印象深刻。《洋官与鸡》描绘洋官搜刮民脂民膏的丑恶嘴脸,也十分生动。主人公虽是漂泊于社会底层的流浪者,却表现出可贵的正义感(《我诅咒你那么一笑》)、反思性与革命性(《松岭上》)、坚韧精神(《人生哲学的一课》),传递着昂扬向上的精神力量。
\par \rightline{2022年1月1日}

\subsection*{俗世奇人(二、三)}
\addcontentsline{toc}{subsection}{俗世奇人(二、三)}
\par \emph{冯骥才} 
\par 中学时读冯骥才的小说集《俗世奇人》,缘于课本所选《刷子李》等篇,书中也的确有不少精彩篇目。近年来冯骥才连推续作,却未能摆脱“一部不如一部”的魔咒,让人读完印象深刻的篇目减少了。个人觉得写市井奇人绝技比较出色的有第二册的《四十八样》,以及第三册的《白四爷说小说》。此外第二册《张果老》讲了一个精心设计的骗局,揭示了收藏者追求收藏品完整(集齐一套)的弱点,很是有趣。第三册《十三不靠》中的汪无奇,则称得上是一个让人拍案叫绝的人物。
\par \rightline{2020年12月8日}

\subsection*{俗世奇人(四)}
\addcontentsline{toc}{subsection}{俗世奇人(四)}
\par \emph{冯骥才} 

\par 逛网上书店偶然发现冯骥才的“俗世奇人”系列又续了一本。还好一直买的是作家出版社的分集版;另一种所谓的“全本”“足本”又变得不全和不足了。故事依旧短小可读,但与前三本相比还是一本不如一本;虽说也算得上奇人奇事,却不足以令人称绝、读后给人留下深刻印象了。
\par \rightline{2023年5月15日}


\section{外国小说:中世纪及以前}

\section{外国小说:文艺复兴至十八世纪}

\section{外国文学:十九世纪}

\subsection*{欧也妮·葛朗台}
\addcontentsline{toc}{subsection}{欧也妮·葛朗台}
\par \textbf{Eugénie Grandet}
\par \emph{Honoré de Balzac / 李恒基译} 
\par 小说的故事情节并无特别的新意——对金钱和地位的狂热追逐,穿插在“痴情女子负心汉”的感情线中。类似题材的作品中,男女主角地位交换的情况在文学作品中也有出现,如皮兰德娄的《西西里柠檬》,可与之对读。
\par 值得一提的是,作者常常中断情节的叙述,置身事外对故事中人物做一番评论和分析,而这些评论多有引人共鸣之处。作者不仅构造故事,还探讨社会的现象、人的精神的现象,并试图分析其原因,从中可以看出其风俗(社会现状)——哲理(原因)——分析(原则)三步框架的影子。我想这也是本书得以成为名著的重要原因。
\par 小说中的葛朗台老爹被刻画为一个受金钱异化到无以复加地步的典型形象。人类文明发展进步了近两百年,面对金钱的诱惑,我们的表现却似乎并没有多少长进。他们说看透了这世界,其实常常只是看到了表象之上的谎言:因为金钱是手段而非目的,并且很多东西如果一定要标个价,那会是阿列夫零。
\par \rightline{2021年4月24日}

\section{外国文学:二十世纪至今}

\subsection*{二十世纪外国短篇小说精选}
\addcontentsline{toc}{subsection}{二十世纪外国短篇小说精选}
\par \emph{王向远选编}

\par 这是人民文学版语文必读书目中的一本,一直以来被我当作火车上打发时间的读物,断断续续看了数年之久。
\par 本书所选篇目中,我最欣赏的是马赛尔·埃梅的《穿墙记》,其构思之精彩、叙述之生动堪称典范,作为穿墙者的小人物煊赫一时,被困墙中的形象则留下悲凉的余味。其次是泰戈尔的《喀布尔人》,其语言亲切温暖,是一曲跨越种族的人性之美的高歌。
\par 一篇小说如果没有奇妙的构思、细致的刻画,其主题再深刻,也只是一个空架子。卡夫卡的《判决》对心理的刻画细致入微,卡尔维诺的《阿根廷蚂蚁》中对各种灭蚁装置的设想更是让我惊叹,体现了虚构场景下讲出逼真故事的高超能力。我想即使不知道作者,读这两篇文本也能看出大家的功力。
\par 亨利·巴比塞的《十字勋章》和萧伯纳的《皇帝与小姑娘》同为反战题材,前者基调沉重,后者则以幽默、讽刺的笔触叙事,都具有强烈的感染力。博尔赫斯的《小径分岔的花园》文字带些意识流色彩,将时间分岔(平行宇宙)的设想嵌入一个构思奇巧的谍战故事中。此外库尔特·冯内古特的《今天我演什么角色》,杜鲁门·卡波特的《圣诞节忆旧》,川端康成的《伊豆的舞女》,志贺直哉的《清兵卫与葫芦》,都是值得一读的佳作。
\par \rightline{2022年8月25日}

\subsection*{老人与海}
\addcontentsline{toc}{subsection}{老人与海}
\par \textbf{The Old Man and the Sea}
\par \emph{Ernest Hemingway / 孙致礼译} 
\par 这篇小说情节算不上扣人心弦,但能把老人一段出海打鱼的经历写得这么长,实属不易。读到最后人们称道鱼骨架之大时,我才感到一丝悲壮:用“硬汉”的气概与命运抗争的人,纵使功业未成,依然令人肃然起敬。这一点让我想起《布兰诗歌》里的“O Fortuna”。
\par \rightline{2020年2月16日}

\subsection*{诺贝尔的囚徒}
\addcontentsline{toc}{subsection}{诺贝尔的囚徒}
\par \textbf{Cantor’s Dilemma}
\par \emph{Carl Djerassi / 黄群译} 
\par 这是王骏教授在《自然辩证法》课上提到的一本书。作者本人是一位知名的化学教授,退休后致力于用文学作品帮助公众了解“科学江湖”的文化。的确,科研团队的生态较少成为文学作品表达的对象,因此我读此书颇有兴致。事实上,此书涉及的主要是生物学或化学领域的实验室生态,它与我所在领域的情形有一定差异。小说引入了一些与主题关联较弱的人物情感元素,使我读来有些网文的感觉。情节方面,康托和杰里的研究在缺乏重复实验验证的情况下短时间内获得诺贝尔奖,应该说是一个较为牵强的地方。
\par 作者在后记中说:“发表论文、优先权、作者的排序、杂志的选择、大学的终身教职、资助的申请、诺贝尔奖……这些是当代科学的灵魂和包袱。”小说对科学文化中类似议题做了十分细致的刻画,如杰里在领取诺贝尔奖时如何表明自己的贡献。
\par 科学发现的可重复性,是科学界的重要议题之一。前不久国内学界发生了一场围绕G蛋白偶联受体相关研究可重复性的论争,可谓是让人大开眼界。本书中实验的可重复性构成了康托最大的忧虑,也成为克劳斯要挟康托的手段之一。相比之下,我所在的偏数据科学的领域,重复实验就是跑一跑作者公开的代码那样简单,这看来是一件好事。
\par 以R.K.默顿为代表的科学社会学理论是《自然辩证法》课的主要内容之一,其中一个核心观点是“科学家的工作是为了获得同行的承认”,包括奖章、奖金、Fellow头衔在内的各种科学奖励都是同行承认的表现形式。这部小说也在试图宣传这一观点。而在我看来,如果把论文篇数、引用数以及各类科学奖励视为从事科学研究的全部动力,那无疑是另一种异化。
\par \rightline{2021年2月14日}

\subsection*{雪国}
\addcontentsline{toc}{subsection}{雪国}
\par \emph{川端康成/ 叶渭渠译}
\par 这是一部“弱情节”的小说,作者的着力点在于日常场景的叙述和情境的刻画。从开篇车窗玻璃内外风景与人的叠加,到结尾火光之上的银河,作者精心营造的意境,读来的确让人印象深刻。不过从个人角度,这类“散文化”的小说相比于情节清晰紧凑的小说,并不受到我的偏爱。小说写岛村与艺伎的朦胧恋情,从“美的徒劳”写到美的毁灭,这其中是否包含着对人世的感伤?
\par \rightline{2021年12月4日}

\subsection*{情书}
\addcontentsline{toc}{subsection}{情书}
\par \textbf{Love Letter}
\par \emph{岩井俊二 /  穆晓芳译}
\par 今年520那天和女朋友看了重映的同名电影。电影简单而动人的情节打动了我,也给那晚的约会增添了一些淡淡的忧伤。我很快找到了这本书买来看,用了和看电影差不多的时间就读完了。人物对话占了小说的大部分篇幅,使人感觉小说只是在以类似电影剧本的方式叙事;这也使我觉得电影对这个故事的表达更为成功。
\par 在情节构思方面,同名同姓的设定自有其巧妙之处。而在主题方面,这个故事可以说是爱和生死的交织。从中学时代朦胧的感情,到青年时期热烈的爱恋;图书管理员、玻璃工匠、油画学徒……故事的主人公们并非什么成功人士,但正是众多平凡人物的生活中,存在着有关“爱”的动人故事啊。长眠雪山的登山者,让我想到白银百公里越野赛意外逝去的跑者们,令人叹惋。面对高烧四十度的藤井树(女),在家等救护车还是冒着风雪护送病人,又该是何等艰难的抉择啊。在山间小屋的火锅旁,当生者追忆逝者的往事,幸存者为了更多人的生命而坚守,影片的配乐深沉却又澄澈,大概是我最喜欢的片段;一句“你好吗?我很好”,又似乎用爱情为交织的种种作了简单的解答:爱可以跨越有机界与无机界,独立于时间箭头的延伸,乃至“超越永恒”。
\par \rightline{2021年7月11日}

\subsection*{铃芽之旅}
\addcontentsline{toc}{subsection}{铃芽之旅}
\par \textbf{Suzume}
\par \emph{新海诚 / 吴春燕译}

\par 和《你的名字》一样,观影之后找来小说一读。新海诚的小说基本完全重复了电影的台本,增加的信息主要来自人物的心理描写。小说以铃芽为第一人称叙述,但也有草太的心理描写,似有些不合理。
\par 就故事情节而言,草太变成三条腿椅子的设定比较有趣,给影片前半部分带来了不少欢乐。再就是蚓厄的形象颇具震撼力,特别是在东京上空开出花朵的壮观场景。(我感觉蚓厄的设计似乎与日本的妖怪传统有些关联,尽管我对日本妖怪的几乎全部印象都来自桌面游戏《旭日战魂录》。)顺便一提,小说译者译作“蚯蚓”远不及“蚓厄”;闭门咒的翻译在我看来也不及电影字幕的译本整齐、有力。最费解之处是大臣的所作所为,看完电影我一直以为大臣打开了沿途的后门,并拔出了左大臣;目的是让草太变成要石,自己获得自由和(愿望中的)铃芽的喜爱和陪伴。然而小说明确交代,大臣没有打开后门,只是为了把铃芽引到儿时开启的后门。那么左大臣是怎么出来的?大臣神通广大到能预知哪里后门会打开,却直到东京都没看出来铃芽讨厌自己?
\par 看完电影就觉得这一作不同于爱情主题的《你的名字》,故事的核心应该是创伤后的疗愈;一见钟情的恋爱虽然是情节的主要推动力,却是这种疗愈得以实现的因素之一(“铃芽,你今后会有自己最喜欢的人”)。原著小说和作者的后记使我更确信这一点:铃芽见到的常世从废墟变为绿野正是这种疗愈的象征;她听到的种种“我出门了”,恰恰包含对逝者的怀念——说这些话的人,或许永远没能再回来。对幼年铃芽来说,失去妈妈几乎就是失去一切;而常世中错乱的时空之下,少年铃芽告诉她未来的美好,给她前行的力量,读来令人感动(“我的人生导师竟是我自己”)。回望常世,铃芽、草太站在星河之下、原野之间,这图景与《你的名字》中泷和三叶的相遇又有几分相似。
\par \rightline{2023年4月22日}


\section{科幻文学}
\subsection*{三体:黑暗森林}
\addcontentsline{toc}{subsection}{三体:黑暗森林}
\par \emph{刘慈欣} 
\par 今年暑假开始读这本书之前,不止一位朋友和我谈到《三体》三部曲。我便称赞道:“我只读过第一部,它并不以细致的描写取胜,情节设计却真是巧妙!称得上‘高潮迭起’,从头到尾都很精彩。”朋友便说,快去看看后两部吧——相比于整个系列展示的宏大图景,第一部不过是个引子罢了。读来果然令人欲罢不能。
\par “黑暗森林”无疑是这部小说的核心。当三体入侵的危机袭来,巨大的技术差距之下,直接对抗的手段一一失败;面壁者罗辑在叶文洁的指引下悟出“黑暗森林”法则,才拥有了谈判的筹码。故事情节则是戏剧化的大起大落:当人类经过大低谷的洗礼,技术突飞猛进,两千艘威力强大的星际战舰让人类达到了乐观的顶峰;谁料一个小小的水滴让太阳系成了又一个威海卫,其惨烈情状无可言说。当其他面壁者紧锣密鼓地展开计划之时,罗辑居于北欧的世外桃源,上演一出“梦中情人走向现实”的恋爱戏,触到人内心的柔软。行文中亦不乏新奇的观点和设想,读来令人称妙。
\par 先说“黑暗森林”理论。一部小说的深刻性常来源于所涉及的有关社会、人生以及人类精神的重大问题。这部小说试图处理“宇宙社会学”这一课题,可谓格局宏大;不论是将人类世界的社会学推向宇宙,还是在科幻中引入社会学思想,都具有相当的启发性。至于“黑暗森林”理论本身,网络上有不少从两条公理、两个基本概念(猜疑链和技术爆炸)出发的反驳意见,但这些讨论不完全是纯粹、非功利的;毕竟要使发言引人注意、显示“水平”,大家不懂的东西要说好;大家能懂且说好的东西要说不行。在我看来,这一理论有其合理之处,却不能说可靠。
\par 尽管作者把星舰地球的生存死局和宇宙中的文明之争都作为黑暗森林理论的实例,在我看来它们有着本质的区别,区别在于竞争中的一个单元是否有能力在遭到反击之前彻底消灭对方。在星舰地球的场景中,次声武器可以彻底消灭一艘星际战舰上的全部人员,一旦成功,再不会受到威胁。这种情形类似于几个人被困在封闭环境(如荒岛、洞穴),生存资源有限的场景,可以设想类似于《饥饿游戏》的恐怖事件会发生,只不过这里的竞争单元由战舰换成了个体。彼得·萨伯讨论的法哲学公案《洞穴奇案》情境与之类似,不过几个人没有直接诉诸暴力,而是试图进行协商。而宇宙中的文明情形就不同了。我们是否可以把文明与单一星球或者太阳系这样的系统等同起来呢?答案是否定的,就连面临三体危机的地球都保留了星舰地球这样的分支;具备星际远航能力的三体文明乃至更高文明,建立多星球的太空帝国完全是可能的。如果是这样,我们还能遵循黑暗森林的原理,对探测到的文明直接“消灭之”吗?如果是袭击了一个庞大帝国的珍珠港,进而招致大规模的反击,这样的行动能称得上符合“生存是文明的第一需要”吗?一个类似的情景是后核武器时代地球上的国家之争,就算将来有一天出现资源危机,要攻击一个战略纵深较广的有核国家,恐怕也要掂量掂量可能的反击吧。所以,真正的宇宙文明图景,很可能不像刘慈欣设想得那么简单。
\par 再说“面壁计划”。破壁人从面壁者的行为中推测战略意图,犹如一场格局宏大的解谜游戏,作为一个推理解谜爱好者,读来也是十分过瘾。不过我觉得至少对于泰勒、雷迪亚兹而言,他们的计划于事无益却消耗大量资源,由此看来ETO的破壁也是帮了地球人的忙。
\par 除罗辑外,章北海算得上是另一个成功的面壁者,尽管他保全的“自然选择”号、以及他本人都毁灭于黑暗战役,地球文明的火种的确在他的精心策划下实现了逃亡。抛开他的战略不谈,读到“看不透”的人,显示着坚定信念的目光,似乎觉得有种特别的魅力。
\par \rightline{2021年12月19日}

\subsection*{三体:死神永生}
\addcontentsline{toc}{subsection}{三体:死神永生}
\par \emph{刘慈欣} 
\par 从我翻开这《地球往事》终章的那一天起,晚上回到宿舍继续阅读就成了我每天的期待。短短十二天时间我就读完了全书。在我看来,《黑暗森林》的艺术成就和《三体》相当,我甚至更欣赏《三体》中现实与游戏交织的结构;而《死神永生》以更为宏大的时空跨度、更为惊人的奇妙想象超越了前两部,可以说是我读过的最杰出的科幻小说(当然这个评价没什么分量,因为读过的还不多)。
\par 这部小说分为六部,是一部地球文明的史诗。前半部分承接《黑暗森林》,交代了黑暗森林威慑和“蓝色空间”号的结局;后半部分则是人类防备黑暗森林打击失败,太阳系毁灭的悲剧。不同于一些叙事视角仅限于若干主角或一个群体的小说,作者能出色地驾驭整个地球文明的宏大视角,能对国际社会面对各种事态的反应做出合理可信的设想;对主角心理、行为的刻画,又不失细致入微。
\par 《三体》三部曲各有一个核心的“科学概念”,作者分别在小说中展现了自己对它们的思考和想象。第一部是“三体问题”,第二部是“宇宙社会学”,第三部则是“维度”。在这部小说中,作者继承并发扬了Edwin Abbott Abbott《平面国》的精神:开篇女魔法师摘取大脑的情节,让人联想到对球的无所不知大为惊异的正方形。 对“蓝色空间”号乘员进入四维空间的描写尽管谈不上令人拍案叫绝,却也符合二、三维空间之间关系的类推。小说进一步设想,高维空间可以在低维空间中以碎块的形式存在;低维生物甚至可以进入高维空间并保持生存。这一点初看让人难以接受,三维空间怎么能和四维空间衔接呢?三维生物由三维基本粒子构成,如何能在四维空间的粒子中安然存在呢?细细想来,又无法断然否定:考虑到维度可以卷曲在微观中,谁知道在穿越维度分界的时候究竟会发生什么呢?小说还设想生物可以主动降维,从而可以对原本同一维度的文明进行维度攻击。这又是有数学基础的:平面上[0,1]和[0,1]×[0,1]基数是一样的,也就是降维可以不损失信息。
\par 地球系统的研究中有所谓“盖娅假说”(Gaia Hypothesis),即生命对地球环境有巨大影响,且朝着对自己有利的方向进行。这部小说通过杨冬生前的思考、关一帆与程心关于星际战争的对话,提出了“生命有意识地改变宇宙规律”这一惊人假说,或可称为“宙斯假说”(Zeus Hypothesis),从而为宇宙的不和谐(光速有限、多个维度卷曲在微观,关一帆称为“三与三十万的综合征”)给出了大胆的解释。这当然是未经证实的猜测(我本人倾向其不成立),但作者没有把光速有限且恒定等现代物理的基本图景视为理所当然,让自己的想象力在宇宙规律上驰骋,这份气魄令人赞叹。至于说数学规律被改造,那不会。正如与“魔戒”的交流中,素数序列可以作为智慧生命的证明,正是由于数学规律天然的时空恒定性。
\par 在故事中嵌入诗歌等其它文体,可为小说增添艺术性,这一点做到极致的是《红楼梦》。这部小说在这方面同样值得赞赏,与四维空间中“魔戒”的对话,云天明的三个童话,歌者吟唱的古老歌谣,语言含蓄,令人回味。隐喻(metaphor)一直是文学艺术中最令我着迷的手法,这些穿插的片段都展现了隐喻的迷人魅力。
\par 继第二部的章北海、罗辑之后,又有褚岩、云天明等令人敬佩的杰出人物,为地球文明延续做出了关键性的贡献。而作者笔下的大众,则自私、软弱、短视、愚昧,与之形成鲜明对照。当引力波威慑被摧毁,众多人争当“球奸”,加入效忠于三体的“治安军”;启动引力波广播的“蓝色空间”号最初被视为英雄,三体入侵的阴云散去之后,又成了带来毁灭厄运的魔鬼;面对黑暗森林打击的威胁,大批民众诉诸宗教,当智子透露可能向宇宙发出安全声明之后, 甚至三体文明都成为了朝拜的对象,读来令人唏嘘。教会和神学家花了一个多世纪重新解释经典,使三体人的存在能够符合教义,这是何其辛辣的讽刺!
\par 歌者向太阳系抛出一片小小的二向箔,整个太阳系随之毁灭,成为一幅悲壮的巨画。这是地球文明的终曲,一幕带有浪漫色彩的悲剧;而这浪漫是黑色的,正如茫茫空宇。按照作者的设想,这巨画也难以在宇宙中保留;整个太阳系将成为令当今天文学界困惑不解的暗物质。最后一代人苦心经营掩体工程,可二向箔的维度打击又有谁能预知?
\par 在太阳系灭亡的大背景下,作者重新追问生命的意义。默斯肯岛的老杰森说:“一切都将逝去,只有死神永生。”天文学家威纳尔这样评论被毁灭的三体世界:“没有真正毁掉什么,更没有灭掉什么,物质总量一点不少都还在,……,只是组合方式变了变,像一副扑克牌,仅仅重洗而已……可生命是一手同花顺,一洗什么都没了。”这个比喻很好地解释了生命的精巧和脆弱:其存在本身就是宇宙中的奇迹。更进一步地,歌者发挥了Erwin Schrödinger的思想:在熵增的宇宙中唯有低熵体(生命体)在变得更有序,“这是最高层的意义”;要维持这种意义,生命体就必须存在和延续,这是比出于好奇心探索宇宙更高层的意义。注意到这与《朝闻道》传递的观念截然相反。在我看来,正如生命的存在降低宇宙的无序度,通过探索未知发现世界的秩序,可以降低我们认识中世界的不确定度,这便是学人求索的直接意义;此外研究对象的性质和研究过程本身可以为我们提供审美价值。但我不赞同将这种探索上升到高于存在的意义层面,特别是针对整个文明而言(如《朝闻道》中的星云生物)。
\par 除探索未知之外,还有什么可能置于比存在更高的意义层面呢?孟子说“舍生取义”,这关乎伦理道德,同样是这部小说引人思考的主题。星舰地球的生存困境带来了类似于《洞穴奇案》的道德困境;不管是危机纪元还是后来的掩体纪元,平等的价值观一定程度上成为了人类飞向外太空、广播文明火种的包袱;而黑暗森林打击、维度攻击更是以“生存”的名义完全抛弃了道德。有趣的是,小说结尾重提“责任的阶梯”,宇宙的未来让关一帆、程心做出牺牲重返太空。
\par 在《黑暗森林》的闲谈中,我提出了对黑暗森林理论的质疑。这部小说使我的思考更加清晰。像故事中的三体文明和银河系人类一样,很可能多数文明都不会局限在一个恒星系内,在掌握光速飞船技术的条件下,即使遭到打击也能部分逃脱。这意味着“清理”的作用极为有限,很可能只会暂时削弱一个文明;由于最为重要的科技思想得到保留,文明有能力在很短时间内再度产生威胁。“清理”成立还有一个前提,即来源无法追踪,这消除了反击的可能性。但在我看来,这种追踪能力并非不可能实现;一旦被掌握,“清理”伴随着巨大的潜在风险,宇宙文明将是完全不同的图景。
\par 云天明对程心的单恋故事凄婉感人;他在三体世界得以生存,又精心设计为人类传递情报,堪称可歌可泣。程心自愧于两次为人类做出了错误的选择,我以为她竞选执剑人且未能启动广播是个错误,但制止星环城战争不应苛责。两人赴约于“我们的星星”;艾AA和关一帆亦有情愫,如果一切顺利,想来是美好的结局。只可惜意外使他们的命运发生了交错,两对恋人“交叉互换”。程心“终于看清,使自己这里沙尘飘飞的,是怎样的天风;把自己这片小叶送向远方的,是怎样的大河”。在苍茫的宇宙面前,生存尚且是一种奢望,何况爱情。好在关一帆和程心小世界中的学者生活依然恬静温馨。
\par 程心著述《时间之外的往事》之时依然年轻美丽,可她已见证了地球文明从危机纪元到掩体失败四百年的波澜历史。读完三部曲的读者同样是如此,百感交集之时,程心的一段话令人震撼:
\par “每个文明的历程都是这样:从一个狭小的摇篮世界中觉醒,蹒跚地走出去,飞起来,越飞越快,越飞越远,最后与宇宙的命运融为一体。对于智慧文明来说,它们最后总变得和自己的思想一样大。”
\par 我想,这段话应当能在许许多多和我一样,以及更年轻的读者心中种下一颗种子,我们会一同仰望星空,了解天体物理和宇宙学,随着人类深空探索的前哨(如JWST)看宇宙,思索宇宙中文明的关系、宇宙的前世和未来。
\par “心事浩茫连广宇,于无声处听惊雷。”是以为记。
\par \rightline{2022年11月5日}

\subsection*{寂寞的伏兵:当代中国科幻短篇精选}
\addcontentsline{toc}{subsection}{寂寞的伏兵:当代中国科幻短篇精选}
\par \emph{夏笳编} 

\par 这本选集是在几段高铁旅行中看完的。编者夏笳既是科幻作家,又是科幻研究者,选文水准高且有代表性。此前我对于中国科幻的了解仅限于刘慈欣、郝景芳,此书让我接触到了更为多样的写作风格和文学主张。
\par 书中我最欣赏的有两篇,除之前读过的《北京折叠》外,还有柳文扬的《一日囚》。“一日无期徒刑”的服刑者首先不以为然,接着渐渐因一成不变的外界环境而疯狂,最后竟开始思考这种刑罚的实现机制。“时间”既是日常生活中的基本概念,又是物理学的终极概念。小说的想法本身已足够新奇,更为可贵的是作者借主人公的思辨,很好地展现了“时间”的复杂性。
\par 赵海虹的《桦树的眼睛》使用了万物有灵的浪漫理念,但相比于奇幻故事增加了科学化的解释。正如编者指出的那样,桦树的“苏醒”不仅是为“树木之友”许瑟瑟复仇,而是具有更为深远的象征意义,代表着自然对具支配地位的现代科技(特别是屈从于利益、服务于恶行的现代科技)的一种抗争。
\par 刘慈欣的《流浪地球》早已因电影改编广为人知,但其实电影只是借用了流浪地球的概念和设定,其人物、情节可以说完全没有关系。宏大场景下的叙事,国际社会的意见分歧,坚持真理的殉道者……这些元素都与《三体》十分相似。
\par 何夕的《伤心者》尽管对悲情形象的塑造有些刻意,读来仍十分感人。主人公何夕(没错,故事中的数学家用作者的笔名命名,而完成大统一理论的物理学家用作者的真名何宏伟命名)本不必急切地为了完成和出版“微连续理论”不顾一切,正如陶哲轩指出张益唐那样“暮年诗赋动江关”的成功之路难以复制。他完全可以先通过其它成果获得稳定的收入和学界的声誉,那之后说不定《微连续原本》就不用自费出版了,也能传播更广。(或者只要等个十来年,就有arXiv了,这绝对比小学图书馆更容易被人看到。arXiv功德无量。)另一方面,我不认为数学理论的价值只有通过应用于自然科学或工程实践才能体现,无应用的数学理论也能满足人们对抽象世界的求知欲;因其优美与和谐为人们提供审美价值。为何夕这样的数学家提供资源,才反映出社会的文明进步。母亲夏群芳对何夕儿子无条件的认可和支持是小说非常感人的一点,而何夕与江雪恋情的结束则令人唏嘘。我无法苛责江雪,因她并非为了利益背弃感情;是何夕把他的理论看得比感情更重要。有趣的是这小说读到最后一节才能看出科幻体现在哪儿。
\par 另有三篇值得推荐的作品:王晋康的《七重外壳》,作为对虚拟现实、元宇宙的想象,1997年能写到这个程度已经非常不易。刘维佳的《高塔下的小镇》,尽管文笔稍显稚嫩,故事本身反映了对文明进化的思考。夏笳的《2044年春节旧事》,着眼于日常生活,细致入微地讨论了视频直播等新技术的影响,令人回味;“相亲”一章读来格外有趣。
\par 当然集中也有我不喜欢的作品。陈楸帆《G代表女神》的设定在我看来已经足够变态了,姑且可以理解为对人类受感官欲望支配的讽刺(可以说是聚斯金德《香水》的现代翻版);韩松《地铁惊变》裹着批判的外衣刻意描绘人性的丑恶,更是引发强烈不适,实在是应了曹文轩的批评:彻底抛弃了对‘善’与‘美’的追求。
\par \rightline{2023年9月16日}


\section{奇幻文学}

\subsection*{平面国}
\addcontentsline{toc}{subsection}{平面国}
\par \textbf{Flatland: A Romance of Many Dimensions}
\par \emph{Edwin Abbott Abbott /  陈凤洁译} 
\par 终于读完高中时同学推荐的这部“数学幻想小说”。其中的数学内涵现在看来是基本的,高维立方体和高维球已经是大一的议题了。不过小说中提到,从三维空间能看到平面国物体的内部,由此类比,在四维空间中能看到我们日常所见的三维物体的内部,这一点我之前没有想到过。
\par 小说有一段振奋人心的献词,鼓舞人们探求更高维空间的奥秘。我想起一位专长于几何的数院同学告诉我的话:“对于三维空间中的几何学,你可以看《曲线与曲面的微分几何》;但你要是想了解更高维空间发生的事情,就要先学拓扑学作为基础。”因而我读到这一段时,竟感到心潮澎湃。
\par \rightline{2020年2月12日}

\subsection*{霍比特人}
\addcontentsline{toc}{subsection}{霍比特人}
\par \textbf{The Hobbit}
\par \emph{J. R. R. Tolkien / 吴刚译} 

\par 这本书首先吸引我的地方是标题、瑟罗尔地图等处出现的如尼文,在开始阅读之前,我先根据作者说明的提示研究了如尼文的解译。书中的如尼文大多与英文字母一一对应;一些两字母组合(TH, NG, EA, ST, EE)对应一个如尼文字母。自己动手解译瑟罗尔地图上的如尼文确是一件有乐趣的事,但后来知道这只是托尔金笔下阿尔达(Arda)世界语言的冰山一角。我找到昆雅语(Quenya)的网上词典和教程(这种语言有十种变格),这还只是托尔金众多人造语言中的一种!
\par 托尔金笔下的世界可称是虚构世界构建的典范示例,作者对其神话、纪元、地理、种族都有详尽的设计。尽管《霍比特人》一书是讲述世界较小时空范围内的一段具体故事(故事发生的大荒野只是中洲的一部分,中洲又只是阿尔达四块大陆之一;故事时间是第三纪元2941-2942年),作者在叙述中往往不经意间展示出世界的宏大(如对绿野之战、莫瑞亚矿坑之战、死灵法师的提及),而不是过度展开无关主要剧情的设定,让读者感觉作者还有大量构想没有讲出来。阅读之中我不时查阅魔戒Wiki了解背景信息,看来事实的确如此。
\par 小说情节的设定还是十分精彩的,尽管也摆脱不了陷入困境——惊险逃脱二元循环的套路(这一点想来竟和《西游记》有些像)。作者不对人物和剧情做理想化的设定,如梭林等多次让比尔博首先涉险,最后沉迷珍宝,不愿与人类、精灵和解等,亦是对人性弱点的反映。这使得作品更像是中洲历史的陈述,不像有的幻想小说戏剧性过强而有失真实。梭林临终前终于醒悟,他的遗言令人印象深刻:“如果我们都能把食物和笑语欢歌看得比黄金宝藏还重,世界将会比现在快乐许多。”
\par \rightline{2023年2月3日}

\subsection*{魔戒}
\addcontentsline{toc}{subsection}{魔戒}
\par \textbf{The Lord of the Rings}
\par \emph{J. R. R. Tolkien / 邓嘉宛、石中歌、杜蕴慈译} 

\par 所谓“《魔戒》三部曲”的说法并不准确,三本书其实是一部完整的奇幻史诗,即人类、矮人、精灵等自由种族对抗黑暗魔君索隆的故事。相比《霍比特人》,《魔戒》不仅篇幅更长、涉及的空间范围更为广阔,其叙事也更具崇高感:梭林等人冒险的目的是寻找族人失落的宝藏,主要敌人只是一条恶龙;而魔戒大战是中洲正义与邪恶势力的全面较量。一百余万字的三大卷书陪我度过了半年时光,我一度将护戒远征队的旅程与我面临的学业压力和任务相类比(前者显然更加艰险无望),并从冒险者们顽强乐观的精神中受到鼓舞。
\par 这本书题材属于奇幻文学,其文字读来却更像纯文学——书中不惜笔墨描绘征程上的每一处山川原野,细细讲述远征队的每一次遭遇。一些类型文学作品会标榜“全球读者平均X小时读完”,作为情节紧张刺激、扣人心弦的宣传;但我读《魔戒》却并不能也不想读得很快,甚至会对阅读的过程感到珍惜——等魔戒大战迎来大结局,也意味着一个时代、一个传奇的落幕。或许正因作者这种细致到有些迟缓的笔调,中洲世界才显得那样丰富和真实。主要种族之外的恩特、野人,以及神秘的汤姆·邦巴迪尔,更是体现了世界的丰富和广阔。书中穿插的诗歌也为故事增色不少,至尊戒之诗、波洛米尔的挽歌、“真金未必闪亮”,以及比尔博的行路歌(有多个版本,我最欣赏《霍比特人》中的版本,《霍比特人》电影的改编歌曲The Last Goodbye是对该诗的绝佳演绎),都是值得回味的佳作。顺便说,三位译者的译文也十分出色(特别是杜蕴慈译的诗歌,我对比过Wiki上的其它译文,这一译本明显更好),我庆幸自己在决定读《魔戒》时遇到了这个诞生不久的译本。
\par 《魔戒》再次让我为故事背后庞大的世界观设定所折服,阅读时我常常需要查阅Wiki,随着远征队的脚步学习阿尔达世界的历史和地理。书中时常追溯久远的历史,远古英雄人物的传奇与正在发生的大事件相互映照;正因为此,山姆和弗罗多在危难之时,仍意识到自己正在谱写为后世传颂的伟大史诗。附录 “列王纪事”对努门诺尔、阿尔诺、刚铎诸国历史的设定很像回事,外族入侵、政变、瘟疫应有尽有;对中洲语言发音的概述更是看得我一头雾水,不禁要说“太秀了”——这恐怕只有语言学家才编得出吧!我想《魔戒》不应仅仅用文学的标准来评判,至少还包括历史学和语言学的标准。
\par 《哈利·波特》中伏地魔的设定很可能受到了《魔戒》的影响。承载索隆法力的至尊戒,设定与中伏地魔的魂器相似,“不提其名者”的称呼更是完全吻合。对于善恶的对立,法拉米尔有一段精彩的独白:“我不会因其锐利而爱雪亮的刀剑,不会因其迅疾而爱箭矢,也不会因其荣耀而爱战士,我只爱他们保卫的对象——努门诺尔人类的城市。并且,我愿人们是为她的古老、美丽和如今的智慧而爱她,我不愿人们畏惧她,除非那感情如同人们对睿智长者之威仪的敬畏。”相对地,索隆利用恐怖建立统治,这与食死徒一党的“魔法即强权”亦有共通之处。(罗琳本人则说:“到《哈利·波特》写到最后一卷时,我仍然坚持认为自己不会超过托尔金,他的作品里有全新的语言和神话,而我的魔幻世界里没有这些东西。”)
\par 相比于自由人民在洛汗和佩兰诺平野的胜利,我觉得弗罗多和山姆能摧毁魔戒简直是极具戏剧性的意外。按理说弗罗多在奇立斯乌苟被捕,又在末日山被诱惑压倒戴上魔戒,二者任何一个都足以导致任务失败。谁知奇立斯乌苟之塔的敌人火并,自相残杀殆尽;咕噜誓言的诅咒最后应验,戴着戒指坠入了末日山。读弗罗多和山姆最后这段冒险时常让我感到绝望。面对坐拥天险又有制空权的索隆,他们显得太弱小了。索隆若是得知两个半身人到他眼皮底下摧毁了魔戒,不知作何感想?
\par 《魔戒》也是一部小人物的成长史诗。尽管在夏尔,梅里、皮平贵为大家族的公子,但霍比特人在中洲是一个存在感很低的种族。于是有了埃尔隆德在远征队出发时的话:“推动世界之轮的功绩,常常是遵循着这样的进程:当伟人的目光投向别处,是那些微渺之手因为感到责无旁贷而开始行动。”作者在大团圆的美好结局中穿插夏尔平乱一章,可谓独具匠心。萨茹曼已成丧家之犬,但霍比特人未经外敌考验,因而萨茹曼仍能在夏尔兴风作浪。但载誉归来的梅里、皮平、山姆历经磨炼,已经具备了英雄的智慧和勇气。他们领导村民战胜了恶势力,并在日后真正成为了霍比特人的领袖,被冠以统领、长官、市长等头衔,读来令人鼓舞。
\par 作为为数不多的爱情戏之一,伊奥温请战一节写得十分精彩。贵为洛汗公主的她不能直接表达对阿拉贡的倾慕,于是请求一同前往亡者之路,对危险、战斗和荣誉的渴望都可以翻译成爱情。阿拉贡看破不说破,委婉道出“我心牵挂之地”在幽谷。末了,伊奥温说:“跟随你去的人如此选择,是因为不愿与你分离,因为他们爱你。”是谁刚说要跟去来着?第二天临行前,伊奥温顾不上其它,跪地请求同行;阿拉贡为公主本人、为大局考虑,依然不肯。作者的评述点到为止却意味深长:“只有那些深深了解他又离得很近的人,才看得出他所承受的痛苦。”
\par 我的中洲之旅暂告一段落,但我还愿去读《精灵宝钻》《未完的传说》,了解中洲的更多历史故事。我对黄金之王阿尔-法拉宗远征阿门洲一节尤其感兴趣。为什么试图推翻神灵、追求长生被定为恶行?平等的价值观为何能无视人与神之间的不平等?正如阿拉贡去世之前,面对生离死别的阿尔玟终于理解了“人类的堕落”。
\par \rightline{2023年8月12日}




\subsection*{时间的皱折}
\addcontentsline{toc}{subsection}{时间的皱折}
\par \textbf{A Wrinkle in Time}
\par \emph{Madeleine L’Engle / 黄聿君译} 

\par 得知这本书是缘于Ian Stewart在《给年青数学人的信》中的推荐。作者想象丰富,善于刻画独特的人物形象,比如敢跟校长叫板的梅格,拥有不逊于成人的见识和勇气,带着哥哥姐姐一起冒险的查尔斯·华莱士;总喜欢引用名言来表达观点的谁太太也让人印象深刻。整体而言,这是一部精彩、令人愉悦的童书。最出彩之处或许是对卡玛卓兹星社会的刻画,所有人同质化地生活以达到严格的秩序,智能体“它”控制并统一所有人的思想。这使人反思个体独特性对于社会的价值:应当追求“平等但不相同”,正如啥太太的比喻:人生如同写十四行诗,是格律之下的自由抒写。
\par 这本书曾荣获纽伯瑞金奖、美国国家图书奖,或许有“政治正确”的因素:凯文和梅格在对抗试图控制意识的 “它”时,背诵了《葛底斯堡宣言》和《独立宣言》的句子;书中更是多处体现宗教观念,明明《圣经》都没有处理一个星球以外的问题,作者还是让宇宙中遥远的星球也颂唱耶稣的福音。时空皱折,或者“超时空挪移”的设想触及了维度的概念,即在高维空间中,两点距离可能比低维空间中短;但作者的解释反而会误导小读者,什么三维是二维与二维相乘,五维是四维与四维相乘,又把抽象的四维空间与加入时间维的Minkowski空间混为一谈,完全说不通。故事中对“黑暗势力”的设定也十分含糊,谈到对抗“黑暗势力”的人又把科学家、艺术家、哲学家统统纳入(而且第一个提到的是耶稣)。最后“爱”战胜一切的结局,只能说很童话。
\par \rightline{2023年10月12日}


\subsection*{哈利波特与凤凰社}
\addcontentsline{toc}{subsection}{哈利波特与凤凰社}
\par \textbf{Harry Potter and the Order of Phoenix}
\par \emph{J. K. Rowling / 马爱农、马爱新译} 
\par 在2017年暑假终于读完《纳尼亚传奇》后,我开始读《哈利波特》系列。比起同龄人,20岁开始读这七本书显得有些晚。小学阅读课上,曾有一名同学借给我《哈利波特与魔法石》看,我看了十几页似乎觉得并没什么意思?而我渐渐发现罗琳构建的这个魔法世界对我们这一代人,实在有着广泛的影响。初入大学,一位同学在自己的创意画上写下“Accio GPA”的宣言;我也曾见到有人对着宿舍的门大喝一声“Alohomora”。高中时一位热心哈迷向大家介绍了Pottermore网站(现在叫Wizarding World),其中有一个“分院测试”项目。我最喜欢的学院是Ravenclaw,但分院测试的结果是Hufflepuff,想来那的确更符合自己的特质。
\par 言归正传。七部小说的故事情节,逐渐从校园生活转变为残酷的战争;《哈利波特与火焰杯》中伏地魔的复活,可谓是一个转折点。在《哈利波特与凤凰社》的神秘事务司之战中,哈利和同学们与食死徒展开真正的对抗,而D.A.是他们得以这样做的基础。我觉得D.A.的创建和活动,实在是这部书的Highlight,正如现实生活中那些能定期开展活动、成员自发参与的学生组织,常常给参与者带来真挚的友谊和难忘的回忆。
\par Fred和George Weasley,通过一个“飞向自由”的情节,把一直以来表现出的性格和志向发挥到了极致。新出场的Luna Lovegood也是我比较喜欢的一个角色,她有着鲜明的不同寻常的特质,令人印象深刻。
\par \rightline{2020年3月13日}

\subsection*{哈利波特与“混血王子”}
\addcontentsline{toc}{subsection}{哈利波特与“混血王子”}
\par \textbf{Harry Potter and the Half-Blood Prince}
\par \emph{J.K.Rowling / 马爱农、马爱新译} 
\par 田先生在他的课上说,《哈利波特》绝不仅仅是一部魔幻小说,它有着深刻的现实意义。快要读到系列尾声的时候,我想起了这句话。与伏地魔的两次战争被称为“第一次巫师战争”和“第二次巫师战争”;目前人类历史上被冠之以“世界大战”的战争,也恰好是两次。伏地魔和食死徒,邓布利多和凤凰社,这两个阵营有什么鲜明的差异呢?伏地魔利用恐怖统治他人,这一点在德拉科身上体现得很突出:他虽声称成功杀死邓布利多将在食死徒中获得无上的荣耀,却同样在天文塔上说出伏地魔以杀死全家相逼。此外,伏地魔及其党羽宣扬纯血统巫师的优越,这在人类历史上也似曾相识;而邓布利多一方不仅对麻瓜出身的巫师没有偏见,对于家养小精灵等其它的生命也能表示出尊重。
\par 这部小说的主线之一是邓布利多向哈利展示伏地魔的过去,并带他走上寻找伏地魔魂器的道路。他没有把这份重任托付给学校里其他资深的巫师,或是法术高强的傲罗,而是选择了哈利和他的朋友们。这一点是耐人寻味的。从寻找挂坠的过程看,有不少环节是哈利一个人难以过关的(邓布利多本人也说“跟我的力量相比,你的力量恐怕可以忽略不计”)。但从哈利在随邓布利多出发前对罗恩、赫敏的交代中,我看出他的从容镇静,这代表着他正走向成熟。“分一点给金妮,替我向她说声再见。”这部小说在校园恋情上着墨不少,而我觉得其动人之处则在于大敌当前的危险、斗争的重任给恋情带上的悲壮意味,这是一般的校园恋情所没有的。
\par 参加天文塔之战的学生,恰恰是一年前在神秘事务司抵抗食死徒的那六人。有的地方管他们叫“D.A.六杰”。为什么其他人没有来并肩作战呢?诚然,在严峻的形势下,明智的人不难明白要学习防御本领用于自卫。而主动加入与恶势力的抗争,则需要真正的勇气。读到小说结尾哈利与金妮的对话时,我想起这样的诗句:“弃身锋刃端,性命安可怀?父母且不顾,何言子与妻。名编壮士籍,不得中顾私。捐躯赴国难,视死忽如归。”
\par \rightline{2020年4月3日}

\subsection*{哈利波特与死亡圣器}
\addcontentsline{toc}{subsection}{哈利波特与死亡圣器}
\par \textbf{Harry Potter and the Deathly Hallows}
\par \emph{J.K.Rowling / 马爱农、马爱新译} 
\par 小说的前六部都写到“校园生活”,有上课、考试、魁地奇比赛这样的插曲;相比之下,最后一部的情节显得紧凑得多。第六部写完,至少有四个魂器需要处理;罗琳还嫌内容不够充实,又加入了“死亡圣器”。应该说“找齐几件东西”是幻想故事中拉长篇幅的常见技法,比如“虹猫蓝兔”的七剑合璧、《福娃》先找如意碎片,再找“精神力量”等等。这种方法容易把本来完整的情节分段化,一旦处理不当,就有长篇小说退化为短篇小说集之嫌。不过在我看来,罗琳的“找齐魂器”还是比较成功的:里德尔的日记给了第二部剧情一个更明白的解释;哈利等人摧毁的四个魂器,其方法各不相同,而且在本书中情节的节奏逐渐加快:寻找和摧毁挂坠盒已经用了一半篇幅,金杯、冠冕和蛇则是在霍格沃兹之战的硝烟中被接连摧毁的。
\par 斯内普的反转可谓是作者为读者设的一个圈套——作为双面间谍,你最后说他真正效忠哪一边都是合理的;特别是斯内普是一个“高超的大脑封闭术师”,邓布利多和伏地魔信任错了人都是可能的。斯内普在亲手杀死邓布利多后,又击伤乔治,读者怎么可能保留着他是正面人物的猜测呢?
\par 通过斯内普的故事,作者似乎在传递这样的认识:我们常常谈论选择、价值观乃至信仰,但行动时常是被情感驱使的。斯内普在自身的成长中并没有选择邓布利多一方的博爱,而是走向了伏地魔的“巫师至上”;对莉莉的爱却使他最终转变了阵营。马尔福夫妇大战中的“一切为了儿子”,不再关心他们的“主人”是不是胜利,同样出于此。现实生活中是否当真如此呢?我只知道,自己也会因为一个人而对一个城市、一门学科心向往之,再说不出它们的坏话。
\par “不要低估爱的力量。”在作者的笔下,哈利在成长中不断体会亲情、友情、师生情以及爱情的。我想,邓布利多和伏地魔的不同,不是简单的正义与邪恶、光明与黑暗。伏地魔不懂得爱,甚至随意屠戮自己的追随者,削弱自己的力量。如果巫师真的谋求统治麻瓜,当魔法遇上巫师瞧不起的先进科技,谁能获胜还是个未知数;但不去这样做的出发点是共情。作者似乎希望告诉孩子们,真正的爱将引导一个人走上正确的道路。
\par 在“第二次巫师战争”中,邓布利多无疑是那个幕后的统帅。事情的发展远远超出了“神机妙算”可以把握的范围,想来很多时候画像里的他也是见机行事吧。至于死去的人如何做到在画像里活着,难道就像模仿人思维的神经网络可以在人死后继续写博客?
\par \rightline{2020年5月15日}


\subsection*{倒悬的天空}
\addcontentsline{toc}{subsection}{倒悬的天空}
\par \emph{程婧波} 

\par 程婧波的一些作品兼具科幻与奇幻风格,而我恰好同时喜欢科幻和奇幻。购买此书的另一原因是作者题签了一句颇浪漫的话:“亲爱的读者:写这本书的我,读这本书的你,137亿年前属于同一粒星尘。”此外作者的创作理念也让我有共鸣:她称自己并不创造故事,只是记录故事。的确,当角色、设定和灵感齐备,故事会自己生长出来。
\par 这本合集包含五个短篇和一个中篇。前三篇(即“行星三部曲”)体现了奇幻与科幻结合的风格,这种在虚构星球设计的新世界是我感兴趣的,但读来只有《艾罗斯特拉特的雨》可称佳作:按“幸福指南”举办婚礼的新娘,竟不知道还需要一位新郎;星球的开拓者为了制造降水蒸干了海洋……故事展现了异世界的独特风味。《倒悬的天空》情节过于简单,《萤火虫之墓》又令人费解:神通广大到能熄灭恒星,制造全宇宙的大灾难,居然不能给心上人传个话?《赶在陷落之前》读来也让我疑惑,巨龙骨架拖动洛阳城,让它永远是黑夜?你能把洛阳拉过大西洋?后来说这是主人公的梦。好吧,那没事了。
\par 占全书一半篇幅的《去他的时间尽头》重访了柳文扬《一日囚》中的时段循环主题,不过设定有所不同:(1)人死后来到一个新的平行世界,回到死亡时刻14h之前;再次到死亡时间时,重复上述过程,共137次(这里作者故弄玄虚地引用了精细结构常数);(2)若遇到同处于该过程的人,则循环的时间段变为两人的并集;(3)137次循环之中记忆不清除,但循环结束去往新世界时清除循环期间的记忆。如此繁复而反直觉的规则显然服务于网文般的恋爱桥段,最后落脚点是两人是选择保留一起经历的记忆,还是再度相逢的可能。我以为这篇属于“可以一读”,谈不上值得推荐,更不至于就能得了华语科幻银河奖、星云奖最佳中篇。既然得了,只能说中国科幻整体水平的进步还任重道远吧。
\par \rightline{2023年11月13日}


\section{推理、悬疑文学}

\subsection*{东方快车谋杀案}
\addcontentsline{toc}{subsection}{东方快车谋杀案}
\par \textbf{Murder on the Orient Express}
\par \emph{Agatha Christie / 郑桥译} 
\par 这是我读的第一本阿加莎推理小说,阅读时间在七八个小时。书的前两部平铺直叙,节奏不紧不慢。波洛对不同的乘客采用不同的讯问方式;频频用诈,在不经意间得到需要的信息,展示出高超的技巧。他与布克先生、康斯坦汀医生的对话也不时带着些反讽和冷幽默。推理进程在最后六分之一的篇幅突然加快,直到谜底揭晓,颠覆读者的认知。证词中埋下的伏笔在波洛的分析中一点点揭开;直到最后波洛阐述结论,仍有我没有注意到的问讯细节成为破案的线索,令人拍案叫绝。
\par 这本书之所以能成为阿加莎最有影响力的名作之一,首先在于其颠覆性的设计。阅读证词时,我思考的前提假设是:大部分证词是可靠的(除了一两个可能的凶手)。而作者恰恰推翻了这个前提假设:车厢上的十五位乘客和列车员,除去侦探和死者,都是案件的策划者。他们的行动和证词只是给侦探演了一个难解的故事,以此为基础推理什么时间线、不在场证明,都是推了个寂寞!而谨慎、细致的波洛从护照上的油渍入手,挖掘出了乘客隐藏的身份和作案动机。以此为基础,死者伤口的疑点就得到了解释;再回头看来,印在精装版封面上的麦奎因先生的话“这趟列车可真是座无虚席啊!”竟是整个故事的大伏笔。此外,侦探小说中的故事常常是悲剧的,其揭露的犯罪显示出人性阴暗、邪恶的一面;而这本书的谋杀却是正义的伸张,我想这也是读者愿意看到的。
\par 波洛对金钱的态度值得赞赏,在拒绝雷切特的重金委托时,他说:“我在事业上很走运,所赚的钱完全可以满足我的现实需要和各种任性的想法。我现在只接受感兴趣的案子。”
\par \rightline{2022年11月13日}

\subsection*{尼罗河上的惨案}
\addcontentsline{toc}{subsection}{尼罗河上的惨案}
\par \textbf{Death on the Nile}
\par \emph{Agatha Christie / 张乐敏译} 
\par 读完《东方快车谋杀案》后,我很快又在一个周末分两天读完了《尼罗河上的惨案》。第一天睡前担心睡不好觉去看了剧透,但即使知道凶手,读波洛的推理过程依然十分过瘾。
\par 这部小说的案情虽没有《东方快车谋杀案》那样颠覆性的构思,却也十分巧妙:西蒙和杰奎琳互相制造不在场证明,相当程度上成功转移了大家的怀疑。穿插其中的珠宝盗窃、国际危险分子,也为案件增添了不少复杂性。但波洛说得好:“清除外表的杂质,以便发现真相。”通过对每个人言行、心理细致的体察,首先解释一些无关的疑点,剩下的将指明真相。波洛用考古发掘的例子说明这个道理,我想或许是受了阿加莎考古学家丈夫的影响。杰奎琳的话同样令人回味:“每个人都得追随自己的星星,不管它引导我们走向何处。”这一无奈之语,道出了盲目、失去原则的爱引发的悲剧。她接着说“那是一颗坏星星,那颗星星会掉下来”,分别暗示了西蒙的作案蓄谋与失败结局,也是十分高明的伏笔。
\par \rightline{2022年11月20日}

\subsection*{无人生还}
\addcontentsline{toc}{subsection}{无人生还}
\par \textbf{And Then There Were None}
\par \emph{Agatha Christie / 夏阳译} 

\par 《无人生还》是阿加莎作品中恐怖色彩比较浓的一篇,阿加莎外孙的推荐序说它会使读者“被震惊和恐惧牢牢钉在原地”。我读来倒未感到难以承受的巨大恐惧,但孤岛上十个人接连死去的离奇案情的确牢牢吸引了读者,让我迫切地想知道真相。
\par 隆巴德是维拉开枪打死、维拉是悬梁自尽可以确定;对于其余八人的死,我想到了凶手不止一个的可能性:没准像《东方快车谋杀案》那样有隐藏的人际关系,B之前害死了A的朋友所以A杀了B,A之前害死了C的朋友所以C又杀了A呢。结果是和《尼罗河上的惨案》相似的“诈死”设计,作者巧妙用阿姆斯特朗医生的失踪转移读者的注意力,等阿姆斯特朗遗体被发现,如果不停下来仔细思考,跟着快节奏的叙事一路走来的读者很可能像维拉和隆巴德一样,头脑已不太清醒了。这时其实布洛尔的死是很关键的暗示,即除他们之外还有人存活。 
\par 除了穿插“诈死”的设计外,小说的出彩之处还在于作者成功运用暗黑童谣“十个小士兵”设计了连环命案(对应减少的士兵瓷人更烘托了恐怖气氛),以及对“法外正义”的揭示。与之对应,如此精巧复杂的情节给合理解释带来了困难。回头看小说开篇法官在火车上的心理活动,就有些不自然;十个人的行动是很难完全预知的,法官计划的执行过程中可以问出很多个“如果”。如果维拉受惊时大家没有都上去看呢?如果医生没有与法官合作,甚至出于恐惧把法官推下了悬崖呢?如果大家在搬运法官“尸体”时发现了诈死呢?士兵岛谋杀案的构思富有戏剧性,而其顺利执行似乎同样需要戏剧性。
\par \rightline{2023年2月14日}

\subsection*{斯泰尔斯庄园奇案}
\addcontentsline{toc}{subsection}{斯泰尔斯庄园奇案}
\par \textbf{The Mysterious Affair at Styles}
\par \emph{Agatha Christie / 郑卫明译} 

\par 这是阿加莎的首部侦探小说。尽管没有《东方快车谋杀案》那样戏剧性的情节,或是《尼罗河上的惨案》《无人生还》中对人性的反思,斯泰尔斯庄园一案在推理方面已经有了很高的水准。案件有足够多的细节和疑点,有玛丽在案发当晚闯入死者卧室,凶手伪造证据嫁祸他人,家庭成员作伪证洗脱嫌疑;唯有细致的思考和推理才能将混乱的证物和证言分离成不同的链条,最终指明真相。读者的思路大概会受黑斯廷斯影响,这个立志成为大侦探的叙述者会像一般读者一样做出一个个推测,从阿尔弗雷德,包斯坦医生,劳伦斯,再到约翰(作者有意引导读者依次接受这些推测,特别是约翰被捕时用了一些叙述诡计,读来我也以为波洛认为约翰就是凶手,只是证据不足);然而每一种推测都不能解释所有疑点。最后的真相既出乎意料,又令人信服。经典的本格推理读来还是过瘾的!
\par \rightline{2023年8月23日}

\subsection*{嫌疑人X的献身}
\addcontentsline{toc}{subsection}{嫌疑人X的献身}
\par \emph{东野圭吾 / 张舟译}

\par 开始读东野圭吾的缘起是看了缪时客出品,冒海飞、王培杰、王骏迪等主演的同名音乐剧,因此我是在完全剧透的情况下看完此书的。小说篇幅相对于不甚复杂的推理来说有些长,作者以全知的第三人称视角详细刻画了警察、侦探、凶手间的多轮交锋,以及角色的心理活动。小说所表达的不仅是推理或者诡计本身,还着力刻画了花冈靖子、石神哲哉两人的生活状态和精神世界。总之社会派推理就是话多哈哈。
\par 小说开始将富{\CJKfontspec{simsun.ttc}樫}慎二被杀的真相和盘托出,读者似乎是在已知真相的情况下看警方如何寻找答案。但是你以为的真相就是真相吗?关键的反转其实有点叙述性诡计的成分:前面始终没提到富{\CJKfontspec{simsun.ttc}樫}慎二被杀当晚的日期,读者其实并不知道警方断定的死亡时间是第二天。
\par 要在警方的反复盘问中掩盖真相是困难的,稍不注意就会露出破绽。一个例子是草{\CJKfontspec{simsun.ttc}薙}俊平去学校找石神委婉地确认不在场证明,石神先是故意说错了案发地点,接着对自己被要求提供不在场证明表示困惑和意外。这才是局外人该有的反应!作者对细节的构思,称得上十分细致了。
\par 花冈靖子虽是杀人凶手,想来能得到读者一定程度上的同情。最后的结局也令人唏嘘,靖子无法承受良心的不安前往自首,兜兜转转石神白杀了个人。小说也揭示了这样的社会问题:如果靖子和美里没有对富{\CJKfontspec{simsun.ttc}樫}动手会怎样?在一个无赖男子面前,母女的弱势处境显而易见。和工藤邦明结婚能得到某种“保护”,但她们的生活只能依赖可靠丈夫的保护吗?
\par 最后,书中论及P与NP问题有些曲解:问题本身不是验证答案与找出答案“哪个更容易”,而是前者是否真的比后者更容易。看到汤川学给业余数学家石神一份黎曼猜想伪证,石神检查了一夜发现“对素数分布的刻画存在本质问题”,我笑了,东野你还算下了点功夫嘛。
\par \rightline{2023年11月7日}



\subsection*{煞风景的早间首班车}
\addcontentsline{toc}{subsection}{煞风景的早间首班车}
\par \emph{青崎有吾 / 郑晓蕾译} 

\par 这部短篇小说集包含五个日常推理故事,除与全书同名的第一篇外,其余都是高中生活的微妙小事,推理思路比较简单。第一篇的推理较为精彩,加藤木与煞风景(没错,女主叫煞风景)在早班车上展开一场特别的竞赛:看谁先推测出对方乘早班车的原因。但真相却不尽合理:闯入美术馆这样的小事,会需要借助公共场所监控提供不在场证明吗?每天早上到公园看看前一晚的痕迹,就能找到罪犯?何况距案发时间也是越来越久。值得一提的是,警方对叶井的性侵案无能为力,两位主人公通过坚持不懈的调查锁定了罪犯,并通过报复实现了“法外正义”,在直面社会问题的同时提供了一个大快人心的结局。
\par 文末的后记是全书出彩之处,在讲述第一篇结局的同时,作者还交代了其它四个故事的后续。五条没有交集的故事线在这里交汇,给人一种时空共现的奇妙感受:不同的人生活在同一个城市,谈论、到访相同的场所;而每个故事的结局都体现着亲友、同窗间的温情。在我看来,这本书较好地代表了校园轻小说+日常推理的风格。
\par \rightline{2023年10月1日}

\subsection*{心灵侦探城{\CJKfontspec{simhei.ttf}塚}翡翠}
\addcontentsline{toc}{subsection}{心灵侦探城{\CJKfontspec{simsun.ttc}塚}翡翠}
\par \textbf{Medium}
\par \emph{相泽沙呼 / 罗亚星译} 

\par 对日常推理情有独钟的相泽沙呼,这本书转向本格写起了凶杀案,甚至不乏连环杀人魔这样的变态凶手。我是在一个周日晚上拿起这本书的,计划每晚读一点,给平淡的生活增添一些刺激。谁料这对我来说过于刺激了:三个案件一个meta,每部分看不完都很难放下,不觉就过了每晚规定的睡眠时间;这四天里,就是躺下了精神也仍处于兴奋状态,就算不回想案件也会想这想那,很久才能入睡。看来读此类作品还是要找周末白天的整段时间。
\par 本书的成功主要在于meta部分颠覆全书的惊人反转。前三个案件读来让人进入了“设定系推理”的状态,城{\CJKfontspec{simsun.ttc}塚}翡翠通灵得到关键信息,香月则运用逻辑推理建立指证凶手的证据,帮助她让能力发挥作用。两人之间既有作为破案搭档的相互鼓励,又有暧昧对象间的情欲撩拨,读来竟有些温情的感觉。第二案的镜子推理读来细致精巧;第三案推理部分很弱,倒是香月、翡翠千钧一发之际阻止凶手再次作案,读来惊心动魄。
\par 到了meta部分,香月带翡翠调查连环杀人案,突然露出真面目要对翡翠下手:那个杀人魔就是我。似乎站在正义一边的侦探突然成了最黑暗残暴的凶手,这如何不让人脊背发凉!谁知看似无力反抗的灵媒姑娘这时候放声大笑:我完全不会通灵,所有给出的事实,都是在极短时间内用有限的线索得出的!我是个比你强十倍的侦探!姑娘一边嘲笑香月的无能,一边重访三个案件给出精妙的推理,前文伏下未用的线索一一回收,读来令人大呼过瘾。(值得一提的是,三个案件结尾看似不经意的英文标题,实则暗示了真正的推理要点。)不仅如此,翡翠的全部表现都是表演,目的是看清香月的真面目:假装自己羞怯、孤独、渴望信任,是为了引起感情投射(之前读来还以为这种强弱关系的设置只是为了迎合男性读者的口味);让自己的通灵表现能被分析出逻辑,让“骗术”更易被相信(读者也被诱导接受这些设定,相信灵界的存在);甚至每一次暧昧的挑逗、对死亡预感的言说,都是精心设计的!翡翠不仅是个杰出的侦探,还是个货真价实的魔术师,在最终对决中,她以炫目的手法让杀人魔无计可施、一败涂地。最后游乐园门票的出现,又给翡翠强大的形象加入了一些温情。作者编排整套设定和情节的思考量着实惊人,难怪此书能获得多项年度大奖,被作者自豪地称为代表作呢。
\par 在meta部分真正的推理中,作者借翡翠之口致敬了侦探史上的经典,也反驳了对日常推理的批评(第二案的关键就在于细节的推理——一样在侦查中并未出现的证物)。有趣的是,第三案作者利用制服式样设计推理,然后借翡翠之口自嘲:“如果有那么一个对女高中生制服知之甚详的男性推理小说家……”原来你也知道自己是个变态啊。
\par 回到几个案件本身,其实第一案的凶手和动机都不难猜——总不能突然冒出来一个没登场过的人物,而登场过的女性只有小林舞衣一人,动机也很常见了。第二案看到学徒给作家老师看作品一节,我也就猜到动机是剽窃创意——也不是第一次见了。后两案都是冷血的连环杀手,其动机之扭曲变态常人无法理解。我的问题是,尽管真实历史上此类杀手的确存在,但真的有必要虚构出更多的恐怖杀人魔吗?为了让人对人性的阴暗面更加畏惧而不敢言说?呼延云这样评价阿加莎作品:“作品虽然写谋杀但并不血腥;虽然写到暴力但并不残忍,虽然写到感情但几乎很少煽情,很像英式下午茶,非常优雅和从容。”正因为此,阿婆的推理读来给人更纯粹的享受,这也是我无法给此作更高评价的原因。遗憾的是,当代推理悬疑作品不乏此类血腥、残忍、变态元素,我想这段话我还需要说很多次。
\par \rightline{2023年12月24日}


\subsection*{废墟中的少女侦探}
\addcontentsline{toc}{subsection}{废墟中的少女侦探}
\par \textbf{Matsulica Majoruka}
\par \emph{相泽沙呼 / 林千早译} 

\par 我读的第一部日常推理小说,也可以说是加入推理元素的校园小说。其情节似乎偏向社会派,即反映校园霸凌等人性的阴暗面(前两篇),关注校园中“边缘人”的处境(第三篇)。第二篇对性侵的揭露读来十分沉重,比暴行本身更令人悲哀的是受害者忍气吞声,施暴者因而没有受到应有的惩罚。《黑箱》的作者伊藤诗织公开承认被性侵竟是一种先例,足见背后的社会问题多么严重。大概是为了迎合通俗读者的口味,作者刻意描绘茉莉花对柴山的诱惑与支配,以及柴山对茉莉花露骨的性冲动和性幻想;还好这是校园读物而非成人流行小说,所以总是“发乎情,止乎礼”。
\par 写法上有两点值得注意:一是所谓“连锁式短篇”,各篇之间相互独立又有所承接,最后一篇则用到前面各篇的伏笔,有些像谜题比赛中每章节的meta题目(据译者介绍,这种写法始于若竹七海《我的日常推理》)。二是情节上的映照,即相似、相关的情形在两个人身上发生。第二篇是秋和前辈的遭遇与柴山对茉莉花的冲动;第三篇是有子与柴山渴望被班级接纳的相似处境;第四篇是柴山姐姐的自杀与茉莉花对生命的不珍视;第一篇原始人的逃跑与柴山的“逃离”算是比较弱的映照。
\par \rightline{2023年8月13日}

\subsection*{废墟中的少女侦探2}
\addcontentsline{toc}{subsection}{废墟中的少女侦探2}
\par \textbf{Matsulica Mahalita}
\par \emph{相泽沙呼 / 靳园元、魏寒冰译}

\par 其实我很想看看作者怎么让本该毕业的茉莉花还留在废墟大楼里,结果真就留级,也没有这样做的解释,有点失望。续集结构与初集类似,三个短故事加一个meta:短故事除了第一篇外,推理思路都很简单;对社会问题的反映弱了些,主题更多是学生面临的处境、人与人之间的联系。meta是全书的高潮,伏笔铺垫和反转都很出彩。
\par 除初集的小西菜穗(不太懂为什么在续集里被叫做“小直”)外,主人公柴山在续集里交到了高梨千智、松本茉莉香、村木翔子等更多朋友。长篇故事需要更多的人物,或许作者在写续集时已经有意在为第三作做准备了。茉莉花的语言挑逗更为直接,行动也更向施虐者靠拢,在我眼中减少了一些可爱。此外我还注意到柴山在每篇最初受茉莉花之命调查的六个怪谈,只有初集中的“原始人”给了解答,这也是作品的不尽完美之处。
\par \rightline{2023年8月20日}


\section{寓言、神话}

\subsection*{拉封丹寓言}
\addcontentsline{toc}{subsection}{拉封丹寓言}
\par \textbf{Fables}
\par \emph{La Fontaine / 苏迪译} 
\par 书中的寓言故事篇幅短小,大多只有半页纸。很多故事采用拟人手法,饶有趣味。不少生活中听到的故事在本书找到了出处。值得一提的是,我读的本子收录了Percy J. Billinghurst精美的版画插图,绝大多数都可以直接拿去印明信片,单凭这些插图也值回了买书的钱。
\par 简要总结一些令我印象深刻而不那么著名的篇目。《死神与伐木工》一篇写世人相较于死亡宁愿生而受苦,与周国平人生寓言中《落难的王子》有相通之处。《老鼠的议会》讽刺议会讨论积极却无人执行,令人捧腹。《狐狸和山羊》讨论了合作者的背叛,让我想起玩踩气球游戏的经历。(有人说,当自己的盟友突然踩向自己气球的时候,那是怎样的心理创伤啊。反观古罗马史中的政治同盟,还不是要决个你死我活吗。)在《狼和牧羊人》中,一只因残杀羊受到恶名的狼发现“自诩为羊群保护神的人类也在吃烤羊肉”,最后得出结论“狼的唯一错误在于它不是所有生命的主宰”,角度独特。《橡子和南瓜》十分滑稽。《两只鸽子》写远行的鸽子途中遭遇各种危险,使我思量自己周游世界的计划还是限于相对安全的地方为好。《要求有个国王的青蛙》道出了君主制的弊端,即难以保证君主的贤明。为数不多的几篇体现了作者对于一些议题的立场,如《掉进井里的占星家》批评了占星术的虚伪,还有一两篇意在说明“动物也会思考”。《寓言的威力》一篇在全书中有纲领性的意义,表明了创作寓言的动机:“有人说,世界衰老了;而我却认为,人们如同孩子,他们渴望那些浅显易懂、充满快乐的故事。”

\par \rightline{2021年2月6日}

\section{戏剧}


\subsection*{哈姆雷特}
\addcontentsline{toc}{subsection}{哈姆雷特}
\par \textbf{Hamlet}
\par \emph{William Shakespeare/朱生豪译}
\par 把一个涉世未深的青年推向政治斗争的漩涡,这命运是颇为残酷的;更何况丹麦王子哈姆雷特所要面对的是心狠手辣的僭主和不得不报的深仇。能勇敢地面对残酷的命运和黑暗的现实,已经十分不易;而斗争本身又是难以预知的棋局,稍有不慎便会身死人手,亦无法指望胜利者书写历史的公正。倘若真的遇到了这样的命运,也只有动用全部的胆识和智谋,向着漩涡而去了。在吕氏的威权下韬光养晦的汉文帝刘恒,继位时只有23岁;古罗马的屋大维登上政治舞台,和安东尼、雷必达争夺权力,也不过二十出头。命运会成就英雄,失败者也留下了为后人感叹的故事。哈姆雷特假充疯癫掩饰内心的愤怒,又在被押往英国的路上勇敢逃脱,都值得称赞;演戏影射之举虽有打草惊蛇之嫌,却也借此看清了国王的真面目。最后的同归于尽结局,更是可悲可叹。
\par 宗教对来世的描述影响着剧中人对生死的看法,从无神论者视角看来,国王祈祷时动手再好不过了。
\par \rightline{2022年2月16日}

\subsection*{罗密欧与朱丽叶}
\addcontentsline{toc}{subsection}{罗密欧与朱丽叶}
\par \textbf{Romeo and Juliet}
\par \emph{William Shakespeare / 朱生豪译} 
\par 这是我读的第一部莎士比亚剧作。2017年11月我曾在百讲看过TNT剧院演这部剧。记得出场后,我便问看过原著的同伴:那种可以装死四十二小时的药,原著真是这样写的吗?
\par 看书的时候,我笑着跟老爸讲,Capulet虽为贵族家长,还是会在宴会上说出那样的粗鄙之语。Romeo和Juliet的对白无疑是戏剧化的,若恋人真的要这样聊天,那不得累死。不过最深的感触和看演出时是一致的:那个时候的贵族衣食无忧,又不必有自己的志业。他们谈论爱情时,很轻易地谈到生死,并说出“爱情价更高”。相比之下,现代人的生活就丰富得多了。
\par \rightline{2020年2月23日}

\subsection*{威尼斯商人}
\addcontentsline{toc}{subsection}{威尼斯商人}
\par \textbf{The Merchant of Venice}
\par \emph{William Shakespeare / 朱生豪译} 

\par 这是我读的第一部莎士比亚喜剧作品,剧中不乏戏剧化的情节、令人捧腹的桥段,相信最后的大团圆结局也为观众喜闻乐见。但相较于《哈姆雷特》《罗密欧与朱丽叶》等悲剧,终究少了些震撼人心的力量。
\par 鲍西娅的选匣谜题确是有趣的设定,但细想来有些牵强:选对匣子能够表明求婚者对鲍西娅的爱吗?我原以为作者是想借铅盒子上的话说明爱情意味着甘愿为之牺牲(“选择了我”中的“我”指鲍西娅,而非铅盒子),谁知巴萨尼奥完全是凭着对时人表里不一的认识、对金银的厌恶选对了盒子,完全未提及牺牲二字。然而,错的是人们追逐金银的丑恶行径,金银本身有什么错呢?在小公主萨拉的幻想中,金子不是代表华丽和美好吗?阿拉贡亲王认为只有金盒子才配得上盛放鲍西娅的小像,这难道没有道理吗?若按照巴萨尼奥的逻辑,华丽的外表是靠不住的,可鲍西娅不就是美丽与智慧、品德兼具的例子吗?鲍西娅也不相信盒子能帮她选出正确的伴侣,而是称之为“命运的钳制”,或许正是缘于此吧。
\par \rightline{2023年1月8日}

\subsection*{恋爱的犀牛}
\addcontentsline{toc}{subsection}{恋爱的犀牛}
\par \emph{廖一梅}

\par 今年九月去蜂巢剧场看了这部剧的现场表演,从此进入话剧音乐剧坑,一发不可收拾。观剧之后依例买来原著剧本一读。
\par 这部剧叙事有些跳跃,马路与明明故事的发展与恋爱指导课、故事、格言相穿插,营造出文本复调的效果;对此作者坦言初稿其实是与传统话剧一样“好好讲故事”的,调整风格是为了让“锋利的语言之剑尽情挥舞”。我以为戏剧的语言既可以偏向小说,就像大多数电视连续剧中贴近生活的人物对白;又可以偏向诗歌,具有极强的感染力和冲击力,就像莎士比亚戏剧中一些“不太像日常对话”的话。这部剧的语言自然属于后者,并且兼具强烈的情感表达与犀利的讽刺。书中廖一梅的照片给人以清秀之感,谁知柔软的外表之下是如此有张力的文风。
\par 这部剧的情节起初是令我困惑不解的。自古以来的爱情悲剧,大多都是男女双方相爱而受到外界阻力。单恋不值得言说,因为这根本算不上爱情;正如台词中所说,“忘掉是一般人能做的唯一的事。”当然也可以选择不忘掉,继续守候,像金岳霖那样。但剧中马路的所为超越了道德的界限,明明与陈飞的关系也是一种不对等的、畸形的爱,很难让人有所共鸣。本剧将马路中大奖的情节作为高潮,但能否得到一个人的爱情本就不该与此有关。后来读书中所附编剧的解说,读一些网评,慢慢觉得应该把“爱情”这一层抽象掉,不去受“爱情圣经”赞誉的影响(或者说这一赞誉主要适用于语言;据说剧组有位女演员收到过包含剧中台词的情书);作者或许是想歌颂对美好事物和信念的坚持,即使这种坚持被世人视为偏执,也会给自己带来更多痛苦(准确说是痛苦和幸福都更为强烈)。开篇歌中唱的“爱情是多么美好,但是不堪一击”,由此看来正是对现代社会中人人都懂得“明智选择”的讽刺。
\rightline{2023年10月23日}

\subsection*{琥珀}
\addcontentsline{toc}{subsection}{琥珀}
\par \emph{廖一梅}

\par 《琥珀》是廖一梅“悲观主义三部曲”的第二部。本剧的一条主线是高辕带领十几个无业游民写情色小说《床的叫喊》,请“性体验写作”青年女作者姚妖妖署名。作者进一步打破传统戏剧的禁忌,不少台词是性爱的直接表达。在我看来这当然不构成作品的优点,但我读出了对“美女作家”“下半身文学”等当代文学现象的讽刺,剧中此书受大众热捧,亦是对公众审美的讽刺。另一条主线是小优因高辕被移植了故去男友的心脏而接近他,进而对他产生爱情(剧中的所谓“爱情”中欲望满足带来的快乐似乎占了主要的成分)。和《恋爱的犀牛》一样,爱情的产生似乎没有足够的动因和基础来让观众产生共鸣。正如医生所说,心脏只是血液循环器官,即使真的能移植,也并不带有人的主体性(尽管我们的语言赋予“心”精神层面的含义)。如果换成移植脑,我想这个问题会严肃且有意义得多。
\par 作者在手记中讲述了自己的悲观主义立场,即与世界保持距离,不存奢求。“面对生活,面对命运,我们以前是无能为力的,以后也一样无能为力。唯一可做的就是尽力保持一点尊严。”令人印象深刻的是,作者孕育新生命的过程使她“风度尽失”,想尽办法隔绝负面的新闻,希望孩子远离丑恶和苦难。这使得作者把悲剧结局改成了喜剧结局:高辕没有对生命消沉下去,而是接受了小优的爱,“因为你,我害怕死去。”
\par \rightline{2023年12月8日}

\subsection*{柔软}
\addcontentsline{toc}{subsection}{柔软}
\par \emph{廖一梅}

\par 《柔软》是廖一梅“悲观主义三部曲”的最后一部。剧情涉及变性手术,我不知道作者是否为此做了很多功课,对手术操作过程的描述是否科学,反正我是第二遍才强忍着读下去。从更为口无遮拦的台词中,我读到悲观厌世者对世态的讽刺、借女医生之口讲述的性别观念。作者称这部作品反映了自己对生命思索和追问的结果,即接受有缺憾的世界,“与生命握手言和”。遗憾的是我并未如作者所说,在戏剧结局中鲜明地感受到这一点;结尾处女医生的婚礼致辞想表达的似乎是婚姻(以及生活中的其它事物)没那么好,也没那么坏。
\par \rightline{2023年12月17日}




\section{诗歌}

\subsection*{仓央嘉措圣歌集}
\addcontentsline{toc}{subsection}{仓央嘉措圣歌集}
\par \emph{龙冬译} 
\par 一本颇费些周折才淘到的书。对这本书的兴趣多半来自歌曲《在那东山顶上》,最初听的是谭晶《看见》的版本,后来找到玛吉阿米藏音的版本,觉得这两位藏族歌手演绎得更有韵味。译者的翻译以忠实原文为原则,比如第24首“若要附和美人心愿,恐将失去此生佛缘;若是隐居山里修行,又会背离女子芳心。”由此观之,所谓“世间安得双全法,不负如来不负卿”一定程度上是一种再创作。这本诗集并非简单的情歌,有几首作品明显表达了敌对势力对自己的恶毒中伤。作者甚至认为诗中很多情语也只是一种隐喻,对此我对背景了解不深,尚难以判断。
\par \rightline{2020年7月23日}

\section{散文、随笔}

\subsection*{野草}
\addcontentsline{toc}{subsection}{野草}
\par \emph{鲁迅} 
\par 鲁迅大概是中小学课本选入篇目最多的作家之一了。当时学《从百草园到三味书屋》《社戏》《雪》《藤野先生》,以至后来的《孔乙己》《祝福》,并未觉有何格外伟大之处。直到大学国文读《铸剑》里的“哈哈爱兮歌”,始觉这奇文远非常人所能为;又读令人毛骨悚然的《墓碣文》:“待我成尘时,你将见我的微笑”,只得叹服——终究是第一流的文字。
\par 这学期旁听现代文学史课程,讲鲁迅时,教授从《野草》出发解读鲁迅的人生哲学和思想,讲演对书中多数篇目都有涉及。《野草》的作品大多篇幅短小,一些篇目我随堂即上网查找读完,于是干脆找来全书通读。不少篇目都有浓厚的象征意味,而意象的格调总体是阴暗、残酷的(甚至不乏血腥、恐怖);语言则不乏深刻、有力的句子,如饱含激情的演讲。当然也有篇目如《立论》《我的失恋》,让我了解到鲁迅创作的多样性,读到时忍不住当“笑话”转发给朋友。
\par 除之前读过的《墓碣文》,读来觉得格外欣赏的篇目还有《影的告别》《过客》《这样的战士》《淡淡的血痕中》。《影的告别》关注了“影子”这一意象,“黑暗会吞并我,光明又会是我消失”,很是巧妙。《这样的战士》则具有讽刺意味和震撼人心的力量。《过客》采用剧本体裁,当老人谈及前方的坟地,小孩子关注的则是野百合、野蔷薇,让我很有感触。《淡淡的血痕中》放到现在依然很有警醒意义:网络环境下的公众会震惊、愤慨,但似乎是健忘的;他们会追逐新的热点,而真的猛士“记得一切深广和久远的苦痛”。
\par \rightline{2021年8月12日}


\subsection*{孤独六讲}
\addcontentsline{toc}{subsection}{孤独六讲}
\par \emph{蒋勋} 

\par 看书的标题以为作者要对“孤独”这一生活哲学议题做严肃的探讨,其实是东拉西扯的聊天风格,每一讲中的章节篇幅都十分短小,读来有碎片感,适合作为睡前读物。更绝的是每讲最后都能扯到自己写的小说,还多处大段引用小说的原文,简直不如改名叫“小说创作谈”。
\par 其实只有第一讲有些接近一般理解的孤独,之后的章节其实是在讨论语言、暴力、思维、伦理等议题,至多可以算是产生孤独的情形。由于概念没有明晰的界定,作者会把“孤独”和特立独行、少数派等混为一谈;只要是多数大众不具备的认识,都可以称为孤独。对“暴力”的讨论也与生命力、冒险精神等混为一谈。
\par 抛开上述缺点,书中还是有一些值得思考的见解。我倾向于把孤独界定为“物理上无人陪伴,精神上无人理解”。在此意义上,同样可以有可怜的孤独和圆满的孤独(“独与天地精神往来”)。我赞同最后一章“先完成自我,再建构伦理”的观点,作者对伦理中失去自我的个体观察独到。事实上,“与自己对话”是贯穿全书的理念,学会独处才能更好地与人相处。此外,作者多处谈到对儒家文化的反思,如不鼓励特立独行、不尊重隐私、不正视死亡、压抑情欲等。还有“道德成为一种表演”,“模式化的语言和思想分离”,都是对一些现实情况的精当批判。
\par 在“暴力孤独”一章中,作者宣扬暴力是人性中被压抑的原始本性,并举出现代人通过看暴力电影发泄暴力欲望的例子。对此我并不赞同,即使这种原始本性的确存在,也可以通过精神修养将其超越。我不认为看电影中的暴力是为了满足某种欲望,而只是因为类似的黑暗在现实中的确存在,我们不能回避。不过作者指出合法的暴力可能同样残酷,而历史上也有过暴力被合法化,以至人性的黑暗面暴露无遗的事例,这一点的确引人深思。
\par \rightline{2023年2月17日}


\section{纪实、传记}

\subsection*{假如给我三天光明}
\addcontentsline{toc}{subsection}{假如给我三天光明}
\par \textbf{Three Days to See}
\par \emph{Helen Keller / 林海岑译} 
\par 《假如给我三天光明》是散文名篇,作者告诫读者“像明天将要失明那样去使用你的眼睛”,这样将看到之前从未看见过的东西。作者描述了三天光明时光的计划,观察的对象大致可归为亲友、自然、艺术、社会四个方面。我想对这个问题的回答,几乎就是回答“什么事物值得为之付出时间”,亦即“如何生活”。
\par 作者23岁时写就的自传《我生命的故事》占了全书绝大部分篇幅,介绍了自己从接受启蒙教育直至读大学的经历。印象深刻的一点是“首位成功者”的榜样作用:海伦得知挪威的盲聋女孩朗希尔顿·卡达学会了说话,立即下定学会说话的决心。而第一位以盲聋人身份获文学学士的海伦本人,也成为其他人的榜样。“首位成功者”用实际行动使可能性得到证实,其带来的希望和激励作用是巨大的。此外,海伦的成就固然离不开本人的坚强意志和刻苦努力,也离不开外部条件的支持。海伦的父母能为她聘请专业的盲人教师,海伦的朋友帮助她把需要阅读的书籍和资料制成盲文书,这不是每一个家庭都能做到的。海伦13岁时就在著名发明家贝尔的陪同下参观了世界博览会,这简直令人惊叹。海伦对大学教育节奏过快的评论引人思考,她认为“我们应该把教育视为一次乡间散步,从容不迫,敞开思想,尽情接纳天地万物。”但大学面对着在给定年限内完成人才培养、输送到社会的压力,更不必说被state-of-the-art裹挟着前进的学术界
了。海伦的观点也许只能作为终身学习者的理想。
\par 在《三论乐观》中,海伦阐释了她的乐观主义立场,她似乎认为乐观的态度对事业的成功、社会的进步是必要的,因而“有害的”悲观思想不应被传播。但一个观点是否正确和是否有利是两个问题。成年以来,我逐渐反思这种无充分依据的乐观:既然复杂系统的行为难以预测(特别是影响重大的偶发事件),真的可以说“明天会更好”吗?海伦说要相信人性,“成千上万的人一直在努力维持这世界的美好”。而我似乎觉得需要一套让个体利益与群体利益趋于一致的社会制度;个人生活得到改善,当且仅当他为维护社会美好、推动社会进步做出贡献。似乎有时候,绝大多数人“努力维持世界的美好”都是不够的。
\par \rightline{2023年1月1日}

\section{儿童文学}

\subsection*{小公主}
\addcontentsline{toc}{subsection}{小公主}
\par \textbf{A Little Princess}
\par \emph{Frances H. Burnett / 梅静译} 
\par 伯内特夫人的书,少年时曾读过《秘密花园》,很是喜爱。《小公主》接触过书虫系列的改写本,如今找到全译本来读,仍然觉得是个温暖感人的故事。女主人公萨拉是个典范的儿童形象:善良、礼貌、酷爱读书、多才多艺、富于想象;小说着力刻画的则是她不论处于富贵还是贫贱,都努力像公主一样行事,保持着一份美德和尊严。或许是读书使然(十一岁的女孩居然对法国大革命感兴趣),她有着与年龄不相称的成熟。她对学校的洗碗女仆说:“我和你一样都是小女孩,而我不是你,你不是我,只是个意外而已。”不久父亲破产、去世,她从饱受宠爱的富家女变成了一文不名的女佣,就住在洗碗女仆隔壁。势利的女校长对她极尽剥削、虐待,却没有想到一时失势的人还有东山再起的一天。阁楼宴会一幕一波三折,从现实破灭到美梦成真,亦是整个情节的缩影,具有温暖的浪漫气息。
\par \rightline{2022年12月14日}

\subsection*{精灵鼠小弟}
\addcontentsline{toc}{subsection}{精灵鼠小弟}
\par \textbf{Stuart Little}
\par \emph{E. B. White / 任溶溶译} 

\par 十多年前读过感人的《夏洛的网》,《精灵鼠小弟》是作者更早的一部童话作品。本作的故事和语言都足够风趣幽默,时常让人会心一笑。主人公的父母和兄长没有将像小老鼠的他当成异类抛弃,而是像对正常孩子一样关爱他(比如特意为他修改了蔑视老鼠的圣诞歌词);也许小读者会认为这理所当然,现在读来觉得是感人的亲情。鼠小弟的经历有小男孩成长的影子,如驾驶帆船劈波斩浪的英雄梦想,精心准备、充满美好幻想却最终泡汤的约会。作者刻意让故事在寻找小鸟玛加洛的旅程中戛然而止,其人生中 “追寻”的主题也与曹文轩的《根鸟》有相通之处。
\par \rightline{2023年12月30日}


\subsection*{窗边的小豆豆}
\addcontentsline{toc}{subsection}{窗边的小豆豆}
\par \emph{黑柳彻子 / 赵玉皎译} 
\par 好久没读儿童文学作品了,果然感觉阅读压力小了很多,可以一口气读几十页(大概是最近读论文的缘故)。这本书的版权页将其归类为“儿童文学——长篇小说”,但既然写的是真人真事,归入纪实散文或许更合适。
\par 书中的内容并不限于巴学园,但小林宗作校长的教育理念和方法应当说是全书的主线之一。我相信日常的自由活动时间、亲近自然或是接触生产实践的集体出游对小学生有益处,但对小学生能否主要通过自学奠定知识的基础抱有怀疑。巴学园的小班教学实验当然很难适应竞争激烈的当下社会,但如果高中三年完全以大学为目的,初中三年完全以高中为目的,小学六年完全以初中为目的,我想那也不是最理想的情况。
\par \rightline{2020年7月23日}

\subsection*{天蓝色的彼岸}
\addcontentsline{toc}{subsection}{天蓝色的彼岸}
\par \textbf{The Great Blue Yonder}
\par \emph{Alex Shearer / 吕良忠译} 

\par 这是一部试图阐释“死亡”这一主题的儿童文学作品,国庆假期期间约四小时读完。因车祸意外身亡的小学生哈里以幽灵的形式回到人间,先是因看到没有自己的世界如常前行而沮丧,之后从师友的纪念、亲人的悲伤中得到感动。哈里重见家人的一段,读来感人至深,让人更懂得珍视生活中的点滴,特别是和亲人朋友共度的美好时光。
\par 对于人死后的路径,书中的设定似乎是先到一个叫“他乡”的地方,之后自愿前往“天蓝色的彼岸”,代表个体意识的终结,有机体解体、回归自然。在他乡滞留的时间可以任意长,甚至能以幽灵身份重返人间(尽管不合规);这样做多是有未竟的心愿,如寻找失散的亲人。若真如此,没有挂念也何妨在人间再游荡个千百年。可赏美景,却不可品美食;可观棋,却不能下棋;能思考,却不能著述。想来难受;却也凑合。
\par \rightline{2022年10月3日}


\section{文学研究、文学史}

\subsection*{人间词话}
\addcontentsline{toc}{subsection}{人间词话}
\par \emph{王国维/ 徐调孚校注} 

\par 此书少数章节是对词的一般讨论,其余则似读词的批注合集,对一人、一篇乃至一句而发。可以看出作者读词十分广泛,除作者认可的名家李煜、冯延巳、欧阳修、苏轼、秦观、辛弃疾,书中还论及不少我看到字、号一头雾水,查出名字仍十分陌生的词人。我至今对词缺乏系统的阅读,如今读此书也只能观其大略。
\par 还记得《红楼梦》中林黛玉论诗词,说立意最为重要,词句新奇次之,格律相比之下并不要紧。 “质重于文”的观点我是赞同的,但这些讨论并未体现出词的美学内蕴。本书开篇即说“词以境界为最上”,但作者未给出“境界”概念界定,我几乎一直是按“意境”来理解“境界”的。附录中叶嘉莹的论文《〈人间词话〉之基本理论——境界说》认为本书用“境界”一词意在强调词作对感官和心理体验的真切重现。我认可这一解读,不过我以为在意境(或所谓境界)中重要的是其组成部分(景物、事物和精神体验)间的联系和交互,即“整体大于部分之和”。
\par 书中的很多其它概念也未给出明确的界定,如第三则写有我之境、无我之境,我读来就不甚理解。叶嘉莹通过考察叔本华美学对王国维的影响,辨析了有我与无我、造境与写境、主观与客观三组对立概念的差异,解得清晰。有朋友推崇模糊语言“只可意会不可言传”的丰富意蕴,认为这种概念的准确界定“化神奇为腐朽”。我却觉得十分必要,或许是思维方式的差异使然。
\par 此外各章,也有不少精妙之论。第五则说虚构之境“材料必求于自然,构造亦必从自然之法则”,正与我虚构写作“虚实相生”的理念相合。第三十四则批评替代词的使用,“意足则不暇代,语妙则不必代”。第四十四则言东坡词旷,稼轩词豪,“无二人之胸襟而学其词”有如东施效颦。第五十四则论一种文体兴盛期过后难出新意,故有主流文体的转变,我想这一观点对美术史上的流派、思潮之变同样成立。第六十则说“诗人对宇宙人生,须入乎其内,又须出乎其外;入乎其内,故能写之;出乎其外,故能观之。”读来深以为然。
\par 当然最为著名的还是被选入小学语文课本的一节,“古今成大事业、大学问之三种境界”。我想前两境界确是必经的;至于第三境界,可能有一个灵光乍现、豁然开朗的戏剧化的时刻(解决数学难题可能是很恰当的例子);但也可能是日积月累的精进,最后外人所见的成功只是水到渠成的展现。王国维在这则词话中写尽了其中的孤独与坚守、豪情与快意,引用名句却又都是“断章取义”,实在是古典诗词意涵丰富的绝好说明。
\par 王国维作《人间词》,友人“山阴樊志厚”作序大加称赞。谁知据考证这序就出自王国维本人之手,令人大开眼界。
\par \rightline{2022年12月25日}
