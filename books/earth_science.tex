\section{中国地理}

\subsection*{这里是中国}
\addcontentsline{toc}{subsection}{这里是中国}
\par \textbf{Hi, I’m CHINA}
\par \emph{星球研究所、中国青藏高原研究会} 

\par “有一天,要将中国的雪山看遍;有一天,要将中国的江河看遍;有一天,要将中国的城市看遍……”
\par 这是作者在序言中写下的“我有一个梦想”。正如作者所说,中国很大,但可以划分成一个个区域;只需要逐个区域进行解构——从地质构造、地貌,到气候、植被,再到人类活动的历史。这正是地理学的区域科学传统,关注每个区域不同的特性。相比于《中国国家地理》杂志每年的区域专辑,本书的叙述更为概括;但就发展脉络的梳理、图片的视觉震撼、自然与人文的结合,这本书做到了。

\par 我常站在遥感楼的窗前远望博雅塔,听着Barnaby Taylor的《Wild China Theme》,思绪飞向北国的草原、江南的水乡、西南的雪山深谷,胸中便涌起一种对这片土地深沉的爱,想要用自己的脚步和才思去探索她、拥抱她,正如从冬奥会开幕式纪录片中第一次听到童声唱《万疆》时一样。想起毕业欢送会上,几度哽咽的师兄念了一句稼轩词送给大家:“乘风好去,长空万里,直下看山河。”
\par 不同的区域有着不同的自然景观和文化风俗,这种地理异质性给常居一地的人带来对远方的向往。我越发感到,天地的宏阔与广远,不同地方的多样和差异,竟是那样温暖人心。然而如今的都市似乎更多地呈现出千城一面的同质化,让我不禁思考,我们这个时代又能给后人留下怎样的多样性之美?
\par \rightline{2022年2月13日}
\par \rightline{2022年12月16日补记}


