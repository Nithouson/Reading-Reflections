
\section{地理信息科学}

\subsection*{计量地理学:空间数据分析透视}
\addcontentsline{toc}{subsection}{计量地理学:空间数据分析透视}
\par \textbf{Quantitative Geography: Perspectives on Spatial Data Analysis}
\par \emph{A. Stewart Fotheringham, Chris Brunsdon, Martin Charlton / 王远飞等译} 

\par 这本书的三位作者因共同提出地理加权回归而著名,后者是空间分析领域最有影响力的方法之一。相对于一般的计量地理学教材,这本书突出“空间特色”,着重讲述了不少空间数据分析的专门方法(而非通用的统计分析方法),与我个人感兴趣的研究方向也比较契合。中译本归入“大学教材系列”,但从作者前言和全书风格来看,更像一部供同行阅读的综述性著作,表现为不够自成体系,频繁引用其它文献,不少地方也没有给出必要的推导过程;个别地方对方法的讲述也不够清晰透彻。
\par 本书第一章讨论了计量地理学受关注度下降的原因,引用的一种比较偏激的观点令人印象深刻,大意是“你们这些人文地理学者批评定量方法,是因为你们搞不懂数学。”而事实上一些批评也的确忽视了计量地理学在“计量革命”后的发展,因而不再成立。第五、六、七三章涉及空间分析的核心内容,包括局部分析、点模式分析、空间回归,全面反映了上世纪的方法进展,值得一读。特别是其中对空间关系异质性成因的探讨,对我的综合考试报告有所启发。第九章讲述了上世纪空间交互模型的演变,读来脉络清晰,尽管我不赞同结合人类行为的建模是最佳的发展方向,实际发展也似乎是与网络科学、复杂性科学等交融。
\par \rightline{2023年8月16日}

\section{中国地理}

\subsection*{这里是中国}
\addcontentsline{toc}{subsection}{这里是中国}
\par \textbf{Hi, I’m CHINA}
\par \emph{星球研究所、中国青藏高原研究会} 

\par “有一天,要将中国的雪山看遍;有一天,要将中国的江河看遍;有一天,要将中国的城市看遍……”
\par 这是作者在序言中写下的“我有一个梦想”。正如作者所说,中国很大,但可以划分成一个个区域;只需要逐个区域进行解构——从地质构造、地貌,到气候、植被,再到人类活动的历史。这正是地理学的区域科学传统,关注每个区域不同的特性。相比于《中国国家地理》杂志每年的区域专辑,本书的叙述更为概括;但就发展脉络的梳理、图片的视觉震撼、自然与人文的结合,这本书做到了。

\par 我常站在遥感楼的窗前远望博雅塔,听着Barnaby Taylor的《Wild China Theme》,思绪飞向北国的草原、江南的水乡、西南的雪山深谷,胸中便涌起一种对这片土地深沉的爱,想要用自己的脚步和才思去探索她、拥抱她,正如从冬奥会开幕式纪录片中第一次听到童声唱《万疆》时一样。想起毕业欢送会上,几度哽咽的师兄念了一句稼轩词送给大家:“乘风好去,长空万里,直下看山河。”
\par 不同的区域有着不同的自然景观和文化风俗,这种地理异质性给常居一地的人带来对远方的向往。我越发感到,天地的宏阔与广远,不同地方的多样和差异,竟是那样温暖人心。然而如今的都市似乎更多地呈现出千城一面的同质化,让我不禁思考,我们这个时代又能给后人留下怎样的多样性之美?
\par \rightline{2022年2月13日}
\par \rightline{2022年12月16日补记}




