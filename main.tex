\documentclass{book}
\usepackage{amsmath} %宏包
\usepackage{fontspec}
\usepackage{enumerate}
\usepackage{amssymb}
\usepackage{ctex}
\usepackage{graphicx}
\usepackage{titlesec}
\usepackage{indentfirst}
\usepackage{geometry}
\usepackage[perpage]{footmisc}
%\geometry{left=2.0cm,right=2.0cm,top=2.5cm,bottom=2.5cm}
\usepackage{fancyhdr}
\pagestyle{plain}
%\setCJKmainfont[BoldFont=simhei.ttf]{simsun.ttc}

%导言区
\begin{document}
\large

\title{\Huge{\textbf{阅\ 读\ 闲\ 谈}} \\ \vspace{0.5cm} \huge{Reading Reflections}}
\author{\Large{郭浩}}
\date{\Large{\today}}

\maketitle

\Large

\newpage
\large
\renewcommand{\contentsname}{\centerline{目录}}
\tableofcontents  %%生成目录

\mainmatter %%表示文章的正文部分,在生成目录后将从第一页开始
\setlength{\parskip}{0.5em}
\setlength{\parindent}{0em}
\pagenumbering{Roman}
\section*{\centerline{前言}}

初三以来,我的阅读量一路走低,家人调侃我是“购书如山倒,读书如抽丝”。上了大学,每到寒暑假,总要列出几本要在开学前读完的书;然而实际完成度往往不佳,要开学了,也便作罢。有时看到朋友分享读书总结,赞叹于他们阅读的广度和深度;每每发觉自己想读的书很多,进度又太缓,心中不由生出紧迫之感。

2019年生日时,我起草了一份“30岁前阅读核心计划(Core30)”,规定了2027年前应优先读完的100册经典书籍,其中中国文学29册,外国文学30册,特殊类型文学30册,人文社科11册(有的大部头作品不止一册)。从2020年起每年年末发一篇“阅读闲谈”,分享部分书目的阅读体验(当然,实际读的书不限于Core30)。于是我渐渐形成了读完书后随便写点什么的习惯,或寥寥数语,或长达数千字;阅读中的想法得以付诸笔端留下印记,或许我对书的思考也因此深入了些。这本小册子便是由这些“阅读闲谈”分门别类收集而成的。

杭州西溪的麦家理想谷有这样的话:“读书就是回家。”这或许有些夸张,但读好书的确可以使我们体验自己生活之外的生活、与他人进行跨越时间和空间的思想对话。Core30当然不是终点,我希望这本小册子能随着我的阅读体验逐渐充实起来。

本书的附录收录了一些学生时代的课程读书报告。我写阅读闲谈不会回避剧透,这可能影响阅读体验(特别是对于科幻、推理等类别而言),请未阅读过对应书籍的读者注意。

\rightline{2021年1月3日}
\rightline{2022年11月23日修订}

\pagenumbering{arabic}
\newpage
\chapter{哲学}
\section{西方哲学}

\subsection*{裴洞篇}
\addcontentsline{toc}{subsection}{裴洞篇}
\par \emph{柏拉图 / 王太庆译} 
\par 遥想本科时选修的通识课“哲学导论”,课上介绍的中外哲学观点现在大多都不记得了。但我还清楚地记得第一节课教授讲解什么是哲学:“哲学是追求终极智慧的学问,但和宗教不同,它是以逻辑论证的方式探求终极智慧。”加上我本人受数学和自然科学影响颇深,读哲学著作不免会当做数学理论那样严格的逻辑推演体系来读,其结果就是发现所谓“论证”只能说是一种“说明”。当时读《会饮篇》如此,三年后读《裴洞篇》仍是如此。顺便提一句,当时我的课程小论文写的便是批驳《会饮篇》中阿里斯托芬的“同体神话”,以及探讨爱的本义。可惜文稿遗失,现在我还颇想知道那是我是怎么思考爱的。
\par 《裴洞篇》写苏格拉底临刑前和门人探讨死亡,试图论证灵魂不朽。首要的问题自然是“什么是灵魂”。对话中没有专门的讨论,但从书中的意思理解,灵魂与肉体相对,大致是指人的精神性的部分,并且是灵魂让躯体具有了生命。对话中苏格拉底宣扬的灵魂不朽、投胎转世,此生潜心研习哲学、来世与神灵共处,读来颇有教主之风。至于如何论证灵魂不朽,类比推理是主要的手段:“火伴随热而不容纳冷;三伴随奇(指“奇数”这一性质)而不容纳偶;所以灵魂伴随生命而不容纳死亡”。不死的,就是不灭的,所以灵魂不朽。这种类比推理当然不百分百可靠,不然就可以说“奇数不圆满,不容纳完美,所以奇完全数必然不存在”了。
\par 后来我又忆起一个哲学范畴叫“超验”,意思是超出人类认知的范围,“只可信,不可证”。或许我应该换一种读法,即哲学结论并不是要严格地证明对与错,而是将所谓“论证”看成一种说明,说明不必蕴含结论,但可以让结论更可靠:让人觉得足够可靠的观点,也会让人接受。让我这样的唯物主义者、无神论者接受“灵魂不朽”这样的观点自然是困难的,但书中苏格拉底在生命的最后时刻讨论死亡与灵魂,呼吁门人摆脱形体的欲望和快乐,关注人的理性和精神,其本身就具有十足的浪漫色彩。我想一本哲学书引发了读者对终极问题的思考,这书就没有白读。
\par \rightline{2021年8月28日}

\section{中国哲学}



\chapter{心理、行为}

\chapter{历史、考古}

\chapter{社会、人类学}
\section{科学社会学}

\subsection*{有了博士学位还不够:学术生涯指南}
\addcontentsline{toc}{subsection}{有了博士学位还不够:学术生涯指南}
\par \textbf{A PhD Is not Enough: A Guide to Survival in Science}
\par \emph{Peter J. Feibelman / 张婷婷译} 
\par 问:朋友博士毕业,送什么礼物合适?
\par 答:《有了博士学位还不够》。
\par 此书是学界职业生涯的经验之谈,从标题中的“Survival”便可看出作者强调其中的困难不应被我们这样的年轻人低估。作者写到应当了解所做研究的背景和意义,而非只关注技术细节的解决,我思量这一点已经固化在现今期刊论文的篇章结构里了,难怪读上世纪的论文感觉他们写作更自由些。书中不乏有价值的职业经验,却也显出些“成功学”味道和内卷气息。在我看来,职业的成功和人生的成功不能说是正交的,至少也是有夹角的吧。
\par \rightline{2021年12月19日}

\chapter{经济}

\chapter{政治}

\chapter{法学}

\chapter{语言}

\chapter{文学}

\section{中国古代文学}

\subsection*{三国演义}
\addcontentsline{toc}{subsection}{三国演义}
\par \emph{罗贯中} 
\par 我初中时读了《西游记》和《水浒传》;《三国演义》读了一半左右就上了高中,便搁置下来。六年多过去,这才又拿起来读;由于间隔时间太久,干脆从头读起。22岁读《三国》显得晚了些,甚至说起来有些令人惭愧,但晚读也有晚读的好处:初中时我每天强撑着完成读三回的任务,现在倒觉得其中的权谋计策有些意思,不必规定任务就津津有味的读下去了。初中时只觉得陈琳讨曹操的檄文“畅快淋漓”,如今其中影响最大的精彩片段当然是“诸葛亮骂死王司徒”了,哈哈。
\par 读此书似看戏。在历史舞台上,主公、忠臣、反贼、内奸,你方唱罢我登场。对于没有实权、说被废就被废的君主,处于弱势或战败的诸侯、武将,常常面临“生”与“义”的抉择。我常叹献帝不能掌控命运,被权臣肆意欺凌,还不如效仿高贵乡公曹髦以卵击石,有尊严地死去。至于臣子,有的宁死不降,有的“择良木而栖”,倒都可理解,毕竟内部矛盾不涉及民族大义。最惨的是“生”“义”二者皆失,如荆州刘琮,拱手投降,被诛于赴任路上。而在残酷的政治斗争中,一些小人物难以左右大局,却一样做了牺牲品。比如那些送信的使节就令人同情:有时还不知道信里写的是啥,就被对方“怒斩来使”。
\par “均势”是我的另一点体会。魏国最强,因而孙刘联盟,魏伐蜀则吴伐魏,魏伐吴则蜀伐魏,这才得以成鼎足之势。这种“均势”在现当代国际政治中也屡见不鲜。在一国之内,君臣之间也存在着一种“均势”:曹操、刘备、孙权能统领手下文臣武将,而他们的后继者有的没有那样的雄才,以致手下功高盖主,大权在握。掌握重权的臣子,若没有诸葛亮那样的尽忠之心,行废立、谋篡位便是常见的事了。
\par 小说作为封建时代的产物,竭力渲染“天命”“神鬼”:大将身死之前,必有“将星坠于野”;孔明、关羽死后显灵;甚至出现了管辂教人求神仙改寿数的情节。要是真有一个数据库保存着每个人的寿数,那岂不是会有大量程序员毕生致力于入侵那个数据库?
\par \rightline{2020年7月6日}

\subsection*{红楼梦}
\addcontentsline{toc}{subsection}{红楼梦}
\par \emph{曹雪芹 / 无名氏续\  程伟元、高鹗整理} 
\par 《红楼梦》被誉为中国古典小说的高峰,因其复杂性、残缺性引发了后人极为丰富的讨论和探索。红学家如俞平伯、周汝昌、蔡义江、梁归智,文人学者如张爱玲、王蒙、叶朗、王博,通俗讲者如刘心武、蒋勋、欧丽娟,还有众多关心和参与其讨论的读者和红学爱好者……一部小说受到如此广泛和深入的关注,我想在世界文学史上都是绝无仅有的。而进入这些讨论的钥匙,正是《红楼梦》原著。
\par 《红楼梦》的人物塑造、语言功力之高,也是我读原著之前就知道的。在读原著之前,我就知道“金陵十二钗”,特别是林黛玉的形象已成为日常用典;高中语文老师便让我们背“白玉为床金做马”;初中时候一位女同学谈到作文文笔优美的秘诀,“你们去读《红楼梦》吧,那实在是好词好句的宝库。”我读到宝玉的酒令“滴不尽的相思血泪抛红豆”,想起这正是她作文里曾整段引用的话。
\par 去年夏天去北京植物园,走到曹雪芹纪念馆,我道:“不会有人还没看过《红楼梦》吧?不会吧不会吧?哦,原来这人是我自己。” 终于下决心读四大名著中剩下的最后一部,是在今年春天;到11月20日读完,期间虽也看些别的书,但周末大段的读书时光,都在这里面了。
\par 读书之初最深的感受是叙述之细致:第七回写周瑞家的给姑娘们送宫花,作者花不少笔墨依次写来,区区小事花了近半回笔墨。有人说作者正是从细处着手刻画人物性格;而在我看来,作者能把荣宁府、大观园里每个人的一言一行写得如此细致,这一切在读者看来也变得真实可感了。读着读着,我似乎觉得自己在见证少年宝玉的生活;提到黛玉、湘云、宝琴,竟仿佛现实中朋友的名字一般。我印象很深是一个情节是第二十三回林黛玉听《牡丹亭》,闻“如花美眷,似水流年”而醉心落泪;时隔数百年,这里面也有我们青春的影子。
\par 我曾和父亲交流读《红楼梦》的感受,殊不知我读来最厌烦的对服饰器物的铺叙正是他最感兴趣的;而他一概跳过的诗词曲,却是我最为钟爱、反复咀嚼的。我到书店里翻欧丽娟的书,她提到对雅文化的赞美是作品的主题之一,对此我是赞同的。书中多次写到宝玉和姐妹们结社,赋诗联句,我读来作者本意并不是所谓“封建统治阶级文学的空虚”,的确是展现了作诗的技艺、巧思和乐趣;宝玉题大观园匾额对联、香菱向黛玉学诗、宝玉作{\CJKfontspec{simsun.ttc}姽婳}词几章也都十分精彩。穿插在公子闺秀们的风雅游戏之间的,却是现实世界的种种不堪:从家塾学生的打斗,到家族中的勾心斗角、聚赌通奸……再看作诗联句,芦雪广即景联诗何其热闹,到了凹晶馆则唯有黛玉湘云二人,这不正是贾家由盛转衰的象征吗?
\par 要说诗歌艺术,我最喜欢的还是第五回宝玉梦游太虚幻境,金陵十二钗的判词和红楼梦十二支曲,其中暗暗伏下整个故事的走向和每个主要角色的结局。伏笔照应也是小说艺术中很吸引我的一种,因而红学诸多流派,我对探佚学最有兴趣。虽说续书作者总引用前八十回的情节来表明自己承接前八十回,但看了第五回的“剧透”,就知道续书的故事走向完全错了。续书中虽写到元妃薨逝、王子腾病亡,贾家被抄,但最后结局居然是“沐皇恩贾家延世泽”,贾赦、贾珍被赦,贾政官复原职,死刑犯薛蟠获释,香菱还成了正室;虽说宝玉离家出走,宝钗怀孕留下子嗣,贾兰也得中举人。好一个大团圆结局!可是这恰恰是最不应该大团圆的小说。读到前八十回末尾,我已感到明显的衰败走向;按照由盛而衰的总基调,有理由相信故事会一路下行,走向一个个悲惨的结局:富贵温柔,只是尘世一梦。我甚至觉得按照前八十回下行的惯性,可能全书一百回就结束了,达不到刘心武的108回,张之的110回,更没有程高本的120回。传说曹雪芹写作《红楼梦》泪尽而逝,想来竟是有可能的——读者看来红楼闺秀们尚且真切鲜活,作者更会觉得她们如同活在自己的生活中一样;那么每想一次她们的结局,便为之心痛一次。作家写作一个故事,其中情节不知要在脑中“排演”多少次。由此想来,如何不令人悲恸!
\par 宝钗是封建礼教的代言者,加之对金钏儿之死的冷漠(高中时读到周先慎《琐碎中有无限烟波》的分析),我并无好感。黛玉伶牙俐齿,才华值得欣赏。人民文学出版社出了一套黛玉怼人语录书签,我买了其中两款:“咱们只管乐咱们的”、“我就不能一目十行吗”。我觉得黛玉教香菱学诗说的话更有意思:“什么难事,也值得去学!我虽不通,大略还教得起你。”相比之下我更喜欢湘云、宝琴,除才华外,还有些活泼率真的性格。
\par \rightline{2022年11月26日}

\section{中国现当代文学}

\subsection*{野草}
\addcontentsline{toc}{subsection}{野草}
\par \emph{鲁迅} 
\par 鲁迅大概是中小学课本选入篇目最多的作家之一了。当时学《从百草园到三味书屋》《社戏》《雪》《藤野先生》,以至后来的《孔乙己》《祝福》,并未觉有何格外伟大之处。直到大学国文读《铸剑》里的“哈哈爱兮歌”,始觉这奇文远非常人所能为;又读令人毛骨悚然的《墓碣文》:“待我成尘时,你将见我的微笑”,只得叹服——终究是第一流的文字。
\par 这学期旁听现代文学史课程,讲鲁迅时,教授从《野草》出发解读鲁迅的人生哲学和思想,讲演对书中多数篇目都有涉及。《野草》的作品大多篇幅短小,一些篇目我随堂即上网查找读完,于是干脆找来全书通读。不少篇目都有浓厚的象征意味,而意象的格调总体是阴暗、残酷的(甚至不乏血腥、恐怖);语言则不乏深刻、有力的句子,如饱含激情的演讲。当然也有篇目如《立论》《我的失恋》,让我了解到鲁迅创作的多样性,读到时忍不住当“笑话”转发给朋友。
\par 除之前读过的《墓碣文》,读来觉得格外欣赏的篇目还有《影的告别》《过客》《这样的战士》《淡淡的血痕中》。《影的告别》关注了“影子”这一意象,“黑暗会吞并我,光明又会是我消失”,很是巧妙。《这样的战士》则具有讽刺意味和震撼人心的力量。《过客》采用剧本体裁,当老人谈及前方的坟地,小孩子关注的则是野百合、野蔷薇,让我很有感触。《淡淡的血痕中》放到现在依然很有警醒意义:网络环境下的公众会震惊、愤慨,但似乎是健忘的;他们会追逐新的热点,而真的猛士“记得一切深广和久远的苦痛”。
\par \rightline{2021年8月12日}

\subsection*{围城}
\addcontentsline{toc}{subsection}{围城}
\par \emph{钱钟书} 
\par 《围城》称得上是一部奇书。书中故事是知识阶层青年最为平常的工作、恋爱、家庭、社交经历,也并无什么新奇的想象和夸张,却写得极为细致和真实(特别是人物的心理活动)。此书的另一特点是其连串的俏皮话,不时令人捧腹。比如李梅亭在三闾大学讲授“先秦小说史”,此课可谓内容充实。
\par 初读此书,认为不过是一部幽默小说,作者借以讽刺种种想讽刺的人和事,包括新诗人、旧体诗人、炫才者、贪图小利者、爱慕虚荣者…… 不想越读越觉得沉重,到最后一章达到顶峰:三闾大学中的勾心斗角尚可平心而观,一对知识阶层的青年夫妇,相遇尚存些许美好,终于在两个大家庭的夹缝中失和。这不由得引发我对人性的思考。
\par 大学的一位先生曾在课上引述书中的话:“你是个好人,但没用。”(赵辛楣评价方鸿渐的原话是“你不讨厌,可是全无用处。”)以此告诫我们。当时我不以为然,正所谓“无用为大用”。后来渐渐觉得,这“用处”或许是指人在社会网络中的所谓“势力”,即财力、社会关系以及个人才能的总和。没有谁要求一个人“有用”,但“有用”可以少些依附、嘲讽,生活得舒服些;或者需要自己和亲人少些攀比之心、虚荣之心,“安贫乐道”。
\par 我以为方鸿渐最严重的弱点在于没主意。游学漫无目的,职业只等亲友相助,追求唐晓芙或许是唯一的例外。
\par \rightline{2021年8月18日}

\subsection*{南行记}
\addcontentsline{toc}{subsection}{南行记}
\par \emph{艾芜} 
\par 2021年底读了这本不足十万字的短篇小说集,算是最后冲了一下KPI。八篇小说以上世纪二十年代滇缅一带的漂泊生活为背景,都用第一人称而少有外部视角的评述,读来如记述亲身经历的非虚构作品。得知此书缘于一本短篇小说选集中的《山峡中》,现在看仍是集子中艺术成就最高的一篇,奇险的自然环境和特殊的一群人构成的场域,令人印象深刻。《洋官与鸡》描绘洋官搜刮民脂民膏的丑恶嘴脸,也十分生动。主人公虽是漂泊于社会底层的流浪者,却表现出可贵的正义感(《我诅咒你那么一笑》)、反思性与革命性(《松岭上》)、坚韧精神(《人生哲学的一课》),传递着昂扬向上的精神力量。
\par \rightline{2022年1月1日}

\subsection*{俗世奇人(二、三)}
\addcontentsline{toc}{subsection}{俗世奇人(二、三)}
\par \emph{冯骥才} 
\par 中学时读冯骥才的小说集《俗世奇人》,缘于课本所选《刷子李》等篇,书中也的确有不少精彩篇目。近年来冯骥才连推续作,却未能摆脱“一部不如一部”的魔咒,让人读完印象深刻的篇目减少了。个人觉得写市井奇人绝技比较出色的有第二册的《四十八样》,以及第三册的《白四爷说小说》。此外第二册《张果老》讲了一个精心设计的骗局,揭示了收藏者追求收藏品完整(集齐一套)的弱点,很是有趣。第三册《十三不靠》中的汪无奇,则称得上是一个让人拍案叫绝的人物。
\par \rightline{2020年12月8日}


\section{外国文学:中世纪及以前}

\section{外国文学:文艺复兴至十八世纪}

\subsection*{哈姆雷特}
\addcontentsline{toc}{subsection}{哈姆雷特}
\par \textbf{Hamlet}
\par \emph{William Shakespeare/朱生豪译}
\par 把一个涉世未深的青年推向政治斗争的漩涡,这命运是颇为残酷的;更何况丹麦王子哈姆雷特所要面对的是心狠手辣的僭主和不得不报的深仇。能勇敢地面对残酷的命运和黑暗的现实,已经十分不易;而斗争本身又是难以预知的棋局,稍有不慎便会身死人手,亦无法指望胜利者书写历史的公正。倘若真的遇到了这样的命运,也只有动用全部的胆识和智谋,向着漩涡而去了。在吕氏的威权下韬光养晦的汉文帝刘恒,继位时只有23岁;古罗马的屋大维登上政治舞台,和安东尼、雷必达争夺权力,也不过二十出头。命运会成就英雄,失败者也留下了为后人感叹的故事。哈姆雷特假充疯癫掩饰内心的愤怒,又在被押往英国的路上勇敢逃脱,都值得称赞;演戏影射之举虽有打草惊蛇之嫌,却也借此看清了国王的真面目。最后的同归于尽结局,更是可悲可叹。
\par 宗教对来世的描述影响着剧中人对生死的看法,从无神论者视角看来,国王祈祷时动手再好不过了。
\par \rightline{2022年2月16日}

\subsection*{罗密欧与朱丽叶}
\addcontentsline{toc}{subsection}{罗密欧与朱丽叶}
\par \textbf{Romeo and Juliet}
\par \emph{William Shakespeare / 朱生豪译} 
\par 这是我读的第一部莎士比亚剧作。2017年11月我曾在百讲看过TNT剧院演这部剧。记得出场后,我便问看过原著的同伴:那种可以装死四十二小时的药,原著真是这样写的吗?
\par 看书的时候,我笑着跟老爸讲,Capulet虽为贵族家长,还是会在宴会上说出那样的粗鄙之语。Romeo和Juliet的对白无疑是戏剧化的,若恋人真的要这样聊天,那不得累死。不过最深的感触和看演出时是一致的:那个时候的贵族衣食无忧,又不必有自己的志业。他们谈论爱情时,很轻易地谈到生死,并说出“爱情价更高”。相比之下,现代人的生活就丰富得多了。
\par \rightline{2020年2月23日}


\section{外国文学:十九世纪}

\subsection*{欧也妮·葛朗台}
\addcontentsline{toc}{subsection}{欧也妮·葛朗台}
\par \textbf{Eugénie Grandet}
\par \emph{Honoré de Balzac / 李恒基译} 
\par 小说的故事情节并无特别的新意——对金钱和地位的狂热追逐,穿插在“痴情女子负心汉”的感情线中。类似题材的作品中,男女主角地位交换的情况在文学作品中也有出现,如皮兰德娄的《西西里柠檬》,可与之对读。
\par 值得一提的是,作者常常中断情节的叙述,置身事外对故事中人物做一番评论和分析,而这些评论多有引人共鸣之处。作者不仅构造故事,还探讨社会的现象、人的精神的现象,并试图分析其原因,从中可以看出其风俗(社会现状)——哲理(原因)——分析(原则)三步框架的影子。我想这也是本书得以成为名著的重要原因。
\par 小说中的葛朗台老爹被刻画为一个受金钱异化到无以复加地步的典型形象。人类文明发展进步了近两百年,面对金钱的诱惑,我们的表现却似乎并没有多少长进。他们说看透了这世界,其实常常只是看到了表象之上的谎言:因为金钱是手段而非目的,并且很多东西如果一定要标个价,那会是阿列夫零。
\par \rightline{2021年4月24日}

\section{外国文学:二十世纪至今}

\subsection*{二十世纪外国短篇小说精选}
\addcontentsline{toc}{subsection}{二十世纪外国短篇小说精选}
\par \emph{王向远选编}

\par 这是人民文学版语文必读书目中的一本,一直以来被我当作火车上打发时间的读物,断断续续看了数年之久。
\par 本书所选篇目中,我最欣赏的是马赛尔·埃梅的《穿墙记》,其构思之精彩、叙述之生动堪称典范,作为穿墙者的小人物煊赫一时,被困墙中的形象则留下悲凉的余味。其次是泰戈尔的《喀布尔人》,其语言亲切温暖,是一曲跨越种族的人性之美的高歌。
\par 一篇小说如果没有奇妙的构思、细致的刻画,其主题再深刻,也只是一个空架子。卡夫卡的《判决》对心理的刻画细致入微,卡尔维诺的《阿根廷蚂蚁》中对各种灭蚁装置的设想更是让我惊叹,体现了虚构场景下讲出逼真故事的高超能力。我想即使不知道作者,读这两篇文本也能看出大家的功力。
\par 亨利·巴比塞的《十字勋章》和萧伯纳的《皇帝与小姑娘》同为反战题材,前者基调沉重,后者则以幽默、讽刺的笔触叙事,都具有强烈的感染力。博尔赫斯的《小径分岔的花园》文字带些意识流色彩,将时间分岔(平行宇宙)的设想嵌入一个构思奇巧的谍战故事中。此外库尔特·冯内古特的《今天我演什么角色》,杜鲁门·卡波特的《圣诞节忆旧》,川端康成的《伊豆的舞女》,志贺直哉的《清兵卫与葫芦》,都是值得一读的佳作。
\par \rightline{2022年8月25日}

\subsection*{老人与海}
\addcontentsline{toc}{subsection}{老人与海}
\par \textbf{The Old Man and the Sea}
\par \emph{Ernest Hemingway / 孙致礼译} 
\par 这篇小说情节算不上扣人心弦,但能把老人一段出海打鱼的经历写得这么长,实属不易。读到最后人们称道鱼骨架之大时,我才感到一丝悲壮:用“硬汉”的气概与命运抗争的人,纵使功业未成,依然令人肃然起敬。这一点让我想起《布兰诗歌》里的“O Fortuna”。
\par \rightline{2020年2月16日}

\subsection*{诺贝尔的囚徒}
\addcontentsline{toc}{subsection}{诺贝尔的囚徒}
\par \textbf{Cantor’s Dilemma}
\par \emph{Carl Djerassi / 黄群译} 
\par 这是王骏教授在《自然辩证法》课上提到的一本书。作者本人是一位知名的化学教授,退休后致力于用文学作品帮助公众了解“科学江湖”的文化。的确,科研团队的生态较少成为文学作品表达的对象,因此我读此书颇有兴致。事实上,此书涉及的主要是生物学或化学领域的实验室生态,它与我所在领域的情形有一定差异。小说引入了一些与主题关联较弱的人物情感元素,使我读来有些网文的感觉。情节方面,康托和杰里的研究在缺乏重复实验验证的情况下短时间内获得诺贝尔奖,应该说是一个较为牵强的地方。
\par 作者在后记中说:“发表论文、优先权、作者的排序、杂志的选择、大学的终身教职、资助的申请、诺贝尔奖……这些是当代科学的灵魂和包袱。”小说对科学文化中类似议题做了十分细致的刻画,如杰里在领取诺贝尔奖时如何表明自己的贡献。
\par 科学发现的可重复性,是科学界的重要议题之一。前不久国内学界发生了一场围绕G蛋白偶联受体相关研究可重复性的论争,可谓是让人大开眼界。本书中实验的可重复性构成了康托最大的忧虑,也成为克劳斯要挟康托的手段之一。相比之下,我所在的偏数据科学的领域,重复实验就是跑一跑作者公开的代码那样简单,这看来是一件好事。
\par 以R.K.默顿为代表的科学社会学理论是《自然辩证法》课的主要内容之一,其中一个核心观点是“科学家的工作是为了获得同行的承认”,包括奖章、奖金、Fellow头衔在内的各种科学奖励都是同行承认的表现形式。这部小说也在试图宣传这一观点。而在我看来,如果把论文篇数、引用数以及各类科学奖励视为从事科学研究的全部动力,那无疑是另一种异化。
\par \rightline{2021年2月14日}

\subsection*{雪国}
\addcontentsline{toc}{subsection}{雪国}
\par \emph{川端康成/ 叶渭渠译}
\par 这是一部“弱情节”的小说,作者的着力点在于日常场景的叙述和情境的刻画。从开篇车窗玻璃内外风景与人的叠加,到结尾火光之上的银河,作者精心营造的意境,读来的确让人印象深刻。不过从个人角度,这类“散文化”的小说相比于情节清晰紧凑的小说,并不受到我的偏爱。小说写岛村与艺伎的朦胧恋情,从“美的徒劳”写到美的毁灭,这其中是否包含着对人世的感伤?
\par \rightline{2021年12月4日}

\subsection*{情书}
\addcontentsline{toc}{subsection}{情书}
\par \textbf{Love Letter}
\par \emph{岩井俊二 /  穆晓芳译}
\par 今年520那天和女朋友看了重映的同名电影。电影简单而动人的情节打动了我,也给那晚的约会增添了一些淡淡的忧伤。我很快找到了这本书买来看,用了和看电影差不多的时间就读完了。人物对话占了小说的大部分篇幅,使人感觉小说只是在以类似电影剧本的方式叙事;这也使我觉得电影对这个故事的表达更为成功。
\par 在情节构思方面,同名同姓的设定自有其巧妙之处。而在主题方面,这个故事可以说是爱和生死的交织。从中学时代朦胧的感情,到青年时期热烈的爱恋;图书管理员、玻璃工匠、油画学徒……故事的主人公们并非什么成功人士,但正是众多平凡人物的生活中,存在着有关“爱”的动人故事啊。长眠雪山的登山者,让我想到白银百公里越野赛意外逝去的跑者们,令人叹惋。面对高烧四十度的藤井树(女),在家等救护车还是冒着风雪护送病人,又该是何等艰难的抉择啊。在山间小屋的火锅旁,当生者追忆逝者的往事,幸存者为了更多人的生命而坚守,影片的配乐深沉却又澄澈,大概是我最喜欢的片段;一句“你好吗?我很好”,又似乎用爱情为交织的种种作了简单的解答:爱可以跨越有机界与无机界,独立于时间箭头的延伸,乃至“超越永恒”。
\par \rightline{2021年7月11日}

\section{科幻、奇幻文学}

\subsection*{平面国}
\addcontentsline{toc}{subsection}{平面国}
\par \textbf{Flatland: A Romance of Many Dimensions}
\par \emph{Edwin Abbott Abbott /  陈凤洁译} 
\par 终于读完高中时同学推荐的这部“数学幻想小说”。其中的数学内涵现在看来是基本的,高维立方体和高维球已经是大一的议题了。不过小说中提到,从三维空间能看到平面国物体的内部,由此类比,在四维空间中能看到我们日常所见的三维物体的内部,这一点我之前没有想到过。
\par 小说有一段振奋人心的献词,鼓舞人们探求更高维空间的奥秘。我想起一位专长于几何的数院同学告诉我的话:“对于三维空间中的几何学,你可以看《曲线与曲面的微分几何》;但你要是想了解更高维空间发生的事情,就要先学拓扑学作为基础。”因而我读到这一段时,竟感到心潮澎湃。
\par \rightline{2020年2月12日}

\subsection*{哈利波特与凤凰社}
\addcontentsline{toc}{subsection}{哈利波特与凤凰社}
\par \textbf{Harry Potter and the Order of Phoenix}
\par \emph{J.K.Rowling / 马爱农、马爱新译} 
\par 在2017年暑假终于读完《纳尼亚传奇》后,我开始读《哈利波特》系列。比起同龄人,20岁开始读这七本书显得有些晚。小学阅读课上,曾有一名同学借给我《哈利波特与魔法石》看,我看了十几页似乎觉得并没什么意思?而我渐渐发现罗琳构建的这个魔法世界对我们这一代人,实在有着广泛的影响。初入大学,一位同学在自己的创意画上写下“Accio GPA”的宣言;我也曾见到有人对着宿舍的门大喝一声“Alohomora”。高中时一位热心哈迷向大家介绍了Pottermore网站(现在叫Wizarding World),其中有一个“分院测试”项目。我最喜欢的学院是Ravenclaw,但分院测试的结果是Hufflepuff,想来那的确更符合自己的特质。
\par 言归正传。七部小说的故事情节,逐渐从校园生活转变为残酷的战争;《哈利波特与火焰杯》中伏地魔的复活,可谓是一个转折点。在《哈利波特与凤凰社》的神秘事务司之战中,哈利和同学们与食死徒展开真正的对抗,而D.A.是他们得以这样做的基础。我觉得D.A.的创建和活动,实在是这部书的Highlight,正如现实生活中那些能定期开展活动、成员自发参与的学生组织,常常给参与者带来真挚的友谊和难忘的回忆。
\par Fred和George Weasley,通过一个“飞向自由”的情节,把一直以来表现出的性格和志向发挥到了极致。新出场的Luna Lovegood也是我比较喜欢的一个角色,她有着鲜明的不同寻常的特质,令人印象深刻。
\par \rightline{2020年3月13日}

\subsection*{哈利波特与“混血王子”}
\addcontentsline{toc}{subsection}{哈利波特与“混血王子”}
\par \textbf{Harry Potter and the Half-Blood Prince}
\par \emph{J.K.Rowling / 马爱农、马爱新译} 
\par 田先生在他的课上说,《哈利波特》绝不仅仅是一部魔幻小说,它有着深刻的现实意义。快要读到系列尾声的时候,我想起了这句话。与伏地魔的两次战争被称为“第一次巫师战争”和“第二次巫师战争”;目前人类历史上被冠之以“世界大战”的战争,也恰好是两次。伏地魔和食死徒,邓布利多和凤凰社,这两个阵营有什么鲜明的差异呢?伏地魔利用恐怖统治他人,这一点在德拉科身上体现得很突出:他虽声称成功杀死邓布利多将在食死徒中获得无上的荣耀,却同样在天文塔上说出伏地魔以杀死全家相逼。此外,伏地魔及其党羽宣扬纯血统巫师的优越,这在人类历史上也似曾相识;而邓布利多一方不仅对麻瓜出身的巫师没有偏见,对于家养小精灵等其它的生命也能表示出尊重。
\par 这部小说的主线之一是邓布利多向哈利展示伏地魔的过去,并带他走上寻找伏地魔魂器的道路。他没有把这份重任托付给学校里其他资深的巫师,或是法术高强的傲罗,而是选择了哈利和他的朋友们。这一点是耐人寻味的。从寻找挂坠的过程看,有不少环节是哈利一个人难以过关的(邓布利多本人也说“跟我的力量相比,你的力量恐怕可以忽略不计”)。但从哈利在随邓布利多出发前对罗恩、赫敏的交代中,我看出他的从容镇静,这代表着他正走向成熟。“分一点给金妮,替我向她说声再见。”这部小说在校园恋情上着墨不少,而我觉得其动人之处则在于大敌当前的危险、斗争的重任给恋情带上的悲壮意味,这是一般的校园恋情所没有的。
\par 参加天文塔之战的学生,恰恰是一年前在神秘事务司抵抗食死徒的那六人。有的地方管他们叫“D.A.六杰”。为什么其他人没有来并肩作战呢?诚然,在严峻的形势下,明智的人不难明白要学习防御本领用于自卫。而主动加入与恶势力的抗争,则需要真正的勇气。读到小说结尾哈利与金妮的对话时,我想起这样的诗句:“弃身锋刃端,性命安可怀?父母且不顾,何言子与妻。名编壮士籍,不得中顾私。捐躯赴国难,视死忽如归。”
\par \rightline{2020年4月3日}

\subsection*{哈利波特与死亡圣器}
\addcontentsline{toc}{subsection}{哈利波特与死亡圣器}
\par \textbf{Harry Potter and the Deathly Hallows}
\par \emph{J.K.Rowling / 马爱农、马爱新译} 
\par 小说的前六部都写到“校园生活”,有上课、考试、魁地奇比赛这样的插曲;相比之下,最后一部的情节显得紧凑得多。第六部写完,至少有四个魂器需要处理;罗琳还嫌内容不够充实,又加入了“死亡圣器”。应该说“找齐几件东西”是幻想故事中拉长篇幅的常见技法,比如“虹猫蓝兔”的七剑合璧、《福娃》先找如意碎片,再找“精神力量”等等。这种方法容易把本来完整的情节分段化,一旦处理不当,就有长篇小说退化为短篇小说集之嫌。不过在我看来,罗琳的“找齐魂器”还是比较成功的:里德尔的日记给了第二部剧情一个更明白的解释;哈利等人摧毁的四个魂器,其方法各不相同,而且在本书中情节的节奏逐渐加快:寻找和摧毁挂坠盒已经用了一半篇幅,金杯、冠冕和蛇则是在霍格沃兹之战的硝烟中被接连摧毁的。
\par 斯内普的反转可谓是作者为读者设的一个圈套——作为双面间谍,你最后说他真正效忠哪一边都是合理的;特别是斯内普是一个“高超的大脑封闭术师”,邓布利多和伏地魔信任错了人都是可能的。斯内普在亲手杀死邓布利多后,又击伤乔治,读者怎么可能保留着他是正面人物的猜测呢?
\par 通过斯内普的故事,作者似乎在传递这样的认识:我们常常谈论选择、价值观乃至信仰,但行动时常是被情感驱使的。斯内普在自身的成长中并没有选择邓布利多一方的博爱,而是走向了伏地魔的“巫师至上”;对莉莉的爱却使他最终转变了阵营。马尔福夫妇大战中的“一切为了儿子”,不再关心他们的“主人”是不是胜利,同样出于此。现实生活中是否当真如此呢?我只知道,自己也会因为一个人而对一个城市、一门学科心向往之,再说不出它们的坏话。
\par “不要低估爱的力量。”在作者的笔下,哈利在成长中不断体会亲情、友情、师生情以及爱情的。我想,邓布利多和伏地魔的不同,不是简单的正义与邪恶、光明与黑暗。伏地魔不懂得爱,甚至随意屠戮自己的追随者,削弱自己的力量。如果巫师真的谋求统治麻瓜,当魔法遇上巫师瞧不起的先进科技,谁能获胜还是个未知数;但不去这样做的出发点是共情。作者似乎希望告诉孩子们,真正的爱将引导一个人走上正确的道路。
\par 在“第二次巫师战争”中,邓布利多无疑是那个幕后的统帅。事情的发展远远超出了“神机妙算”可以把握的范围,想来很多时候画像里的他也是见机行事吧。至于死去的人如何做到在画像里活着,难道就像模仿人思维的神经网络可以在人死后继续写博客?
\par \rightline{2020年5月15日}

\subsection*{三体:黑暗森林}
\addcontentsline{toc}{subsection}{三体:黑暗森林}
\par \emph{刘慈欣} 
\par 今年暑假开始读这本书之前,不止一位朋友和我谈到《三体》三部曲。我便称赞道:“我只读过第一部,它并不以细致的描写取胜,情节设计却真是巧妙!称得上‘高潮迭起’,从头到尾都很精彩。”朋友便说,快去看看后两部吧——相比于整个系列展示的宏大图景,第一部不过是个引子罢了。读来果然令人欲罢不能。
\par “黑暗森林”无疑是这部小说的核心。当三体入侵的危机袭来,巨大的技术差距之下,直接对抗的手段一一失败;面壁者罗辑在叶文洁的指引下悟出“黑暗森林”法则,才拥有了谈判的筹码。故事情节则是戏剧化的大起大落:当人类经过大低谷的洗礼,技术突飞猛进,两千艘威力强大的星际战舰让人类达到了乐观的顶峰;谁料一个小小的水滴让太阳系成了又一个威海卫,其惨烈情状无可言说。当其他面壁者紧锣密鼓地展开计划之时,罗辑居于北欧的世外桃源,上演一出“梦中情人走向现实”的恋爱戏,触到人内心的柔软。行文中亦不乏新奇的观点和设想,读来令人称妙。
\par 先说“黑暗森林”理论。一部小说的深刻性常来源于所涉及的有关社会、人生以及人类精神的重大问题。这部小说试图处理“宇宙社会学”这一课题,可谓格局宏大;不论是将人类世界的社会学推向宇宙,还是在科幻中引入社会学思想,都具有相当的启发性。至于“黑暗森林”理论本身,网络上有不少从两条公理、两个基本概念(猜疑链和技术爆炸)出发的反驳意见,但这些讨论不完全是纯粹、非功利的;毕竟要使发言引人注意、显示“水平”,大家不懂的东西要说好;大家能懂且说好的东西要说不行。在我看来,这一理论有其合理之处,却不能说可靠。
\par 尽管作者把星舰地球的生存死局和宇宙中的文明之争都作为黑暗森林理论的实例,在我看来它们有着本质的区别,区别在于竞争中的一个单元是否有能力在遭到反击之前彻底消灭对方。在星舰地球的场景中,次声武器可以彻底消灭一艘星际战舰上的全部人员,一旦成功,再不会受到威胁。这种情形类似于几个人被困在封闭环境(如荒岛、洞穴),生存资源有限的场景,可以设想类似于《饥饿游戏》的恐怖事件会发生,只不过这里的竞争单元由战舰换成了个体。彼得·萨伯讨论的法哲学公案《洞穴奇案》情境与之类似,不过几个人没有直接诉诸暴力,而是试图进行协商。而宇宙中的文明情形就不同了。我们是否可以把文明与单一星球或者太阳系这样的系统等同起来呢?答案是否定的,就连面临三体危机的地球都保留了星舰地球这样的分支;具备星际远航能力的三体文明乃至更高文明,建立多星球的太空帝国完全是可能的。如果是这样,我们还能遵循黑暗森林的原理,对探测到的文明直接“消灭之”吗?如果是袭击了一个庞大帝国的珍珠港,进而招致大规模的反击,这样的行动能称得上符合“生存是文明的第一需要”吗?一个类似的情景是后核武器时代地球上的国家之争,就算将来有一天出现资源危机,要攻击一个战略纵深较广的有核国家,恐怕也要掂量掂量可能的反击吧。所以,真正的宇宙文明图景,很可能不像刘慈欣设想得那么简单。
\par 再说“面壁计划”。破壁人从面壁者的行为中推测战略意图,犹如一场格局宏大的解谜游戏,作为一个推理解谜爱好者,读来也是十分过瘾。不过我觉得至少对于泰勒、雷迪亚兹而言,他们的计划于事无益却消耗大量资源,由此看来ETO的破壁也是帮了地球人的忙。
\par 除罗辑外,章北海算得上是另一个成功的面壁者,尽管他保全的“自然选择”号、以及他本人都毁灭于黑暗战役,地球文明的火种的确在他的精心策划下实现了逃亡。抛开他的战略不谈,读到“看不透”的人,显示着坚定信念的目光,似乎觉得有种特别的魅力。
\par \rightline{2021年12月19日}

\subsection*{三体:死神永生}
\addcontentsline{toc}{subsection}{三体:死神永生}
\par \emph{刘慈欣} 
\par 从我翻开这《地球往事》终章的那一天起,晚上回到宿舍继续阅读就成了我每天的期待。短短十二天时间我就读完了全书。在我看来,《黑暗森林》的艺术成就和《三体》相当,我甚至更欣赏《三体》中现实与游戏交织的结构;而《死神永生》以更为宏大的时空跨度、更为惊人的奇妙想象超越了前两部,可以说是我读过的最杰出的科幻小说(当然这个评价没什么分量,因为读过的还不多)。
\par 这部小说分为六部,是一部地球文明的史诗。前半部分承接《黑暗森林》,交代了黑暗森林威慑和“蓝色空间”号的结局;后半部分则是人类防备黑暗森林打击失败,太阳系毁灭的悲剧。不同于一些叙事视角仅限于若干主角或一个群体的小说,作者能出色地驾驭整个地球文明的宏大视角,能对国际社会面对各种事态的反应做出合理可信的设想;对主角心理、行为的刻画,又不失细致入微。
\par 《三体》三部曲各有一个核心的“科学概念”,作者分别在小说中展现了自己对它们的思考和想象。第一部是“三体问题”,第二部是“宇宙社会学”,第三部则是“维度”。在这部小说中,作者继承并发扬了Edwin Abbott Abbott《平面国》的精神:开篇女魔法师摘取大脑的情节,让人联想到对球的无所不知大为惊异的正方形。 对“蓝色空间”号乘员进入四维空间的描写尽管谈不上令人拍案叫绝,却也符合二、三维空间之间关系的类推。小说进一步设想,高维空间可以在低维空间中以碎块的形式存在;低维生物甚至可以进入高维空间并保持生存。这一点初看让人难以接受,三维空间怎么能和四维空间衔接呢?三维生物由三维基本粒子构成,如何能在四维空间的粒子中安然存在呢?细细想来,又无法断然否定:考虑到维度可以卷曲在微观中,谁知道在穿越维度分界的时候究竟会发生什么呢?小说还设想生物可以主动降维,从而可以对原本同一维度的文明进行维度攻击。这又是有数学基础的:平面上[0,1]和[0,1]×[0,1]基数是一样的,也就是降维可以不损失信息。
\par 地球系统的研究中有所谓“盖娅假说”(Gaia Hypothesis),即生命对地球环境有巨大影响,且朝着对自己有利的方向进行。这部小说通过杨冬生前的思考、关一帆与程心关于星际战争的对话,提出了“生命有意识地改变宇宙规律”这一惊人假说,或可称为“宙斯假说”(Zeus Hypothesis),从而为宇宙的不和谐(光速有限、多个维度卷曲在微观,关一帆称为“三与三十万的综合征”)给出了大胆的解释。这当然是未经证实的猜测(我本人倾向其不成立),但作者没有把光速有限且恒定等现代物理的基本图景视为理所当然,让自己的想象力在宇宙规律上驰骋,这份气魄令人赞叹。至于说数学规律被改造,那不会。正如与“魔戒”的交流中,素数序列可以作为智慧生命的证明,正是由于数学规律天然的时空恒定性。
\par 在故事中嵌入诗歌等其它文体,可为小说增添艺术性,这一点做到极致的是《红楼梦》。这部小说在这方面同样值得赞赏,与四维空间中“魔戒”的对话,云天明的三个童话,歌者吟唱的古老歌谣,语言含蓄,令人回味。隐喻(metaphor)一直是文学艺术中最令我着迷的手法,这些穿插的片段都展现了隐喻的迷人魅力。
\par 继第二部的章北海、罗辑之后,又有褚岩、云天明等令人敬佩的杰出人物,为地球文明延续做出了关键性的贡献。而作者笔下的大众,则自私、软弱、短视、愚昧,与之形成鲜明对照。当引力波威慑被摧毁,众多人争当“球奸”,加入效忠于三体的“治安军”;启动引力波广播的“蓝色空间”号最初被视为英雄,三体入侵的阴云散去之后,又成了带来毁灭厄运的魔鬼;面对黑暗森林打击的威胁,大批民众诉诸宗教,当智子透露可能向宇宙发出安全声明之后, 甚至三体文明都成为了朝拜的对象,读来令人唏嘘。教会和神学家花了一个多世纪重新解释经典,使三体人的存在能够符合教义,这是何其辛辣的讽刺!
\par 歌者向太阳系抛出一片小小的二向箔,整个太阳系随之毁灭,成为一幅悲壮的巨画。这是地球文明的终曲,一幕带有浪漫色彩的悲剧;而这浪漫是黑色的,正如茫茫空宇。按照作者的设想,这巨画也难以在宇宙中保留;整个太阳系将成为令当今天文学界困惑不解的暗物质。最后一代人苦心经营掩体工程,可二向箔的维度打击又有谁能预知?
\par 在太阳系灭亡的大背景下,作者重新追问生命的意义。默斯肯岛的老杰森说:“一切都将逝去,只有死神永生。”天文学家威纳尔这样评论被毁灭的三体世界:“没有真正毁掉什么,更没有灭掉什么,物质总量一点不少都还在,……,只是组合方式变了变,像一副扑克牌,仅仅重洗而已……可生命是一手同花顺,一洗什么都没了。”这个比喻很好地解释了生命的精巧和脆弱:其存在本身就是宇宙中的奇迹。更进一步地,歌者发挥了Erwin Schrödinger的思想:在熵增的宇宙中唯有低熵体(生命体)在变得更有序,“这是最高层的意义”;要维持这种意义,生命体就必须存在和延续,这是比出于好奇心探索宇宙更高层的意义。注意到这与《朝闻道》传递的观念截然相反。在我看来,正如生命的存在降低宇宙的无序度,通过探索未知发现世界的秩序,可以降低我们认识中世界的不确定度,这便是学人求索的直接意义;此外研究对象的性质和研究过程本身可以为我们提供审美价值。但我不赞同将这种探索上升到高于存在的意义层面,特别是针对整个文明而言(如《朝闻道》中的星云生物)。
\par 除探索未知之外,还有什么可能置于比存在更高的意义层面呢?孟子说“舍生取义”,这关乎伦理道德,同样是这部小说引人思考的主题。星舰地球的生存困境带来了类似于《洞穴奇案》的道德困境;不管是危机纪元还是后来的掩体纪元,平等的价值观一定程度上成为了人类飞向外太空、广播文明火种的包袱;而黑暗森林打击、维度攻击更是以“生存”的名义完全抛弃了道德。有趣的是,小说结尾重提“责任的阶梯”,宇宙的未来让关一帆、程心做出牺牲重返太空。
\par 在《黑暗森林》的闲谈中,我提出了对黑暗森林理论的质疑。这部小说使我的思考更加清晰。像故事中的三体文明和银河系人类一样,很可能多数文明都不会局限在一个恒星系内,在掌握光速飞船技术的条件下,即使遭到打击也能部分逃脱。这意味着“清理”的作用极为有限,很可能只会暂时削弱一个文明;由于最为重要的科技思想得到保留,文明有能力在很短时间内再度产生威胁。“清理”成立还有一个前提,即来源无法追踪,这消除了反击的可能性。但在我看来,这种追踪能力并非不可能实现;一旦被掌握,“清理”伴随着巨大的潜在风险,宇宙文明将是完全不同的图景。
\par 云天明对程心的单恋故事凄婉感人;他在三体世界得以生存,又精心设计为人类传递情报,堪称可歌可泣。程心自愧于两次为人类做出了错误的选择,我以为她竞选执剑人且未能启动广播是个错误,但制止星环城战争不应苛责。两人赴约于“我们的星星”;艾AA和关一帆亦有情愫,如果一切顺利,想来是美好的结局。只可惜意外使他们的命运发生了交错,两对恋人“交叉互换”。程心“终于看清,使自己这里沙尘飘飞的,是怎样的天风;把自己这片小叶送向远方的,是怎样的大河”。在苍茫的宇宙面前,生存尚且是一种奢望,何况爱情。好在关一帆和程心小世界中的学者生活依然恬静温馨。
\par 程心著述《时间之外的往事》之时依然年轻美丽,可她已见证了地球文明从危机纪元到掩体失败四百年的波澜历史。读完三部曲的读者同样是如此,百感交集之时,程心的一段话令人震撼:
\par “每个文明的历程都是这样:从一个狭小的摇篮世界中觉醒,蹒跚地走出去,飞起来,越飞越快,越飞越远,最后与宇宙的命运融为一体。对于智慧文明来说,它们最后总变得和自己的思想一样大。”
\par 我想,这段话应当能在许许多多和我一样,以及更年轻的读者心中种下一颗种子,我们会一同仰望星空,了解天体物理和宇宙学,随着人类深空探索的前哨(如JWST)看宇宙,思索宇宙中文明的关系、宇宙的前世和未来。
\par “心事浩茫连广宇,于无声处听惊雷。”是以为记。
\par \rightline{2022年11月5日}

\section{侦探、推理、悬疑}

\subsection*{东方快车谋杀案}
\addcontentsline{toc}{subsection}{东方快车谋杀案}
\par \textbf{Murder on the Orient Express}
\par \emph{Agatha Christie / 郑桥译} 
\par 这是我读的第一本阿加莎推理小说,阅读时间在七八个小时。书的前两部平铺直叙,节奏不紧不慢。波洛对不同的乘客采用不同的讯问方式;频频用诈,在不经意间得到需要的信息,展示出高超的技巧。他与布克先生、康斯坦汀医生的对话也不时带着些反讽和冷幽默。推理进程在最后六分之一的篇幅突然加快,直到谜底揭晓,颠覆读者的认知。证词中埋下的伏笔在波洛的分析中一点点揭开;直到最后波洛阐述结论,仍有我没有注意到的问讯细节成为破案的线索,令人拍案叫绝。
\par 这本书之所以能成为阿加莎最有影响力的名作之一,首先在于其颠覆性的设计。阅读证词时,我思考的前提假设是:大部分证词是可靠的(除了一两个可能的凶手)。而作者恰恰推翻了这个前提假设:车厢上的十五位乘客和列车员,除去侦探和死者,都是案件的策划者。他们的行动和证词只是给侦探演了一个难解的故事,以此为基础推理什么时间线、不在场证明,都是推了个寂寞!而谨慎、细致的波洛从护照上的油渍入手,挖掘出了乘客隐藏的身份和作案动机。以此为基础,死者伤口的疑点就得到了解释;再回头看来,印在精装版封面上的麦奎因先生的话“这趟列车可真是座无虚席啊!”竟是整个故事的大伏笔。此外,侦探小说中的故事常常是悲剧的,其揭露的犯罪显示出人性阴暗、邪恶的一面;而这本书的谋杀却是正义的伸张,我想这也是读者愿意看到的。
\par 波洛对金钱的态度值得赞赏,在拒绝雷切特的重金委托时,他说:“我在事业上很走运,所赚的钱完全可以满足我的现实需要和各种任性的想法。我现在只接受感兴趣的案子。”
\par \rightline{2022年11月13日}

\subsection*{尼罗河上的惨案}
\addcontentsline{toc}{subsection}{尼罗河上的惨案}
\par \textbf{Death on the Nile}
\par \emph{Agatha Christie / 张乐敏译} 
\par 读完《东方快车谋杀案》后,我很快又在一个周末分两天读完了《尼罗河上的惨案》。第一天睡前担心睡不好觉去看了剧透,但即使知道凶手,读波洛的推理过程依然十分过瘾。
\par 这部小说的案情虽没有《东方快车谋杀案》那样颠覆性的构思,却也十分巧妙:西蒙和杰奎琳互相制造不在场证明,相当程度上成功转移了大家的怀疑。穿插其中的珠宝盗窃、国际危险分子,也为案件增添了不少复杂性。但波洛说得好:“清除外表的杂质,以便发现真相。”通过对每个人言行、心理细致的体察,首先解释一些无关的疑点,剩下的将指明真相。波洛用考古发掘的例子说明这个道理,我想或许是受了阿加莎考古学家丈夫的影响。杰奎琳的话同样令人回味:“每个人都得追随自己的星星,不管它引导我们走向何处。”这一无奈之语,道出了盲目、失去原则的爱引发的悲剧。她接着说“那是一颗坏星星,那颗星星会掉下来”,分别暗示了西蒙的作案蓄谋与失败结局,也是十分高明的伏笔。
\par \rightline{2022年11月20日}

\section{寓言、神话}

\subsection*{拉封丹寓言}
\addcontentsline{toc}{subsection}{拉封丹寓言}
\par \textbf{Fables}
\par \emph{La Fontaine / 苏迪译} 
\par 书中的寓言故事篇幅短小,大多只有半页纸。很多故事采用拟人手法,饶有趣味。不少生活中听到的故事在本书找到了出处。值得一提的是,我读的本子收录了Percy J. Billinghurst精美的版画插图,绝大多数都可以直接拿去印明信片,单凭这些插图也值回了买书的钱。
\par 简要总结一些令我印象深刻而不那么著名的篇目。《死神与伐木工》一篇写世人相较于死亡宁愿生而受苦,与周国平人生寓言中《落难的王子》有相通之处。《老鼠的议会》讽刺议会讨论积极却无人执行,令人捧腹。《狐狸和山羊》讨论了合作者的背叛,让我想起玩踩气球游戏的经历。(有人说,当自己的盟友突然踩向自己气球的时候,那是怎样的心理创伤啊。反观古罗马史中的政治同盟,还不是要决个你死我活吗。)在《狼和牧羊人》中,一只因残杀羊受到恶名的狼发现“自诩为羊群保护神的人类也在吃烤羊肉”,最后得出结论“狼的唯一错误在于它不是所有生命的主宰”,角度独特。《橡子和南瓜》十分滑稽。《两只鸽子》写远行的鸽子途中遭遇各种危险,使我思量自己周游世界的计划还是限于相对安全的地方为好。《要求有个国王的青蛙》道出了君主制的弊端,即难以保证君主的贤明。为数不多的几篇体现了作者对于一些议题的立场,如《掉进井里的占星家》批评了占星术的虚伪,还有一两篇意在说明“动物也会思考”。《寓言的威力》一篇在全书中有纲领性的意义,表明了创作寓言的动机:“有人说,世界衰老了;而我却认为,人们如同孩子,他们渴望那些浅显易懂、充满快乐的故事。”

\par \rightline{2021年2月6日}

\section{诗歌}

\subsection*{仓央嘉措圣歌集}
\addcontentsline{toc}{subsection}{仓央嘉措圣歌集}
\par \emph{龙冬译} 
\par 一本颇费些周折才淘到的书。对这本书的兴趣多半来自歌曲《在那东山顶上》,最初听的是谭晶《看见》的版本,后来找到玛吉阿米藏音的版本,觉得这两位藏族歌手演绎得更有韵味。译者的翻译以忠实原文为原则,比如第24首“若要附和美人心愿,恐将失去此生佛缘;若是隐居山里修行,又会背离女子芳心。”由此观之,所谓“世间安得双全法,不负如来不负卿”一定程度上是一种再创作。这本诗集并非简单的情歌,有几首作品明显表达了敌对势力对自己的恶毒中伤。作者甚至认为诗中很多情语也只是一种隐喻,对此我对背景了解不深,尚难以判断。
\par \rightline{2020年7月23日}

\section{散文、随笔}

\section{纪实、传记}

\subsection*{假如给我三天光明}
\addcontentsline{toc}{subsection}{假如给我三天光明}
\par \textbf{Three Days to See}
\par \emph{Helen Keller / 林海岑译} 
\par 《假如给我三天光明》是散文名篇,作者告诫读者“像明天将要失明那样去使用你的眼睛”,这样将看到之前从未看见过的东西。作者描述了三天光明时光的计划,观察的对象大致可归为亲友、自然、艺术、社会四个方面。我想对这个问题的回答,几乎就是回答“什么事物值得为之付出时间”,亦即“如何生活”。
\par 作者23岁时写就的自传《我生命的故事》占了全书绝大部分篇幅,介绍了自己从接受启蒙教育直至读大学的经历。印象深刻的一点是“首位成功者”的榜样作用:海伦得知挪威的盲聋女孩朗希尔顿·卡达学会了说话,立即下定学会说话的决心。而第一位以盲聋人身份获文学学士的海伦本人,也成为其他人的榜样。“首位成功者”用实际行动使可能性得到证实,其带来的希望和激励作用是巨大的。此外,海伦的成就固然离不开本人的坚强意志和刻苦努力,也离不开外部条件的支持。海伦的父母能为她聘请专业的盲人教师,海伦的朋友帮助她把需要阅读的书籍和资料制成盲文书,这不是每一个家庭都能做到的。海伦13岁时就在著名发明家贝尔的陪同下参观了世界博览会,这简直令人惊叹。海伦对大学教育节奏过快的评论引人思考,她认为“我们应该把教育视为一次乡间散步,从容不迫,敞开思想,尽情接纳天地万物。”但大学面对着在给定年限内完成人才培养、输送到社会的压力,更不必说被state-of-the-art裹挟着前进的学术界
了。海伦的观点也许只能作为终身学习者的理想。
\par 在《三论乐观》中,海伦阐释了她的乐观主义立场,她似乎认为乐观的态度对事业的成功、社会的进步是必要的,因而“有害的”悲观思想不应被传播。但一个观点是否正确和是否有利是两个问题。成年以来,我逐渐反思这种无充分依据的乐观:既然复杂系统的行为难以预测(特别是影响重大的偶发事件),真的可以说“明天会更好”吗?海伦说要相信人性,“成千上万的人一直在努力维持这世界的美好”。而我似乎觉得需要一套让个体利益与群体利益趋于一致的社会制度;个人生活得到改善,当且仅当他为维护社会美好、推动社会进步做出贡献。似乎有时候,绝大多数人“努力维持世界的美好”都是不够的。
\par \rightline{2023年1月1日}

\section{儿童文学}

\subsection*{小公主}
\addcontentsline{toc}{subsection}{小公主}
\par \textbf{A Little Princess}
\par \emph{Frances H. Burnett / 梅静译} 
\par 伯内特夫人的书,少年时曾读过《秘密花园》,很是喜爱。《小公主》接触过书虫系列的改写本,如今找到全译本来读,仍然觉得是个温暖感人的故事。女主人公萨拉是个典范的儿童形象:善良、礼貌、酷爱读书、多才多艺、富于想象;小说着力刻画的则是她不论处于富贵还是贫贱,都努力像公主一样行事,保持着一份美德和尊严。或许是读书使然(十一岁的女孩居然对法国大革命感兴趣),她有着与年龄不相称的成熟。她对学校的洗碗女仆说:“我和你一样都是小女孩,而我不是你,你不是我,只是个意外而已。”不久父亲破产、去世,她从饱受宠爱的富家女变成了一文不名的女佣,就住在洗碗女仆隔壁。势利的女校长对她极尽剥削、虐待,却没有想到一时失势的人还有东山再起的一天。阁楼宴会一幕一波三折,从现实破灭到美梦成真,亦是整个情节的缩影,具有温暖的浪漫气息。
\par \rightline{2022年12月14日}

\subsection*{窗边的小豆豆}
\addcontentsline{toc}{subsection}{窗边的小豆豆}
\par \emph{黑柳彻子 / 赵玉皎译} 
\par 好久没读儿童文学作品了,果然感觉阅读压力小了很多,可以一口气读几十页(大概是最近读论文的缘故)。这本书的版权页将其归类为“儿童文学——长篇小说”,但既然写的是真人真事,归入纪实散文或许更合适。
\par 书中的内容并不限于巴学园,但小林宗作校长的教育理念和方法应当说是全书的主线之一。我相信日常的自由活动时间、亲近自然或是接触生产实践的集体出游对小学生有益处,但对小学生能否主要通过自学奠定知识的基础抱有怀疑。巴学园的小班教学实验当然很难适应竞争激烈的当下社会,但如果高中三年完全以大学为目的,初中三年完全以高中为目的,小学六年完全以初中为目的,我想那也不是最理想的情况。
\par \rightline{2020年7月23日}

\subsection*{天蓝色的彼岸}
\addcontentsline{toc}{subsection}{天蓝色的彼岸}
\par \textbf{The Great Blue Yonder}
\par \emph{Alex Shearer / 吕良忠译} 

\par 这是一部试图阐释“死亡”这一主题的儿童文学作品,国庆假期期间约四小时读完。因车祸意外身亡的小学生哈里以幽灵的形式回到人间,先是因看到没有自己的世界如常前行而沮丧,之后从师友的纪念、亲人的悲伤中得到感动。哈里重见家人的一段,读来感人至深,让人更懂得珍视生活中的点滴,特别是和亲人朋友共度的美好时光。
\par 对于人死后的路径,书中的设定似乎是先到一个叫“他乡”的地方,之后自愿前往“天蓝色的彼岸”,代表个体意识的终结,有机体解体、回归自然。在他乡滞留的时间可以任意长,甚至能以幽灵身份重返人间(尽管不合规);这样做多是有未竟的心愿,如寻找失散的亲人。若真如此,没有挂念也何妨在人间再游荡个千百年。可赏美景,却不可品美食;可观棋,却不能下棋;能思考,却不能著述。想来难受;却也凑合。
\par \rightline{2022年10月3日}


\section{文学研究、文学史}

\subsection*{人间词话}
\addcontentsline{toc}{subsection}{人间词话}
\par \emph{王国维/ 徐调孚校注} 

\par 此书少数章节是对词的一般讨论,其余则似读词的批注合集,对一人、一篇乃至一句而发。可以看出作者读词十分广泛,除作者认可的名家李煜、冯延巳、欧阳修、苏轼、秦观、辛弃疾,书中还论及不少我看到字、号一头雾水,查出名字仍十分陌生的词人。我至今对词缺乏系统的阅读,如今读此书也只能观其大略。
\par 还记得《红楼梦》中林黛玉论诗词,说立意最为重要,词句新奇次之,格律相比之下并不要紧。 “质重于文”的观点我是赞同的,但这些讨论并未体现出词的美学内蕴。本书开篇即说“词以境界为最上”,但作者未给出“境界”概念界定,我几乎一直是按“意境”来理解“境界”的。附录中叶嘉莹的论文《〈人间词话〉之基本理论——境界说》认为本书用“境界”一词意在强调词作对感官和心理体验的真切重现。我认可这一解读,不过我以为在意境(或所谓境界)中重要的是其组成部分(景物、事物和精神体验)间的联系和交互,即“整体大于部分之和”。
\par 书中的很多其它概念也未给出明确的界定,如第三则写有我之境、无我之境,我读来就不甚理解。叶嘉莹通过考察叔本华美学对王国维的影响,辨析了有我与无我、造境与写境、主观与客观三组对立概念的差异,解得清晰。有朋友推崇模糊语言“只可意会不可言传”的丰富意蕴,认为这种概念的准确界定“化神奇为腐朽”。我却觉得十分必要,或许是思维方式的差异使然。
\par 此外各章,也有不少精妙之论。第五则说虚构之境“材料必求于自然,构造亦必从自然之法则”,正与我虚构写作“虚实相生”的理念相合。第三十四则批评替代词的使用,“意足则不暇代,语妙则不必代”。第四十四则言东坡词旷,稼轩词豪,“无二人之胸襟而学其词”有如东施效颦。第五十四则论一种文体兴盛期过后难出新意,故有主流文体的转变,我想这一观点对美术史上的流派、思潮之变同样成立。第六十则说“诗人对宇宙人生,须入乎其内,又须出乎其外;入乎其内,故能写之;出乎其外,故能观之。”读来深以为然。
\par 当然最为著名的还是被选入小学语文课本的一节,“古今成大事业、大学问之三种境界”。我想前两境界确是必经的;至于第三境界,可能有一个灵光乍现、豁然开朗的戏剧化的时刻(解决数学难题可能是很恰当的例子);但也可能是日积月累的精进,最后外人所见的成功只是水到渠成的展现。王国维在这则词话中写尽了其中的孤独与坚守、豪情与快意,引用名句却又都是“断章取义”,实在是古典诗词意涵丰富的绝好说明。
\par 王国维作《人间词》,友人“山阴樊志厚”作序大加称赞。谁知据考证这序就出自王国维本人之手,令人大开眼界。
\par \rightline{2022年12月25日}


\chapter{数学}

\section{数学文化与通俗读物}
\subsection*{无穷的画廊——数学家如何思考无穷}
\addcontentsline{toc}{subsection}{无穷的画廊——数学家如何思考无穷}
\par \textbf{Gallery of the Infinite}
\par \emph{Richard Evan Schwartz} 
\par 这本书是在四川省图书馆外文区看到的,已经做好了进原版书的准备,结果双十一欣喜地发现出了中译本。书中用漫画形式阐释了朴素集合论中的一些内容,特别是Cantor开创的无穷集合论。这一理论的结果可称令人惊叹,比如“有理数和整数一样多”、“实数集是比有理数更高的无穷”,正应了《逍遥游》中所言:“汤问棘曰:‘上下四方有极乎?’ 棘曰:‘无极之外,复无极也。’” 我几乎是一口气翻完了全书,不时为书中有趣的讲法而捧腹大笑。此书的绘画也十分精彩,有着浓郁的现代气息。
\par \rightline{2020年11月21日}

\subsection*{数学之旅:数学的抽象与心智的荣耀}
\addcontentsline{toc}{subsection}{数学之旅:数学的抽象与心智的荣耀}
\par \emph{王维克} 

\par 在大学里,大多数学生接受的数学教育止步于“高等数学”“线性代数”“概率统计”。作为其它学科的工具或许差不多够用了,但其实还有不少 “遗珠”,未曾领略则殊为遗憾。这本书试图为非数学专业学生介绍一些更深刻理论的思想,如康托尔对无穷集合基数的讨论、Banach空间与Hilbert空间的定义、混沌与分形等,我认为是一种有益的尝试。当然对于已系统学习过相关理论的读者,这本书信息量不大;书中有些通俗化的解说也略显勉强。
\par 此书开篇介绍说“数学的目的是理解和揭示自然”,这一点我不敢苟同。我以为数学的发展已远远超过了描述我们所在世界的范畴,而是触及其它种种可能世界甚至不可能世界。数学可以在人类思维和创造力的推动下独立发展,不依赖于其理论对于其它学科的应用价值;它既显示人类精神的成就,也是人类精神的食粮。或许这才是副标题 “数学的抽象与心智的荣耀”所蕴含的精神。
\par \rightline{2022年10月15日}

\subsection*{数学 艺术}
\addcontentsline{toc}{subsection}{数学 艺术}
\par \textbf{Math Art: Truth, Beauty, and Equations}
\par \emph{Stephen Ornes / 杨大地译} 

\par 这本书介绍的并非数学与艺术结合的经典议题,如黄金比例、透视法等,而是十几篇介绍当代艺术家作品和创作理念的小品文。此书译笔流畅,附有不少优美的艺术作品插图,适合作为休闲读物。书中介绍到David Bachman尝试用曲线方程描述自然事物;Melinda Green利用分形集绘出佛陀图案。这让我忆起大一刚接触MathStudio的时候,我和舍友热衷于用它绘出各种空间曲面和分形造型。不过Green的想法更进一步,Buddabrot渲染的是复平面上一部分点的迭代轨迹,而非分形集合本身。Robert Bosch的TSP艺术也十分有趣,他把货郎担问题的求解视为一种作画的方式。每篇后还有相关数学背景知识的介绍,但对比较了解数学文化的读者来说,大多数内容已并不新鲜。其中三周期极小曲面是一个我不甚理解而感兴趣的话题。
\par \rightline{2022年6月17日}



\chapter{信息科学}

\chapter{物理}

\chapter{化学}

\chapter{生命科学}

\chapter{地球科学}

\chapter{艺术}

\chapter{其它}

\chapter*{附录}
\addcontentsline{toc}{chapter}{附录}

\setlength{\parindent}{2em}
\section*{苏联解体的原因与启示——《来自上层的革命》读书报告\footnote{本文为作者大一年级《思想道德修养与法律基础》课的读书报告之一。}}
\addcontentsline{toc}{section}{苏联解体的原因与启示}

\par 美国学者David M. Kotz和Fred Weir撰写的《来自上层的革命——苏联体制的终结》回顾了苏联建国至今的历史进程,重点讲述了戈尔巴乔夫如何在民主社会主义改革中葬送了联盟。书中提出的独特观点——苏联内部拥有统治地位的党—国精英选择资本主义,是苏联解体的主要原因——是令人信服的。同时,本书还批判了“社会主义失败论”的观点,指出苏联只是一次建立社会主义的大规模尝试,社会主义之路才刚刚开始。
\par \textbf{1.苏联解体是一次来自上层的革命}
\par 这一观点并不符合世界的主流,但作者首先让我们看到,众多的主流观点其实并不足以自圆其说。
\par (1)社会主义在经济上不可行。事实上,苏联的前60多年经济出现了世界历史上少有的高速发展。按照苏联的官方统计,1928-1975年苏联的年度NMP(物质生产净值,苏联使用的标准)增长率达9.1\%;从西方经济学家估算的GNP来看,这一时期苏联的4.5\%增速也高于美国的3.1\%\footnote{ 大卫·科兹,弗雷德·威尔《来自上层的革命——苏联体制的终结》,中国人民大学出版社2002年版,第47页。}。并且,戈尔巴乔夫改革的前四年,苏联经济仍在增长。
\par (2)来自下层的群众运动瓦解了苏联。1991年3月的全民公决中,76.4\%的民众赞成保留联盟\footnote{同上,第192页};同年5月对俄罗斯属欧地区的民调显示,46\%的民众支持各种形式的社会主义,而主张实行美、德的自由市场资本主义的只占17\%\footnote{同上,第182页}。由此看出,尽管民众对现行的体制有所不满,但他们并不希望苏联解体,走上资本主义道路。
\par (3)外来势力瓦解了苏联。作者认为,如果西方的力量在苏联刚刚成立、国力尚弱时未能瓦解苏联,那么在它强盛时也难以做到。而我认为“生于忧患,死于安乐”,一个国家可以在新生时团结一心,克服种种困难,却在强盛时放松警惕。不过,这样一个外在因素终究不能成为主要因素,内在原因一般占主导地位。
\par 本书的结论是:苏联解体是一次来自上层的革命。
\par 面对长期僵化的体制和经济情况的恶化,戈尔巴乔夫决定通过大规模的变革使苏联走上民主社会主义道路。意外的是,他的改革使苏联的精英阶层中产生了一个亲资本主义的利益集团,并且为他们夺取政权铺平了道路。可以印证这一观点的是,很多原苏联共产党、共青团、国有企业的高层干部在苏联解体后成为新兴资本家\footnote{同上,第157-162页}。
\par 许多党—国精英轻易背叛社会主义,是因为他们本身就不具有坚定的共产主义信仰。他们入党并努力谋求提升,是出于实际的目的。他们享有特权和较高的生活水平,但比起西方同一阶层还是相对较低;并且在集权的体制下,他们不得不谨小慎微、讨好上级以保住自己的职位。
\par 资本主义的思想随着民主化改革到来,精英们的思想也发生巨大转变。私有制可以打破对积累财富的限制,使他们更好地维护自己的利益;市场经济更是自然地受到那些已转变为新兴资本家的精英的拥护。由此,精英阶层长期支持叶利钦领导的亲资本主义联盟,这在这次变革中起到关键作用。

\par \textbf{2.戈尔巴乔夫改革如何推动苏联解体}
\par (1)经济改革。苏联政府先后于1986年和1988年通过《私人劳动法》和《合作法》,允许私人企业和合作企业建立,与国有企业并存。但事实上,许多打着合作企业旗号建立的工厂实质是资本主义企业。后来国家监管日益减少,资本主义得以更加公开地发展\footnote{同上,第152页}。1988年,苏联政府允许私人公司参与对外贸易,众多私人企业通过出口石油、金属等物资赚取了大额利润\footnote{同上,第121页}。1987年,苏联政府给予企业充分的自主权,取消了中央对经济的日常管理,只保留长期的计划。然而,企业自主提高职工工资带来了消费市场混乱,同时企业的投资下降,国家税收减少,这一系列危机引发了人们对改革的质疑,促使了亲资本主义阵营影响的扩大。
\par (2)政治改革。1988年,苏联新选举法通过,由民主的方式选举产生人民代表大会,一些拥护激进改革的亲资本主义分子得以登上政治舞台。之后的选举民主化程度越来越高,1991年,苏联通过全民选举选出各共和国总统,亲资本主义阵营的领导人叶利钦得以当选俄罗斯总统,拥有了对抗中央的阵营。苏联共产党在改革中逐步放弃了在社会日常生活中的领导权,而让政府担当起这些职能。这使戈尔巴乔夫的权力大大减弱,以致改革无法推行,直至失去对局面的控制力。
\par (3)公开性政策。1986年,戈尔巴乔夫决定解除对公开讨论和个人意见表达的限制。他甚至鼓励对苏联党政机关的批评。许多自由派的知识分子成为苏联主流报刊的主办人。这为资本主义思想的传播创造了条件。
\par \textbf{3.戈尔巴乔夫改革与我国改革开放的比较}
\par 我认为,戈尔巴乔夫的改革的确是在向民主的方向努力,但他在许多措施上缺乏考量,在形势受到威胁时依然连连退让,导致了改革的失败。而几乎同一时期我国的改革开放则取得了举世瞩目的成绩。我发现,两者有以下几点不同:
\par (1)我国在引进非公有制经济时坚持了公有制的主体地位,坚持了国有经济控制关键领域。改革后,我国公有资产在社会总资产中依然占优势,能源、电力、通信、金融、交通等行业仍被国有企业掌控,中国石化、工商银行等国有企业至今处在我国企业前列。而苏联后期大规模的非国有化、私有化政策显得缺乏原则。
\par (2)我国坚持了共产党的领导,保持了政治的稳定。“坚持中国共产党领导”被作为四项基本原则之一写入基本路线,我国的私营企业家群体并没有在政治上对社会主义形成冲击的空间。而苏联变一党制为多党制,最终动摇了国家。 
\par (3)我国的经济改革发动了群众的力量。我国在农村推广家庭联产承包责任制,调动了农民投入改革的积极性。而戈尔巴乔夫官僚型的行事作风没能成功动员群众\footnote{同上,第201页。}
\par (4)我国对舆论有所控制。苏联改革中反对社会主义、宣扬资本主义一类的言论,是绝不会在我国当下的主流媒体中看到的。我认为,取消言论自由固然不可取,但由于我们不能保证群众保持清醒的认识、不被反动言论所迷惑,对舆论的必要控制是必须的,否则完全可能民心动摇。
\par 本文只着眼于苏联的最后时期——戈尔巴乔夫改革的教训,而苏联解体的根源,无疑与之前僵化的体制有很大关系。我国同样是社会主义的探索者,苏联各时期的历史是留给我们的宝贵经验。这些经验将帮助我们克服未来社会主义道路上的艰难险阻,继续推动这一人类历史上的光辉事业。 
\par \rightline{2016年10月6日}

\section*{什么是爱——《爱的艺术》读书报告\footnote{本文为作者大一年级《思想道德修养与法律基础》课的读书报告之一。}}
\addcontentsline{toc}{section}{什么是爱}

\par Erich Fromm的心理学著作《爱的艺术》引起了我对爱这一论题的思考与体系重构。在这篇读书报告中,我将梳理阅读此书后我对爱的内涵、来源、特性等方面的思考。本文所指的“爱”包括父母之爱、朋友之爱、夫妻之爱、对事物的爱等,但出于我无神论的观点,不考虑书中所提到的“神爱”。
\par 正如物理学上的时间、空间是难于定义的基本概念,爱在心理上也不容易给出明确的定义。因此,我把爱(Love)的内涵归纳为三点:欣赏(Appreciation)、奉献(Devotion)和期望(Expectation)。
\par 对事物的爱的产生常常是由于积极的审美感受:我爱星空,因为我欣赏它的浩瀚;我爱大海,因为我欣赏它的博大。喜爱蝴蝶的人比喜爱苍蝇的人多得多,也正是因为蝴蝶引起了更多人的欣赏。爱一门学科,也常常是因为感受到了它的美。对人的爱亦是如此,没有人愿意与自己不喜欢的人产生朋友之爱;而爱情可视为友情的特殊形式,它也由于两个人间的欣赏和吸引而得以产生和发展。父母之爱似乎首先出于责任。然而大多数父母还是会关注子女的闪光点,对子女有相对较高的评价,只有极少数父母难以感受到孩子值得欣赏的地方,完全是出于责任来履行父母之爱,比如孩子有先天疾患,只能靠父母照料生存的情况。所以,欣赏是爱的不可缺少的成分,很多时候是欣赏产生了爱,并为爱提供源源不竭的动力。
\par 我们是幸运的。旧社会的青年只能先接受父母的安排,再和自己的伴侣培养欣赏感,而我们可以让欣赏感自然地生长为爱情。正如电影《Flipped》所说:“Some of us get dipped in flat, some in satin, some in gloss. But every once in a while you find someone who's iridescent, and when you do, nothing will ever compare.”
\par 但正如作者在书中指出的那样,物质社会使得很多人仅仅把爱情看作是衡量双方条件后的一种交易\footnote{艾·弗洛姆《爱的艺术》,上海译文出版社2008版,第3页},而这种以物质为基础的爱情显然缺乏持久的动力。所以为了拥有真正的爱情,我们要能够欣赏他人;为了欣赏他人,我们要去了解和理解他人,而这要求我们首先走出自我封闭,敞开自己的心。
\par 爱在过程中和行动上化为无私的奉献,即时间、精力、情感上的投入。正是无私的奉献超越了依据投入与获利比较的一般理性行为,把爱与不爱区分开来。爱星空,就会花时间去观察、记录星空,或是为反对光污染尽一份力;爱一门学科,就会投入大量时间去学习钻研,以至为它贡献自己的成果。朋友有难,慷慨相助;父母对子女更是尽其所能、倾其所有。没有奉献的爱是难以令人信服的。
\par 作者说:“爱情首先是给而不是得\footnote{同上,第20页}。”他认为,我们不应首先谋求得到别人的爱,而是要先不求回报地付出爱,同时相信自己能够唤起他人的爱。而爱从来都不会是单向的给予,而是双向的分享。我们东方有语:“爱人者,人恒爱之。”从博爱的角度,当尊重、体谅和无私的帮助超越血缘和友谊,在更广泛的人与人之间传递和蔓延,这将有利于缓解全社会人与人之间的距离感,提升每个人的幸福感。
\par 爱源于欣赏,发展于奉献,而归于期望。这里的期望当然指积极的期望,不是对当前和过去条件分析得出的结果,而是不随实际情况改变的主观愿望。希望家人平安健康,就是这种期望的体现。父母会纠正孩子的错误,却不一定会指出别人家孩子同样的错误,这是因为父母希望孩子走上正确的道路。如果不爱他,就不必关心他的状况,因而就不必对他提出要求;有愿望,有期求,正是关注和爱的体现。在爱的行动的动力上,欣赏和期望或许是密不可分的。
\par 书中提到的爱的要素有“给”、关心、责任心、尊重和认识。与我的总结相比,“给”与奉献近义;关心是奉献的一种形式,因为关心也要投入时间和精力,同时是期望的必然结果;认识是欣赏的前提;尊重即是放弃控制和强行改变,它可以由欣赏和期望推出:我欣赏你的某种特质,那么我有什么资格要求你改变,放弃这种特质呢?反倒是我要当心不要迷失自我。我期望你有好的生活,那么我必然要尊重你独特的生活方式,因为那更适合你。至于责任,它是一种契约,除了极端的不负责行为会受社会谴责外,履行到什么程度则在很大程度上依赖于自觉,很多爱的契约甚至是可以解除的(比如离婚和朋友绝交),因此它的效力几乎完全依赖于爱的其它因素。总之,两种表述实质上是统一的。
\par 作者深表遗憾地认为,真正的爱情在当今西方社会是罕见的现象\footnote{同上,第77页}。这是因为大多数爱情不具备真正爱情的特性——他特别强调两点,我概括为独立性和统一性。
\par 爱的独立性是指在爱的过程中保持自己人格的独立与发展,而不是失去自我。作者认为,爱是“从自己生命的本质出发,体验到通过与自己的一致与对方结成一体,而不是逃离自我\footnote{同上,第95页}”。既不能把爱的对象偶像化,使自己的精神依附于人,也不能要求对方的人格依附于我。由此我想到许多文艺作品对爱情的描写其实是误导,诸如“我可以为你做任何事”这样的话,显然是缺乏理性思考、迷失自我的。
\par 爱的统一性是指爱不是仅针对一个具体的对象,而是对自我、他人、世界、生活的一种总体态度。“如果我能对一个人说‘我爱你’,我也应该可以说‘我在你身上爱所有的人,爱世界,也爱我自己\footnote{同上,第43页}。’”在这个意义上,爱是一个人内心修养的产物;研习爱的艺术,就是修养人的内心。
\par 最后,用书中我最喜欢的一句话作为本文的结尾:“爱情的存在只有一个证明:那就是双方联系的深度和每个所爱之人身上的活力和生命力。”
\par \rightline{2016年10月30日}

\section*{关于雇佣劳动的思考——《资本论》选篇读书报告\footnote{本文为作者大一年级《思想道德修养与法律基础》课的读书报告之一。}}
\addcontentsline{toc}{section}{关于雇佣劳动的思考}

\par 在两个月前的我看来,上班打工、挣得工资是一件再正常不过的事情:工人有劳动力,却没有生产资料,没有施展劳动力的平台;企业主则为他们提供了这样的平台,这可看成社会发展进程的一种合作形式;企业主占有了产品,但也给工人发放劳动报酬,看起来是公平的;况且工人的工作是自己选择的。而且从解决生计问题的角度看,给一个人工作做,似乎还是一种恩惠。
\par 几堂政治课后,我开始认识到资本主义远不像它看起来那样体面、光鲜;读过几十页《资本论》之后,我知道:资本家的收益恰恰来源于他所雇佣的劳动力;资本家付出的仅仅是劳动力的价值,即他所需的生活资料的价值;他所创造的远大于这一数量的价值,则被贪婪的资本家全部占有。在雇佣劳动的关系中,双方绝不是平等的。
\par \textbf{1.雇佣劳动的不平等性}
\par 马克思指出,单纯的商品流通并不能解释剩余价值的产生。我们知道商品的交换是以等价交换为原则的,即使考虑到价格在价值上下波动,也无法理解资本家获取的巨额利润。那么只可能是资本家所购买的商品的使用价值给他带来利润。这种商品就是劳动力;这种使用价值就是创造价值。
\par 将劳动力视为商品,这一观点让人眼前一亮。这就容易理解劳动力之间的工资差异:资本家向工人支付的工资表现了工人劳动力的价值,它既包括生产劳动力的费用(即工人维持生计所需生活资料的价值),也包括工人的教育、培训费用。所以技术工人的工资高于普通工人,正是因为技术工人接受的教育、培训也要耗费一定的社会资源。
\par 从理论上看,我们为满足生活所需、承担社会平均劳动义务而劳动的时间要小于为雇主实际工作的时间。这之间的差值,本该是属于我们的自由时间!而仁慈的雇主迫使我们相当于是为他无偿地劳动!这难道是公平的吗?
\par 从史实看,这种不平等性几乎是无可辩驳的。在金权至上的资本主义国家,国家权力掌握在资产阶级手中。他们可以运用这种权力轻而易举地维护和扩大自己的利益。
\par 早期的工人没有八小时工作制的保护,拼命地工作也只能换取微薄的工资,勉强度日;大量工人死于职业病和安全事故。他们体会不到自己的尊严,他们的生命只是资本家账本上的一个数字。这一切激起了工人的反抗,包括欧洲三大工人运动、共产国际等。现在看来,受利益驱使过度剥削工人的行为也是不明智的。作为资产阶级,应当维护自己统治的稳定,而这必须要求工人的持续存在。工人都死于剥削,谁来为资本家们劳动?
\par 所以我们现在看到,资本主义自身也在进步。它在不放弃自身核心利益的前提下作出调整,以缓和阶级矛盾。比如现代工人的生产条件、工作时长都有了一定的保证。但雇佣劳动的不平等性并未改变。生产力提高和经济波动引起的裁员,使留任者话语权降低,被迫更努力地工作;贫富差距依然悬殊;更简单的例子是,时有老板拖欠工人工资,却并未听说有工人拖欠劳动。
\par \textbf{2.个体在异化社会的生存}
\par 资本家或许可以这样为自己的行为辩护:我之所以有能力剥削你们工人,是因为我掌握生产资料。社会中向上的路是开放的,你们也可以通过奋斗去当老板啊!
\par 社会中向上的路必然要开放。封建社会都有“学而优则仕”的制度。开放意味着希望,不管成功率多少,有希望才有奋斗的方向。工人阶级中总有不满自己境况的人,有希望时他把不满化作自身动力,为生活打拼;无希望时就只有召集众人一起反抗,把现有制度砸个粉碎,创造出一条“社会中向上的路”。
\par 当今社会,阶层之间的流动是存在的,但并不容易,至少比上层的人维持自己的地位要难得多。有经济条件支撑,上层的人容易受到良好的教育;有社会人脉支撑,上层的人容易在社会中求得发展。掌权阶层只要保证自己的位置还基本坐得稳就够了。
\par 成功走上更高阶层的人,只是对生活境况做出抗争的人中的一小部分;对生活境况做出抗争的人,只是对生活境况不满的人的一小部分。更多的人选择了忍受。而这恰恰是资本家最希望看到的。
\par 忍受的原因当然有很多,比如自身缺乏勇气,小家庭的温情港湾充当麻醉剂。我不知道企业宣扬的企业文化和行业责任感是不是也有这个目的:心里认同了我是在为社会做贡献,我的生命是有价值的,自然就有更强的忍耐力了。
\par 反观资本主义社会的掌权阶层——资本家,他们的物质需求容易满足,似乎可以尽享人生欢乐。与对工人的有形控制不同,这一异化的金权社会对资本家的作用更多是无形的,思想观念的改变。挥霍般的享受,更多地敛财供自己挥霍,成为他们的人生内容,本来要做的事情,被这种空虚的循环取代。他们剥夺了许多人人生的精彩,同时葬送了自己的人生。
\par 总之,少数人的成功改变不了少数人与多数人对立的事实。老板总不能比雇员多。
\par \textbf{3.自由劳动}
\par 我们做一个理想的假设:我们只需要承担社会的必要劳动,剩余时间归自己支配。我们做什么?
\par 休息?娱乐?难道人生的意义和价值就这样体现?
\par 我认为,只要一个人接受了一定的教育,有一定的见识,并且对生活的态度并不消极,他就会给出这样的答案:工作。
\par 教育和见识使他知道世界上的很多领域。积极的态度使他对其中的一些领域产生兴趣。人生短暂,可每个领域都有做不完的工作。所以他会在社会必要劳动之外继续工作,工作在自己感兴趣的领域,只争朝夕。
\par 这种工作不同于雇主强制的要求或是为生计而工作,他的所有动力来源于领域本身和工作者本身,不受异化的影响,我称之为自由劳动。这也正是共产主义社会的美好愿景,它可以使人的才能得到充分的发挥,它的成果将远大于社会必要劳动部分的成果,并推动社会更好地发展。
\par 在马克思的时代,劳动者的工种很有限,不容易想象在工业流水线上工作的工人有很强的内源动力。随着社会的发展,社会上已经出现了接近自由劳动的可能。需要说明的是,自由劳动并不代表没有约束;劳动需要组织,有组织就必然有约束。接近自由劳动是指大部分的实际工作与自身愿望和规划重合。
\par 在现实社会,不得不首先考虑的是生计问题,但这并不难解决。只要一个人自由劳动的领域是有益于社会和人类的,从认可这一点的人处得到报酬就是可行的。而且获得报酬、维持生活中的种种需求只是工作的副产品,因为他的追求就是工作本身。
\par 实现较大程度的自由劳动,是个人生活的追求,也是个人能力的体现。它的杰出成果也将有利于社会的进步和生产力的发展,加快资本主义向社会主义、共产主义过渡的过程。
\par 看看新大陆发现后贩运黑奴的船只,我们就知道什么是资本主义的道德。资本主义必将被终结。那时我们才能看到真正的平等和自由,真正的道德。
\par \rightline{2016年11月7日}

\section*{现实的法学,冷静的法学人——《制度是如何形成的》读书报告\footnote{本文为作者大一年级《思想道德修养与法律基础》课的读书报告之一。}}
\addcontentsline{toc}{section}{现实的法学,冷静的法学人}

\par 我的专业和兴趣领域与法学没有多少交集。听说法学院的同学在学《民法总则》,我到教材中心略一翻阅,复杂的名词、枯燥的概念让我一页也不想读。但苏力教授所著《制度是如何形成的》让我在思考中愉快地读了下去。
\par 阅读此书,我的收获主要有两方面:既有对法学特点的新理解,又有作者的冷静态度与质疑精神。
\par \textbf{现实的法学}
\par 在此之前,我接触过一点《洞穴奇案》的内容。对于极端条件下的一个杀人案件,有的法官主张判处极刑,有的却认为他们无罪,道理却都很有说服力。我一度陷入深深的思考:同样理性的思维却推出迥然不同的结果。如果我是法官,我该如何判决?这样的问题大概归结到法理学或法哲学的范畴,于是我觉得相对抽象的法理学、法哲学的研究很重要、很有意义。
\par 我还畅想过,如果自然的、公正的法律取代了物理规律成为世间万物运行的准则,世界将会怎样?初想感觉这很美好:当杀人犯将刀刺向受害者,受害者安然无恙,杀人犯将会死亡;合法的财产永远都不会被偷走,非法的财产永远都无法取得…
\par 然而且不说物理上如何实现这样的机制,自然首先要确定好一套绝对公正的是非准则,(对于杀人这样的问题还容易界定,更细节的事件如商业侵权、精神损害,自然如何界定?更何况是非准则随时代在改变,我们当然可以说历史上的文字狱是封建皇权的局限,不应作为准则,但我们难道可以说今天的法律没有局限吗?)还要确定好一套适度的处罚准则,(我们显然知道严刑意味着暴政,量刑太轻又不足以威慑潜在的罪犯,那么这二者的临界值又是多少?自然如何做到精准把握?)还必须是一个全知全能的上帝角色,知道每个人在任何时候做的任何事,不能有漏网之鱼。即使这些问题都解决了,还有最关键而又最可笑的问题:当“杀人犯”将刀刺向受害者,不推演下去,你怎么知道他会死呢?如果按物理规律推演下去受害者只会受轻伤,让“杀人犯”去死,岂不有失公正?
\par 那好我们做一点修正,让自然知道每个人的想法,根据举起刀时他脑中“我要杀了他”的想法来判决死刑。问题更多了:有时凶手想的是“我要报复他”、“我要刺伤他”实际会是杀了他;有时凶手想的是“我要杀了他”,却因为对方武艺高强只造成轻伤。这样判决就显得有失公平。再者说,恐怕脑中闪过“我要杀了他”的人数比实际作案并成功的要多好多倍,这样要多处罚多少人,会不会造成恐怖气氛?还有有时“我要杀了他”是一种抒情方式,并不是真要杀人,自然识别语义稍有不当,就会造成冤假错案…
\par 这番思考再告诉我,并不是抽象的法理不重要,但绝不是有一套法理就可以解决社会秩序的调整问题,正如书中所说:“法学…一个重要特点是务实和世俗。”(《反思法学的特点》)面对同样的法律,一般的法庭调解撤诉率为20\%左右,宋鱼水和金桂兰则因调解撤诉率达到70\%以上成为模范法官,可见法律的执行者很重要;《民事诉讼法》等规定诉讼程序的法律的出台,也表明法律发挥作用的程序也很重要。法律在实际运行中涉及的因素是多方面的。
\par 反观书名,制度究竟是如何形成的?不是个别人头脑中的应然规划,而是“社会多种因素制约的产物”。(《认真对待人治》)。比如书中论及美国司法审查制度的确立时,认为它并不是马歇尔法官个人智慧的必然结果,而是各党派政治家斗争和妥协的产物。但是“制度实际发生的作用和意义并不因起源的神圣而增加,也不因起源的卑贱而减少”。(《制度是如何形成的》)
\par 我想,在法学中的确不能引入各种理想化的假设,而是要在承认当前社会所有条件的前提下,构建一个能够比较好地维护社会秩序,实现社会总体效益的系统。幻想的世界中没有法律,法律不需要幻想。
\par \textbf{冷静的法学人}
\par 阅读本书,时常能感受到作者看待事物的冷静、客观立场和批判、质疑的精神,这的确令人佩服。
\par 面对英国王妃戴安娜的死,作者并没有像一些大众媒体一样简单地谴责追逐戴安娜所乘轿车的摄影记者,而是客观地指出,摄影记者并没有造成戴安娜的死亡,至多只是因素之一,关键的因素是司机酒后驾车、超速行驶和死者没有系安全带。有的人为戴安娜的死而悲痛,或痛恨某些记者对名人私隐的打探,但这不是他们不公正地把主要责任推给记者的理由。(《我和你深深嵌在这个世界之中》)
\par 对于宣判之前“罪犯”只能称为“犯罪嫌疑人”之一问题,很久以前准备司法考试的妈妈就给我讲过,我不觉得有什么问题。然而作者却发觉这并不能一概而论,而是要有特定的语境,即“司法机关在判案时不能先入为主地认定被告就是罪犯,而是要用证据来证明被告是否罪犯。”而对于受害者一方,如果确知那位“犯罪嫌疑人”就是伤害自己的人,称之“罪犯”并无过错。(《罪犯、嫌疑人与政治正确》)
\par 在《“法”的故事》一文中,作者对当下几乎所有版本《法理学》教科书中对“法”字起源的解释产生了疑问:它们都采用许慎《说文解字》的解释,即水旁代表“平”,即公平。但象形文字的“水”显示的波纹方向和流动特征表明古人更多关注的是水自上而下的流动。为什么它在“法”中代表的不是法律自上而下颁布施行的特征呢?从这些文字,我看到“凯蒂旺普斯”的精神,而我觉得自己读到类似看来合理,似乎无需置疑的结论,是倾向于接受的。
\par 作者在一篇书评中这样评价波斯纳:“从不迷信前人研究的结论,不相信流行的结论,不相信别人对这些学者思想的概括或其他第二手的资料,总是认真研读第一手的材料。”即使是面对第一手资料,也不能失去批判的眼光。我想,这种严谨认真的态度是没有学科界限的,相对于理论描述相对精确,不易被曲解的自然科学,人文、社会科学尤其需要。
\par \rightline{2016年12月10日}

\section*{从乌托邦到新和谐公社——莫尔与欧文社会主义思想之比较\footnote{本文为作者大二年级《马克思主义基本原理》课的课程作业。}}
\addcontentsline{toc}{section}{从乌托邦到新和谐公社}
\par 16世纪初期,新航路已经开辟,正是大航海时代的兴盛之时;英国杰出的人文主义者托马斯·莫尔所著描绘海外理想国的《乌托邦》正符合这一时代背景,给当时人以真假难辨之感,这也成为空想社会主义早期的重要文献。莫尔所处的时期资本主义还处在原始积累阶段,发达的资本主义工业还未产生;而三百多年后,三大空想社会主义者之一的罗伯特·欧文所处的英国已经处于工业革命中,“蒸汽和新的工具机把工场手工业变成了现代的大工业……社会愈来愈迅速地分化为大资本家和无产者”\footnote{恩格斯《社会主义从空想到科学的发展》,《马克思恩格斯选集》第3卷,人民出版社1972年版,412页。}。欧文目睹工人悲惨的生活境遇,在管理苏格兰新拉纳克大棉纺厂时进行了改善工人劳动和生活条件、创办教育等试验,取得了闻名全欧的成效。从一名资产阶级慈善家转向共产主义者后,欧文在印第安纳进行“新和谐公社”的试验,但公社以破产告终。
\par 将莫尔与欧文的社会主义思想进行对比,有助于我们结合时代变迁,理解空想社会主义三百多年间的发展脉络。此外,他们的理想社会设想对于我们未来社会的构建同样有参考价值。下文的对比将从对社会现实的批判和对理想社会的构想两方面展开。

\par \textbf {1.对社会现实的批判}
\par 在莫尔所处的时期,英国在都铎王朝统治之下,在位的亨利八世是一位残暴的君主。莫尔身为伦敦行政司法长官,目睹了社会的种种黑暗和百姓的不幸。在《乌托邦》第一部中,莫尔借拉斐尔·希斯拉德之口,对英国社会进行了隐晦但严厉的批判。
\par “你们的羊,一向是那么驯服,那么容易喂饱,据说现在变得很贪婪、很凶蛮,以至于吃人,并把你们的田地、家园和城市蹂躏成废墟。\footnote{托马斯·莫尔《乌托邦》,商务印书馆1982年版,20页。}” 这段著名的话,就是莫尔用来揭露“圈地运动”这一资本原始积累过程的。在这一运动中,贵族将大片田地变为饲养羊的牧场,许多农民在欺诈、暴力手段之下被剥夺了赖以为生的田产,只好举家流浪。更残酷的是,当时的大贵族、大商人与政府勾结,用残暴的刑罚迫害底层民众,流浪、讨饭者被抓入监狱,大批盗窃者被处以死刑。
\par 除此之外,在拉斐尔对欧洲各国王室的批评中,莫尔还批判了君主对外的尚武好战、侵犯别国利益,对内的聚敛财富、压榨百姓;臣僚的阿谀奉承等。这些议论虽是借其他欧洲王室提出,但都直指英王。莫尔还借拉斐尔之口提出了自己的政治主张:与其侵犯别国,谋求扩张领土,不如用心治理好已有的领地;国王应该为百姓谋福利,而不是搜刮百姓:“国王所统治的不是繁荣幸福的人民,而是一群乞丐,这样的国王还像什么话!\footnote{同上,38页。}”这一反对征战、提倡仁政爱民的主张,与我国孟子的思想有相似之处。但正如书中拉斐尔自己评述的那样,这些观点完全背离当时统治者的主张。
\par 在莫尔的时期,资本主义生产关系所带来的对工人阶级的剥削还不明显,莫尔所批评的更多是一种封建时期的暴政,从中可以看出莫尔的正义感和人文关怀。
\par 而在欧文的时期,资本主义生产关系已经比较发达,资本家对工人的剥削也十分残酷,具体包括童工的雇佣,低劣的劳动和生活条件,过长的劳动时间等。欧文在著作中就曾提及,他的新拉纳克大棉纺厂规定工人每天劳动十个半小时,而同业工厂强迫工人每天劳动十三、十四甚至十五小时\footnote{《人类思想和实践中的革命或将来从无理性到有理性的过渡》,《欧文选集》第2卷,商务印书馆1981年版,102页。}。
\par 然而,欧文并没有在著作中激烈地抨击资产阶级的丑恶,他甚至反对工人阶级用暴力手段反抗资产阶级。其原因是,欧文认为一个人的性格和所作所为是由外力即他人生中所处的环境决定的,他将其视为重要的真理,并在著作中反复强调类似的观点。他说:“(资产阶级)也是由于他们所不能控制的环境而变成了你们(劳动者)的敌人和凶恶的压迫者……他们对这些结果应负的责任并不比你们更大……\footnote{《告劳动阶级书》,《欧文选集》第1卷,商务印书馆1981年版,171页。}”因此,他号召劳动阶级放下对剥削阶级的仇恨和愤怒,以期在不使用暴力的条件下实行新措施,实现所有阶层的共同利益。
\par 欧文期许:“……(较高阶级)只会在为了你们的利益而提出的一切改革中仅仅要求把自己的利益至少保持在现有的水平上\footnote{同上,177页。}。”但事实上,资产阶级中不乏贪得无厌者;在众多工厂主都把工人剥削到极致时,也只有欧文能够为工人着想,改善工人的条件。此外,资本家加大剥削也有行业竞争、追求扩大生产的因素。由此看来,欧文提出的劳动者与资本家就共同利益达成共识、推动改革的主张,虽符合其“泛爱全人类”的原则,确是理想化和不切实际的。
 
\par \textbf{2.对理想社会的构想}
\par 对比莫尔笔下的乌托邦国和欧文建立的新和谐公社可以发现,二者在财产公有、平等、民主等准则上具有一致性。
\par (1)财产公有
\par《乌托邦》在第一部中旗帜鲜明地表明了公有制的理念:“任何地方私有制存在,所有的人凭现金价值衡量所有的事物,那么,一个国家就难以有正义和繁荣。\footnote{托马斯·莫尔《乌托邦》,商务印书馆1982年版,43页。}”第二部中则细致地描述了乌托邦的公有制。在乌托邦,居民没有固定的地产和生产工具,市民轮流到村庄中居住务农,城市居民定期抽签更换房屋。物资实行按需分配,所有产品运到城市中的市场,各户人家到市场根据需要领取物资,分文不付。由于一切货品供应充足,没有必要储存物资,而且乌托邦人也并没有以占有物多于别人为荣的风尚,因此没有人领取超出所需的物品。每三十户居民集中在一个厅馆中共同用餐。城乡之间、不同城市之间物资同样是共享的。
\par《新和谐公社组织法》也规定了财产公有的原则。此文指出:“以个体所有制为基础的制度,必然反对人们权利平等的原则……反对这一原则,就会引起竞争和敌视、嫉妒和纷争、奢侈和贫困、专横和奴役\footnote{《新和谐公社组织法》,《欧文选集》第2卷附录,商务印书馆1981年版,187页。}。”由此,莫尔和欧文都视私有制为万恶之源,主张实行公有制以维护广大群众的利益、实现公平正义。
\par (2)平等
\par 乌托邦国的平等首先表现在人人都必须劳动,只有极少数高级官员和专门从事学术研究的人除外。劳动以务农为本,居民兼干自己的一两门手艺。乌托邦人每天只劳动六小时,却能生产出足够本国使用、还可出口海外的产品,作者认为人人从事生产劳动便是主要原因,相比之下其它国家由于妇女、僧侣、绅士、贵族及仆从不从事劳动,还有不少人从事“不实用的、多余的行业”。除此之外,乌托邦人衣着统一朴素,房屋轮换等处也体现了平等理念。
\par 《新和谐公社组织法》同样把“所有的成年人,不分性别和地位,权力一律平等”和“随着智力和体力的适应程度而变化的义务一律平等”列为前两条原则。在供应允许的范围内,全体社员得到同样的食物、衣服、住宅和教育。
\par (3)民主
\par 在乌托邦,每三十户人选出长官一人(称为“摄护格朗特”),每十名“摄护格朗特”及掌管的各户隶属于一名高级长官(称为“特朗尼菩尔”);所有“摄护格朗特”秘密投票选举全城的总督。除此之外,教士也由国民选出。城市中的重要事务会提交摄护格朗特会议,由摄护格朗特通知管理的各户开会讨论。由此,乌托邦国民有选举、参与公共事务等民主权利。
\par 在“新和谐公社”,全体年满21岁的公民组成全体大会,掌握立法权和书记、总经理、司库、管理员等职的选举权,书记、总经理、司库、管理员组成的执行理事会掌握行政权,每周向全体大会报告工作。通过全体大会,社员的民主权利也得到实现。

\par 以上三点中,民主、平等是资本主义政治制度也具有的概念,但公有制与资本主义制度有鲜明的区别。在财产公有、人人劳动的条件下,平等不再是契约关系下的形式平等,而是实际的平等;民主也不再是资产阶级式的民主,而是社会主义的民主。

\par \textbf{3.若干问题的讨论}
\par (1)公有制下劳动的积极性
\par 我国在社会主义道路的探索中一度因急于向共产主义过渡遭遇挫折,其主要原因是平均主义的分配挫伤了劳动的积极性,“干多干少一个样”,就必然面临劳动积极性如何保证的问题。有观点认为,将来进入共产主义时,劳动成为人的第一需要,人的道德水平普遍提高,也就不存在分配平均化导致的消极怠工问题。但如何使道德水平普遍提高,在劳动风气良好的条件下如何处理少数人的消极怠工问题,依然是需要回答的。
\par 在《乌托邦》中,多数公民并不能自然地乐于从事农业劳动,依然认为农业劳动是艰苦的。但这并不妨碍农业劳动高效有序地进行,原因有每日劳动时间不长,以及农业人员更换的制度使一般人免于长期从事艰苦的工作。但我认为更重要的是深入人心的义务观念,即从事农业生产及自己的专门手艺是自己对社会承担的义务。这在基本上是道德约束,同时也有制度约束。根据规定,居民可以去其它城市或郊区旅行,但要在到访的地方参加农业生产或干本行的活,没有借口逃避工作,这可视为这种约束的一个事例。
\par 制度可以约束公民参加工作,却难以对工作的效率和质量进行精细的管控;以怎样的认真程度和投入程度对待劳动,更多地还是依靠自觉。因此公有制下劳动的实行确需要一定的道德基础。
\par (2)公有制下按需分配的条件
毫无疑问,人们需求能得到完全满足的按需分配需要“物质极大丰富”作为基础。但怎样的丰富才是“极大丰富”呢?如果每个人的需求都无限制地上升到很高的水平,恐怕社会难以满足这样的需要。
\par 在乌托邦国,这种按需分配的体制得以正常运转与国民的观念和生活态度相关。比如,公民没有攀比占有物的习惯;而且,公民性情淳朴,不追求华丽的服装和饰物。的确,如果人人梦想过上奢华的生活,按需分配的制度是难以满足所有人需要的。因而按需分配的实行也需要倡导一种拒绝浮华和矫饰、追求朴素和实用的生活态度。
\par 另一方面,乌托邦的设定依然是农业为主的早期社会,其产品的丰富程度与现在无法相比。汽车、计算机、手机……如此丰富多样的产品,如果按需分配,需要是否也难以满足呢?事实上,相当多的产品可以按照共享的方式使用,如自行车完全可以采用当前共享单车的模式,只是免费使用;计算机如果做不到人手一台,可以为每个社区配备一定数目,供大家需要时使用,这也使得资源得到更充分的利用。
\par (3)莫尔与欧文构想的共同局限性
\par 恩格斯指出:“对所有这些人(空想社会主义者)来说,社会主义是绝对真理、理性和正义的表现,只要把它发现出来,他就能用自己的力量征服世界,因为绝对真理是不依赖于时间、空间和人类的历史发展的。\footnote{恩格斯《社会主义从空想到科学的发展》,《马克思恩格斯选集》第3卷,人民出版社1972年版,416页。}”莫尔和欧文也是如此,他们致力于寻找一个超越时代因素的理想的、完美的制度,但没有为制度建立坚实的现实基础。莫尔详细描述了乌托邦国的方方面面,却没有清晰地交代这一制度建立的过程,把一切归功于开国君主的伟大设计;欧文也没有找到通往共产主义的正确道路,寄希望于开明的统治阶级实行改革,忽视了无产阶级的革命力量。因而他们不乏智慧的设想只能成为空想。
\par 现在我们认识到,任何制度都与一定的时代、一定的生产力基础相联系,脱离现实、凭空设计的“理想制度”终会因缺乏进入这一制度的道路或是制度不与实际相适应而告失败。但另一方面,与当前状况真正适应的制度对于我们同样是未知的,这就需要我们进行持续的制度探索和制度创新,改善制度中不符合实际、不合理的因素。
\par \rightline{2018年5月25日}

\setlength{\parindent}{0em}
\section*{On \emph{Annabel Lee}\footnote{本文为作者大一年级《美国诗歌导读》课的课程作业。}}
\addcontentsline{toc}{section}{On \emph{Annabel Lee}}

\par I have always been interested in stories, for a story has its separate value, which can never be replaced by one or two so called main ideas. In this essay, I will focus on the story of Annabel Lee, a famous narrative poem by Edgar Allan Poe.
\par \textbf{An Abstract Story}
\par A story described a bragging competition. The person who could eat the biggest thing would be the winner. ``I can swallow two space shuttles at a time,'' one said. Another claimed that he could have galaxies for breakfast. However, a monk won at last. He said, ``The thing I eat is the very biggest of all.'' It was true that he actually said nothing, but nor could others!
\par That is just what Poe did. When describing the love between ``I'' and Annabel Lee, he wrote, 
\par \setlength{\parindent}{2em}\emph{But our love it was stronger by far than the love}
\par \emph{Of those who were older than we—}
\par \emph{Of many far wiser than we—}
\par \setlength{\parindent}{0em}He also wrote,
\par \setlength{\parindent}{2em}\emph{With a love that the winged seraphs of Heaven}
\par \emph{Coveted her and me.}
\par \setlength{\parindent}{0em}But what did they exactly do together every day? How did ``I'' and Annabel Lee show their love to each other? How can we know that their love was great? Poe did not offer any concrete description, other than repeating ``it was great.''
\par Poe also used simplification in this poem. For example, he described Annabel Lee as ``lived with no other thought than to love and be loved by me''. In real world, even a child would have thoughts toward family, nature and many other things. How could a human being be that simple? Such exaggeration could make this character too far from reality.
\par On this abstract story, some may think the absence of delicate description leaves space for us to imagine. I would say, what Poe showed us was an empty frame which you could fill  with anything, just like a mathematical formula applied to all certain numbers. It may look moving at first, but like a hanging garden high above us without anything resonating with us, it can hardly touch us deeply and constantly.
\par \textbf{Darkness in Heaven}
\par The cause of Annabel Lee's death was shocking. Angels, a symbol of beauty, kindness and other virtues, envied the love and happiness ``I'' and Annabel Lee enjoyed, and then killed her with a chilling wind! 
\par Other examples can be found that gods or other noble spirits in heaven did immoral or even evil things, which I refer to as ``darkness in heaven''. In Greek mythology, Hera, the wife of Zeus, envied the beauty of Echo, the goddess of forest, and deprived her of speaking ability, who could only repeat others' last few words after that. In another story, a woman called Arachne challenged Athena on weaving and beat the goddess. With anger, Athena changed her into a spider.
\par People worship gods, then why do they make gods do bad things at the same time? Real life involves kindness and vice, honesty and deceit, so people who live in it want heaven to be a reflection of the real world, or they did it without realizing it.
\par Thus, there are both glory and darkness in heaven.
\par Back to this poem, if she had died of hunger or diseases, she might have not taken good care of herself; but she died of ``natural disaster'', so she was no fault at all. If she had been murdered by another people, ``I'' could have tried to catch and punish him; but the murderers are seraphs, so ``I'' could only cry until the eyes were dry! 
\par Poe used this element to strengthen the despair and sorrow of ``I'': it seemed that the whole world, even the angels, were against him. He could feel no justice, no kindness, no warmth, none.
\par It is reasonable that this poem was in memory of Poe's wife, Virginia, as some suggest. The loss of dear wife could throw him into despair, which was a similar situation to that in this poem.
\par \textbf{Conclusion}
\par The poem Annabel Lee is not a vivid story, but its tragedy effect is successful. Regardless of people's comments, Annabel Lee will always be a symbol of eternal love.
\par \rightline{November 12, 2016}

\section*{Escape and Return——Reflection on Frost’s \emph{Birches}\footnote{本文为作者大一年级《美国诗歌导读》课的课程作业。}}
\addcontentsline{toc}{section}{Escape and Return}

\par Frost wrote in Birches,``I’d like to get away from earth awhile, and then come back to it and begin over.'' Obviously, his word is more than talking about swinging trees. Instead, it concerns escape from a hard life situation and return to real life afterward. I refer this process to ``escape and return''.
\par One kind of escape is to transfer your thoughts or action to other real things, usually what you like. If we think of Frost’s swinging birches ``when I’m weary of considerations, and life is too much like a pathless wood'' as a specific reference, it can be one example. When I read this part, it made me think of \emph{My Favorite Things}, a song in the movie \emph{Sound of Music}. 
\par \setlength{\parindent}{2em}\emph{When the dog bites, when the bee stings}
\par \emph{When I’m feeling sad}
\par \emph{I simply remember my favorite things}
\par \emph{And then I don’t feel so bad}
\par “My favorite things” includes snowflakes, ponies, sleigh bells and so on. These things or activities can make us forget our worries and remind us of the beauty and fun in life. The mood is cheered up and the man is refreshed. In hard situations, hold fast to merely several good things in life, and there is no reason for us to be hopeless.
\par Another kind of escape involves imagination. To seek comfort or satisfaction, we tend to use our imagination to beautify things around us. Since the real world is disappointed, why not create a better one in our mind? In Birches, ice-storms bend the trees down, but the poet would think “some boy’s been swinging them”. That is a relatively more beautiful story. 
\par In \emph{Anne of Green Gables}, little Anne Shirley often use imagination to “improve her life”. In her imagination, her plain room becomes magnificent, her poor life becomes exciting. The interesting thing is that we know it is not real yet can’t stop doing this, and it does cheer us up just like real good things. Such as the story of Santa Claus living in the Arctic, we like to be cheated by this fiction.
\par When facing difficulties and almost starting to crumble, some aren’t optimistic enough to use the power of escape, and finally fall into the abyss of depression; some find the escape so wonderful that they don’t come back to real life anymore; only the strong-minded ones can resist the temptation of escape and face their real life again.
\par With imagination, our fantasy can be as good as we want. Then why do we have to come back? Frost’s answer is ``Earth is the right place to love''. Although we sometimes feel the real world is not that real while some fantasies are moving and make us feel they are real, nothing is more real than the real world itself. You cannot ride centaurs or visit Archinland, but you can ride ponies and visit Iceland. You can see the real world with naked eyes instead of mental eyes. No need for imagination, you can perceive the real world with your own ears to hear and your own fingers to touch. You can build real friendship with real person. In this aspect, no fantasy can ever compare.
\par Back to the real world, we have to confront all the troubles. That is unavoidable. Fantasies can be free of worries and cruelty just because it can be modified by our mind, or subjective; the real world can be harsh just because it is objective. The real world should always be our main battlefield. We are to achieve the value of our life on this land, this soil. During my escape, my battery is recharged. And now, let it break in all its fury. 
\par \rightline{December 18, 2016}

\end{document}