
\setlength{\parindent}{2em}
\section*{苏联解体的原因与启示——《来自上层的革命》读书报告\footnote{本文为作者大一年级《思想道德修养与法律基础》课的读书报告之一。}}
\addcontentsline{toc}{section}{苏联解体的原因与启示}

\par 美国学者David M. Kotz和Fred Weir撰写的《来自上层的革命——苏联体制的终结》回顾了苏联建国至今的历史进程,重点讲述了戈尔巴乔夫如何在民主社会主义改革中葬送了联盟。书中提出的独特观点——苏联内部拥有统治地位的党—国精英选择资本主义,是苏联解体的主要原因——是令人信服的。同时,本书还批判了“社会主义失败论”的观点,指出苏联只是一次建立社会主义的大规模尝试,社会主义之路才刚刚开始。
\par \textbf{1.苏联解体是一次来自上层的革命}
\par 这一观点并不符合世界的主流,但作者首先让我们看到,众多的主流观点其实并不足以自圆其说。
\par (1)社会主义在经济上不可行。事实上,苏联的前60多年经济出现了世界历史上少有的高速发展。按照苏联的官方统计,1928-1975年苏联的年度NMP(物质生产净值,苏联使用的标准)增长率达9.1\%;从西方经济学家估算的GNP来看,这一时期苏联的4.5\%增速也高于美国的3.1\%\footnote{ 大卫·科兹,弗雷德·威尔《来自上层的革命——苏联体制的终结》,中国人民大学出版社2002年版,第47页。}。并且,戈尔巴乔夫改革的前四年,苏联经济仍在增长。
\par (2)来自下层的群众运动瓦解了苏联。1991年3月的全民公决中,76.4\%的民众赞成保留联盟\footnote{同上,第192页};同年5月对俄罗斯属欧地区的民调显示,46\%的民众支持各种形式的社会主义,而主张实行美、德的自由市场资本主义的只占17\%\footnote{同上,第182页}。由此看出,尽管民众对现行的体制有所不满,但他们并不希望苏联解体,走上资本主义道路。
\par (3)外来势力瓦解了苏联。作者认为,如果西方的力量在苏联刚刚成立、国力尚弱时未能瓦解苏联,那么在它强盛时也难以做到。而我认为“生于忧患,死于安乐”,一个国家可以在新生时团结一心,克服种种困难,却在强盛时放松警惕。不过,这样一个外在因素终究不能成为主要因素,内在原因一般占主导地位。
\par 本书的结论是:苏联解体是一次来自上层的革命。
\par 面对长期僵化的体制和经济情况的恶化,戈尔巴乔夫决定通过大规模的变革使苏联走上民主社会主义道路。意外的是,他的改革使苏联的精英阶层中产生了一个亲资本主义的利益集团,并且为他们夺取政权铺平了道路。可以印证这一观点的是,很多原苏联共产党、共青团、国有企业的高层干部在苏联解体后成为新兴资本家\footnote{同上,第157-162页}。
\par 许多党—国精英轻易背叛社会主义,是因为他们本身就不具有坚定的共产主义信仰。他们入党并努力谋求提升,是出于实际的目的。他们享有特权和较高的生活水平,但比起西方同一阶层还是相对较低;并且在集权的体制下,他们不得不谨小慎微、讨好上级以保住自己的职位。
\par 资本主义的思想随着民主化改革到来,精英们的思想也发生巨大转变。私有制可以打破对积累财富的限制,使他们更好地维护自己的利益;市场经济更是自然地受到那些已转变为新兴资本家的精英的拥护。由此,精英阶层长期支持叶利钦领导的亲资本主义联盟,这在这次变革中起到关键作用。

\par \textbf{2.戈尔巴乔夫改革如何推动苏联解体}
\par (1)经济改革。苏联政府先后于1986年和1988年通过《私人劳动法》和《合作法》,允许私人企业和合作企业建立,与国有企业并存。但事实上,许多打着合作企业旗号建立的工厂实质是资本主义企业。后来国家监管日益减少,资本主义得以更加公开地发展\footnote{同上,第152页}。1988年,苏联政府允许私人公司参与对外贸易,众多私人企业通过出口石油、金属等物资赚取了大额利润\footnote{同上,第121页}。1987年,苏联政府给予企业充分的自主权,取消了中央对经济的日常管理,只保留长期的计划。然而,企业自主提高职工工资带来了消费市场混乱,同时企业的投资下降,国家税收减少,这一系列危机引发了人们对改革的质疑,促使了亲资本主义阵营影响的扩大。
\par (2)政治改革。1988年,苏联新选举法通过,由民主的方式选举产生人民代表大会,一些拥护激进改革的亲资本主义分子得以登上政治舞台。之后的选举民主化程度越来越高,1991年,苏联通过全民选举选出各共和国总统,亲资本主义阵营的领导人叶利钦得以当选俄罗斯总统,拥有了对抗中央的阵营。苏联共产党在改革中逐步放弃了在社会日常生活中的领导权,而让政府担当起这些职能。这使戈尔巴乔夫的权力大大减弱,以致改革无法推行,直至失去对局面的控制力。
\par (3)公开性政策。1986年,戈尔巴乔夫决定解除对公开讨论和个人意见表达的限制。他甚至鼓励对苏联党政机关的批评。许多自由派的知识分子成为苏联主流报刊的主办人。这为资本主义思想的传播创造了条件。
\par \textbf{3.戈尔巴乔夫改革与我国改革开放的比较}
\par 我认为,戈尔巴乔夫的改革的确是在向民主的方向努力,但他在许多措施上缺乏考量,在形势受到威胁时依然连连退让,导致了改革的失败。而几乎同一时期我国的改革开放则取得了举世瞩目的成绩。我发现,两者有以下几点不同:
\par (1)我国在引进非公有制经济时坚持了公有制的主体地位,坚持了国有经济控制关键领域。改革后,我国公有资产在社会总资产中依然占优势,能源、电力、通信、金融、交通等行业仍被国有企业掌控,中国石化、工商银行等国有企业至今处在我国企业前列。而苏联后期大规模的非国有化、私有化政策显得缺乏原则。
\par (2)我国坚持了共产党的领导,保持了政治的稳定。“坚持中国共产党领导”被作为四项基本原则之一写入基本路线,我国的私营企业家群体并没有在政治上对社会主义形成冲击的空间。而苏联变一党制为多党制,最终动摇了国家。 
\par (3)我国的经济改革发动了群众的力量。我国在农村推广家庭联产承包责任制,调动了农民投入改革的积极性。而戈尔巴乔夫官僚型的行事作风没能成功动员群众\footnote{同上,第201页。}
\par (4)我国对舆论有所控制。苏联改革中反对社会主义、宣扬资本主义一类的言论,是绝不会在我国当下的主流媒体中看到的。我认为,取消言论自由固然不可取,但由于我们不能保证群众保持清醒的认识、不被反动言论所迷惑,对舆论的必要控制是必须的,否则完全可能民心动摇。
\par 本文只着眼于苏联的最后时期——戈尔巴乔夫改革的教训,而苏联解体的根源,无疑与之前僵化的体制有很大关系。我国同样是社会主义的探索者,苏联各时期的历史是留给我们的宝贵经验。这些经验将帮助我们克服未来社会主义道路上的艰难险阻,继续推动这一人类历史上的光辉事业。 
\par \rightline{2016年10月6日}

\section*{什么是爱——《爱的艺术》读书报告\footnote{本文为作者大一年级《思想道德修养与法律基础》课的读书报告之一。}}
\addcontentsline{toc}{section}{什么是爱}

\par Erich Fromm的心理学著作《爱的艺术》引起了我对爱这一论题的思考与体系重构。在这篇读书报告中,我将梳理阅读此书后我对爱的内涵、来源、特性等方面的思考。本文所指的“爱”包括父母之爱、朋友之爱、夫妻之爱、对事物的爱等,但出于我无神论的观点,不考虑书中所提到的“神爱”。
\par 正如物理学上的时间、空间是难于定义的基本概念,爱在心理上也不容易给出明确的定义。因此,我把爱(Love)的内涵归纳为三点:欣赏(Appreciation)、奉献(Devotion)和期望(Expectation)。
\par 对事物的爱的产生常常是由于积极的审美感受:我爱星空,因为我欣赏它的浩瀚;我爱大海,因为我欣赏它的博大。喜爱蝴蝶的人比喜爱苍蝇的人多得多,也正是因为蝴蝶引起了更多人的欣赏。爱一门学科,也常常是因为感受到了它的美。对人的爱亦是如此,没有人愿意与自己不喜欢的人产生朋友之爱;而爱情可视为友情的特殊形式,它也由于两个人间的欣赏和吸引而得以产生和发展。父母之爱似乎首先出于责任。然而大多数父母还是会关注子女的闪光点,对子女有相对较高的评价,只有极少数父母难以感受到孩子值得欣赏的地方,完全是出于责任来履行父母之爱,比如孩子有先天疾患,只能靠父母照料生存的情况。所以,欣赏是爱的不可缺少的成分,很多时候是欣赏产生了爱,并为爱提供源源不竭的动力。
\par 我们是幸运的。旧社会的青年只能先接受父母的安排,再和自己的伴侣培养欣赏感,而我们可以让欣赏感自然地生长为爱情。正如电影《Flipped》所说:“Some of us get dipped in flat, some in satin, some in gloss. But every once in a while you find someone who's iridescent, and when you do, nothing will ever compare.”
\par 但正如作者在书中指出的那样,物质社会使得很多人仅仅把爱情看作是衡量双方条件后的一种交易\footnote{艾·弗洛姆《爱的艺术》,上海译文出版社2008版,第3页},而这种以物质为基础的爱情显然缺乏持久的动力。所以为了拥有真正的爱情,我们要能够欣赏他人;为了欣赏他人,我们要去了解和理解他人,而这要求我们首先走出自我封闭,敞开自己的心。
\par 爱在过程中和行动上化为无私的奉献,即时间、精力、情感上的投入。正是无私的奉献超越了依据投入与获利比较的一般理性行为,把爱与不爱区分开来。爱星空,就会花时间去观察、记录星空,或是为反对光污染尽一份力;爱一门学科,就会投入大量时间去学习钻研,以至为它贡献自己的成果。朋友有难,慷慨相助;父母对子女更是尽其所能、倾其所有。没有奉献的爱是难以令人信服的。
\par 作者说:“爱情首先是给而不是得\footnote{同上,第20页}。”他认为,我们不应首先谋求得到别人的爱,而是要先不求回报地付出爱,同时相信自己能够唤起他人的爱。而爱从来都不会是单向的给予,而是双向的分享。我们东方有语:“爱人者,人恒爱之。”从博爱的角度,当尊重、体谅和无私的帮助超越血缘和友谊,在更广泛的人与人之间传递和蔓延,这将有利于缓解全社会人与人之间的距离感,提升每个人的幸福感。
\par 爱源于欣赏,发展于奉献,而归于期望。这里的期望当然指积极的期望,不是对当前和过去条件分析得出的结果,而是不随实际情况改变的主观愿望。希望家人平安健康,就是这种期望的体现。父母会纠正孩子的错误,却不一定会指出别人家孩子同样的错误,这是因为父母希望孩子走上正确的道路。如果不爱他,就不必关心他的状况,因而就不必对他提出要求;有愿望,有期求,正是关注和爱的体现。在爱的行动的动力上,欣赏和期望或许是密不可分的。
\par 书中提到的爱的要素有“给”、关心、责任心、尊重和认识。与我的总结相比,“给”与奉献近义;关心是奉献的一种形式,因为关心也要投入时间和精力,同时是期望的必然结果;认识是欣赏的前提;尊重即是放弃控制和强行改变,它可以由欣赏和期望推出:我欣赏你的某种特质,那么我有什么资格要求你改变,放弃这种特质呢?反倒是我要当心不要迷失自我。我期望你有好的生活,那么我必然要尊重你独特的生活方式,因为那更适合你。至于责任,它是一种契约,除了极端的不负责行为会受社会谴责外,履行到什么程度则在很大程度上依赖于自觉,很多爱的契约甚至是可以解除的(比如离婚和朋友绝交),因此它的效力几乎完全依赖于爱的其它因素。总之,两种表述实质上是统一的。
\par 作者深表遗憾地认为,真正的爱情在当今西方社会是罕见的现象\footnote{同上,第77页}。这是因为大多数爱情不具备真正爱情的特性——他特别强调两点,我概括为独立性和统一性。
\par 爱的独立性是指在爱的过程中保持自己人格的独立与发展,而不是失去自我。作者认为,爱是“从自己生命的本质出发,体验到通过与自己的一致与对方结成一体,而不是逃离自我\footnote{同上,第95页}”。既不能把爱的对象偶像化,使自己的精神依附于人,也不能要求对方的人格依附于我。由此我想到许多文艺作品对爱情的描写其实是误导,诸如“我可以为你做任何事”这样的话,显然是缺乏理性思考、迷失自我的。
\par 爱的统一性是指爱不是仅针对一个具体的对象,而是对自我、他人、世界、生活的一种总体态度。“如果我能对一个人说‘我爱你’,我也应该可以说‘我在你身上爱所有的人,爱世界,也爱我自己\footnote{同上,第43页}。’”在这个意义上,爱是一个人内心修养的产物;研习爱的艺术,就是修养人的内心。
\par 最后,用书中我最喜欢的一句话作为本文的结尾:“爱情的存在只有一个证明:那就是双方联系的深度和每个所爱之人身上的活力和生命力。”
\par \rightline{2016年10月30日}

\section*{关于雇佣劳动的思考——《资本论》选篇读书报告\footnote{本文为作者大一年级《思想道德修养与法律基础》课的读书报告之一。}}
\addcontentsline{toc}{section}{关于雇佣劳动的思考}

\par 在两个月前的我看来,上班打工、挣得工资是一件再正常不过的事情:工人有劳动力,却没有生产资料,没有施展劳动力的平台;企业主则为他们提供了这样的平台,这可看成社会发展进程的一种合作形式;企业主占有了产品,但也给工人发放劳动报酬,看起来是公平的;况且工人的工作是自己选择的。而且从解决生计问题的角度看,给一个人工作做,似乎还是一种恩惠。
\par 几堂政治课后,我开始认识到资本主义远不像它看起来那样体面、光鲜;读过几十页《资本论》之后,我知道:资本家的收益恰恰来源于他所雇佣的劳动力;资本家付出的仅仅是劳动力的价值,即他所需的生活资料的价值;他所创造的远大于这一数量的价值,则被贪婪的资本家全部占有。在雇佣劳动的关系中,双方绝不是平等的。
\par \textbf{1.雇佣劳动的不平等性}
\par 马克思指出,单纯的商品流通并不能解释剩余价值的产生。我们知道商品的交换是以等价交换为原则的,即使考虑到价格在价值上下波动,也无法理解资本家获取的巨额利润。那么只可能是资本家所购买的商品的使用价值给他带来利润。这种商品就是劳动力;这种使用价值就是创造价值。
\par 将劳动力视为商品,这一观点让人眼前一亮。这就容易理解劳动力之间的工资差异:资本家向工人支付的工资表现了工人劳动力的价值,它既包括生产劳动力的费用(即工人维持生计所需生活资料的价值),也包括工人的教育、培训费用。所以技术工人的工资高于普通工人,正是因为技术工人接受的教育、培训也要耗费一定的社会资源。
\par 从理论上看,我们为满足生活所需、承担社会平均劳动义务而劳动的时间要小于为雇主实际工作的时间。这之间的差值,本该是属于我们的自由时间!而仁慈的雇主迫使我们相当于是为他无偿地劳动!这难道是公平的吗?
\par 从史实看,这种不平等性几乎是无可辩驳的。在金权至上的资本主义国家,国家权力掌握在资产阶级手中。他们可以运用这种权力轻而易举地维护和扩大自己的利益。
\par 早期的工人没有八小时工作制的保护,拼命地工作也只能换取微薄的工资,勉强度日;大量工人死于职业病和安全事故。他们体会不到自己的尊严,他们的生命只是资本家账本上的一个数字。这一切激起了工人的反抗,包括欧洲三大工人运动、共产国际等。现在看来,受利益驱使过度剥削工人的行为也是不明智的。作为资产阶级,应当维护自己统治的稳定,而这必须要求工人的持续存在。工人都死于剥削,谁来为资本家们劳动?
\par 所以我们现在看到,资本主义自身也在进步。它在不放弃自身核心利益的前提下作出调整,以缓和阶级矛盾。比如现代工人的生产条件、工作时长都有了一定的保证。但雇佣劳动的不平等性并未改变。生产力提高和经济波动引起的裁员,使留任者话语权降低,被迫更努力地工作;贫富差距依然悬殊;更简单的例子是,时有老板拖欠工人工资,却并未听说有工人拖欠劳动。
\par \textbf{2.个体在异化社会的生存}
\par 资本家或许可以这样为自己的行为辩护:我之所以有能力剥削你们工人,是因为我掌握生产资料。社会中向上的路是开放的,你们也可以通过奋斗去当老板啊!
\par 社会中向上的路必然要开放。封建社会都有“学而优则仕”的制度。开放意味着希望,不管成功率多少,有希望才有奋斗的方向。工人阶级中总有不满自己境况的人,有希望时他把不满化作自身动力,为生活打拼;无希望时就只有召集众人一起反抗,把现有制度砸个粉碎,创造出一条“社会中向上的路”。
\par 当今社会,阶层之间的流动是存在的,但并不容易,至少比上层的人维持自己的地位要难得多。有经济条件支撑,上层的人容易受到良好的教育;有社会人脉支撑,上层的人容易在社会中求得发展。掌权阶层只要保证自己的位置还基本坐得稳就够了。
\par 成功走上更高阶层的人,只是对生活境况做出抗争的人中的一小部分;对生活境况做出抗争的人,只是对生活境况不满的人的一小部分。更多的人选择了忍受。而这恰恰是资本家最希望看到的。
\par 忍受的原因当然有很多,比如自身缺乏勇气,小家庭的温情港湾充当麻醉剂。我不知道企业宣扬的企业文化和行业责任感是不是也有这个目的:心里认同了我是在为社会做贡献,我的生命是有价值的,自然就有更强的忍耐力了。
\par 反观资本主义社会的掌权阶层——资本家,他们的物质需求容易满足,似乎可以尽享人生欢乐。与对工人的有形控制不同,这一异化的金权社会对资本家的作用更多是无形的,思想观念的改变。挥霍般的享受,更多地敛财供自己挥霍,成为他们的人生内容,本来要做的事情,被这种空虚的循环取代。他们剥夺了许多人人生的精彩,同时葬送了自己的人生。
\par 总之,少数人的成功改变不了少数人与多数人对立的事实。老板总不能比雇员多。
\par \textbf{3.自由劳动}
\par 我们做一个理想的假设:我们只需要承担社会的必要劳动,剩余时间归自己支配。我们做什么?
\par 休息?娱乐?难道人生的意义和价值就这样体现?
\par 我认为,只要一个人接受了一定的教育,有一定的见识,并且对生活的态度并不消极,他就会给出这样的答案:工作。
\par 教育和见识使他知道世界上的很多领域。积极的态度使他对其中的一些领域产生兴趣。人生短暂,可每个领域都有做不完的工作。所以他会在社会必要劳动之外继续工作,工作在自己感兴趣的领域,只争朝夕。
\par 这种工作不同于雇主强制的要求或是为生计而工作,他的所有动力来源于领域本身和工作者本身,不受异化的影响,我称之为自由劳动。这也正是共产主义社会的美好愿景,它可以使人的才能得到充分的发挥,它的成果将远大于社会必要劳动部分的成果,并推动社会更好地发展。
\par 在马克思的时代,劳动者的工种很有限,不容易想象在工业流水线上工作的工人有很强的内源动力。随着社会的发展,社会上已经出现了接近自由劳动的可能。需要说明的是,自由劳动并不代表没有约束;劳动需要组织,有组织就必然有约束。接近自由劳动是指大部分的实际工作与自身愿望和规划重合。
\par 在现实社会,不得不首先考虑的是生计问题,但这并不难解决。只要一个人自由劳动的领域是有益于社会和人类的,从认可这一点的人处得到报酬就是可行的。而且获得报酬、维持生活中的种种需求只是工作的副产品,因为他的追求就是工作本身。
\par 实现较大程度的自由劳动,是个人生活的追求,也是个人能力的体现。它的杰出成果也将有利于社会的进步和生产力的发展,加快资本主义向社会主义、共产主义过渡的过程。
\par 看看新大陆发现后贩运黑奴的船只,我们就知道什么是资本主义的道德。资本主义必将被终结。那时我们才能看到真正的平等和自由,真正的道德。
\par \rightline{2016年11月7日}

\section*{现实的法学,冷静的法学人——《制度是如何形成的》读书报告\footnote{本文为作者大一年级《思想道德修养与法律基础》课的读书报告之一。}}
\addcontentsline{toc}{section}{现实的法学,冷静的法学人}

\par 我的专业和兴趣领域与法学没有多少交集。听说法学院的同学在学《民法总则》,我到教材中心略一翻阅,复杂的名词、枯燥的概念让我一页也不想读。但苏力教授所著《制度是如何形成的》让我在思考中愉快地读了下去。
\par 阅读此书,我的收获主要有两方面:既有对法学特点的新理解,又有作者的冷静态度与质疑精神。
\par \textbf{现实的法学}
\par 在此之前,我接触过一点《洞穴奇案》的内容。对于极端条件下的一个杀人案件,有的法官主张判处极刑,有的却认为他们无罪,道理却都很有说服力。我一度陷入深深的思考:同样理性的思维却推出迥然不同的结果。如果我是法官,我该如何判决?这样的问题大概归结到法理学或法哲学的范畴,于是我觉得相对抽象的法理学、法哲学的研究很重要、很有意义。
\par 我还畅想过,如果自然的、公正的法律取代了物理规律成为世间万物运行的准则,世界将会怎样?初想感觉这很美好:当杀人犯将刀刺向受害者,受害者安然无恙,杀人犯将会死亡;合法的财产永远都不会被偷走,非法的财产永远都无法取得…
\par 然而且不说物理上如何实现这样的机制,自然首先要确定好一套绝对公正的是非准则,(对于杀人这样的问题还容易界定,更细节的事件如商业侵权、精神损害,自然如何界定?更何况是非准则随时代在改变,我们当然可以说历史上的文字狱是封建皇权的局限,不应作为准则,但我们难道可以说今天的法律没有局限吗?)还要确定好一套适度的处罚准则,(我们显然知道严刑意味着暴政,量刑太轻又不足以威慑潜在的罪犯,那么这二者的临界值又是多少?自然如何做到精准把握?)还必须是一个全知全能的上帝角色,知道每个人在任何时候做的任何事,不能有漏网之鱼。即使这些问题都解决了,还有最关键而又最可笑的问题:当“杀人犯”将刀刺向受害者,不推演下去,你怎么知道他会死呢?如果按物理规律推演下去受害者只会受轻伤,让“杀人犯”去死,岂不有失公正?
\par 那好我们做一点修正,让自然知道每个人的想法,根据举起刀时他脑中“我要杀了他”的想法来判决死刑。问题更多了:有时凶手想的是“我要报复他”、“我要刺伤他”实际会是杀了他;有时凶手想的是“我要杀了他”,却因为对方武艺高强只造成轻伤。这样判决就显得有失公平。再者说,恐怕脑中闪过“我要杀了他”的人数比实际作案并成功的要多好多倍,这样要多处罚多少人,会不会造成恐怖气氛?还有有时“我要杀了他”是一种抒情方式,并不是真要杀人,自然识别语义稍有不当,就会造成冤假错案…
\par 这番思考再告诉我,并不是抽象的法理不重要,但绝不是有一套法理就可以解决社会秩序的调整问题,正如书中所说:“法学…一个重要特点是务实和世俗。”(《反思法学的特点》)面对同样的法律,一般的法庭调解撤诉率为20\%左右,宋鱼水和金桂兰则因调解撤诉率达到70\%以上成为模范法官,可见法律的执行者很重要;《民事诉讼法》等规定诉讼程序的法律的出台,也表明法律发挥作用的程序也很重要。法律在实际运行中涉及的因素是多方面的。
\par 反观书名,制度究竟是如何形成的?不是个别人头脑中的应然规划,而是“社会多种因素制约的产物”。(《认真对待人治》)。比如书中论及美国司法审查制度的确立时,认为它并不是马歇尔法官个人智慧的必然结果,而是各党派政治家斗争和妥协的产物。但是“制度实际发生的作用和意义并不因起源的神圣而增加,也不因起源的卑贱而减少”。(《制度是如何形成的》)
\par 我想,在法学中的确不能引入各种理想化的假设,而是要在承认当前社会所有条件的前提下,构建一个能够比较好地维护社会秩序,实现社会总体效益的系统。幻想的世界中没有法律,法律不需要幻想。
\par \textbf{冷静的法学人}
\par 阅读本书,时常能感受到作者看待事物的冷静、客观立场和批判、质疑的精神,这的确令人佩服。
\par 面对英国王妃戴安娜的死,作者并没有像一些大众媒体一样简单地谴责追逐戴安娜所乘轿车的摄影记者,而是客观地指出,摄影记者并没有造成戴安娜的死亡,至多只是因素之一,关键的因素是司机酒后驾车、超速行驶和死者没有系安全带。有的人为戴安娜的死而悲痛,或痛恨某些记者对名人私隐的打探,但这不是他们不公正地把主要责任推给记者的理由。(《我和你深深嵌在这个世界之中》)
\par 对于宣判之前“罪犯”只能称为“犯罪嫌疑人”之一问题,很久以前准备司法考试的妈妈就给我讲过,我不觉得有什么问题。然而作者却发觉这并不能一概而论,而是要有特定的语境,即“司法机关在判案时不能先入为主地认定被告就是罪犯,而是要用证据来证明被告是否罪犯。”而对于受害者一方,如果确知那位“犯罪嫌疑人”就是伤害自己的人,称之“罪犯”并无过错。(《罪犯、嫌疑人与政治正确》)
\par 在《“法”的故事》一文中,作者对当下几乎所有版本《法理学》教科书中对“法”字起源的解释产生了疑问:它们都采用许慎《说文解字》的解释,即水旁代表“平”,即公平。但象形文字的“水”显示的波纹方向和流动特征表明古人更多关注的是水自上而下的流动。为什么它在“法”中代表的不是法律自上而下颁布施行的特征呢?从这些文字,我看到“凯蒂旺普斯”的精神,而我觉得自己读到类似看来合理,似乎无需置疑的结论,是倾向于接受的。
\par 作者在一篇书评中这样评价波斯纳:“从不迷信前人研究的结论,不相信流行的结论,不相信别人对这些学者思想的概括或其他第二手的资料,总是认真研读第一手的材料。”即使是面对第一手资料,也不能失去批判的眼光。我想,这种严谨认真的态度是没有学科界限的,相对于理论描述相对精确,不易被曲解的自然科学,人文、社会科学尤其需要。
\par \rightline{2016年12月10日}

\section*{从乌托邦到新和谐公社——莫尔与欧文社会主义思想之比较\footnote{本文为作者大二年级《马克思主义基本原理》课的课程作业。}}
\addcontentsline{toc}{section}{从乌托邦到新和谐公社}
\par 16世纪初期,新航路已经开辟,正是大航海时代的兴盛之时;英国杰出的人文主义者托马斯·莫尔所著描绘海外理想国的《乌托邦》正符合这一时代背景,给当时人以真假难辨之感,这也成为空想社会主义早期的重要文献。莫尔所处的时期资本主义还处在原始积累阶段,发达的资本主义工业还未产生;而三百多年后,三大空想社会主义者之一的罗伯特·欧文所处的英国已经处于工业革命中,“蒸汽和新的工具机把工场手工业变成了现代的大工业……社会愈来愈迅速地分化为大资本家和无产者”\footnote{恩格斯《社会主义从空想到科学的发展》,《马克思恩格斯选集》第3卷,人民出版社1972年版,412页。}。欧文目睹工人悲惨的生活境遇,在管理苏格兰新拉纳克大棉纺厂时进行了改善工人劳动和生活条件、创办教育等试验,取得了闻名全欧的成效。从一名资产阶级慈善家转向共产主义者后,欧文在印第安纳进行“新和谐公社”的试验,但公社以破产告终。
\par 将莫尔与欧文的社会主义思想进行对比,有助于我们结合时代变迁,理解空想社会主义三百多年间的发展脉络。此外,他们的理想社会设想对于我们未来社会的构建同样有参考价值。下文的对比将从对社会现实的批判和对理想社会的构想两方面展开。

\par \textbf {1.对社会现实的批判}
\par 在莫尔所处的时期,英国在都铎王朝统治之下,在位的亨利八世是一位残暴的君主。莫尔身为伦敦行政司法长官,目睹了社会的种种黑暗和百姓的不幸。在《乌托邦》第一部中,莫尔借拉斐尔·希斯拉德之口,对英国社会进行了隐晦但严厉的批判。
\par “你们的羊,一向是那么驯服,那么容易喂饱,据说现在变得很贪婪、很凶蛮,以至于吃人,并把你们的田地、家园和城市蹂躏成废墟。\footnote{托马斯·莫尔《乌托邦》,商务印书馆1982年版,20页。}” 这段著名的话,就是莫尔用来揭露“圈地运动”这一资本原始积累过程的。在这一运动中,贵族将大片田地变为饲养羊的牧场,许多农民在欺诈、暴力手段之下被剥夺了赖以为生的田产,只好举家流浪。更残酷的是,当时的大贵族、大商人与政府勾结,用残暴的刑罚迫害底层民众,流浪、讨饭者被抓入监狱,大批盗窃者被处以死刑。
\par 除此之外,在拉斐尔对欧洲各国王室的批评中,莫尔还批判了君主对外的尚武好战、侵犯别国利益,对内的聚敛财富、压榨百姓;臣僚的阿谀奉承等。这些议论虽是借其他欧洲王室提出,但都直指英王。莫尔还借拉斐尔之口提出了自己的政治主张:与其侵犯别国,谋求扩张领土,不如用心治理好已有的领地;国王应该为百姓谋福利,而不是搜刮百姓:“国王所统治的不是繁荣幸福的人民,而是一群乞丐,这样的国王还像什么话!\footnote{同上,38页。}”这一反对征战、提倡仁政爱民的主张,与我国孟子的思想有相似之处。但正如书中拉斐尔自己评述的那样,这些观点完全背离当时统治者的主张。
\par 在莫尔的时期,资本主义生产关系所带来的对工人阶级的剥削还不明显,莫尔所批评的更多是一种封建时期的暴政,从中可以看出莫尔的正义感和人文关怀。
\par 而在欧文的时期,资本主义生产关系已经比较发达,资本家对工人的剥削也十分残酷,具体包括童工的雇佣,低劣的劳动和生活条件,过长的劳动时间等。欧文在著作中就曾提及,他的新拉纳克大棉纺厂规定工人每天劳动十个半小时,而同业工厂强迫工人每天劳动十三、十四甚至十五小时\footnote{《人类思想和实践中的革命或将来从无理性到有理性的过渡》,《欧文选集》第2卷,商务印书馆1981年版,102页。}。
\par 然而,欧文并没有在著作中激烈地抨击资产阶级的丑恶,他甚至反对工人阶级用暴力手段反抗资产阶级。其原因是,欧文认为一个人的性格和所作所为是由外力即他人生中所处的环境决定的,他将其视为重要的真理,并在著作中反复强调类似的观点。他说:“(资产阶级)也是由于他们所不能控制的环境而变成了你们(劳动者)的敌人和凶恶的压迫者……他们对这些结果应负的责任并不比你们更大……\footnote{《告劳动阶级书》,《欧文选集》第1卷,商务印书馆1981年版,171页。}”因此,他号召劳动阶级放下对剥削阶级的仇恨和愤怒,以期在不使用暴力的条件下实行新措施,实现所有阶层的共同利益。
\par 欧文期许:“……(较高阶级)只会在为了你们的利益而提出的一切改革中仅仅要求把自己的利益至少保持在现有的水平上\footnote{同上,177页。}。”但事实上,资产阶级中不乏贪得无厌者;在众多工厂主都把工人剥削到极致时,也只有欧文能够为工人着想,改善工人的条件。此外,资本家加大剥削也有行业竞争、追求扩大生产的因素。由此看来,欧文提出的劳动者与资本家就共同利益达成共识、推动改革的主张,虽符合其“泛爱全人类”的原则,确是理想化和不切实际的。

\par \textbf{2.对理想社会的构想}
\par 对比莫尔笔下的乌托邦国和欧文建立的新和谐公社可以发现,二者在财产公有、平等、民主等准则上具有一致性。
\par (1)财产公有
\par《乌托邦》在第一部中旗帜鲜明地表明了公有制的理念:“任何地方私有制存在,所有的人凭现金价值衡量所有的事物,那么,一个国家就难以有正义和繁荣。\footnote{托马斯·莫尔《乌托邦》,商务印书馆1982年版,43页。}”第二部中则细致地描述了乌托邦的公有制。在乌托邦,居民没有固定的地产和生产工具,市民轮流到村庄中居住务农,城市居民定期抽签更换房屋。物资实行按需分配,所有产品运到城市中的市场,各户人家到市场根据需要领取物资,分文不付。由于一切货品供应充足,没有必要储存物资,而且乌托邦人也并没有以占有物多于别人为荣的风尚,因此没有人领取超出所需的物品。每三十户居民集中在一个厅馆中共同用餐。城乡之间、不同城市之间物资同样是共享的。
\par《新和谐公社组织法》也规定了财产公有的原则。此文指出:“以个体所有制为基础的制度,必然反对人们权利平等的原则……反对这一原则,就会引起竞争和敌视、嫉妒和纷争、奢侈和贫困、专横和奴役\footnote{《新和谐公社组织法》,《欧文选集》第2卷附录,商务印书馆1981年版,187页。}。”由此,莫尔和欧文都视私有制为万恶之源,主张实行公有制以维护广大群众的利益、实现公平正义。
\par (2)平等
\par 乌托邦国的平等首先表现在人人都必须劳动,只有极少数高级官员和专门从事学术研究的人除外。劳动以务农为本,居民兼干自己的一两门手艺。乌托邦人每天只劳动六小时,却能生产出足够本国使用、还可出口海外的产品,作者认为人人从事生产劳动便是主要原因,相比之下其它国家由于妇女、僧侣、绅士、贵族及仆从不从事劳动,还有不少人从事“不实用的、多余的行业”。除此之外,乌托邦人衣着统一朴素,房屋轮换等处也体现了平等理念。
\par 《新和谐公社组织法》同样把“所有的成年人,不分性别和地位,权力一律平等”和“随着智力和体力的适应程度而变化的义务一律平等”列为前两条原则。在供应允许的范围内,全体社员得到同样的食物、衣服、住宅和教育。
\par (3)民主
\par 在乌托邦,每三十户人选出长官一人(称为“摄护格朗特”),每十名“摄护格朗特”及掌管的各户隶属于一名高级长官(称为“特朗尼菩尔”);所有“摄护格朗特”秘密投票选举全城的总督。除此之外,教士也由国民选出。城市中的重要事务会提交摄护格朗特会议,由摄护格朗特通知管理的各户开会讨论。由此,乌托邦国民有选举、参与公共事务等民主权利。
\par 在“新和谐公社”,全体年满21岁的公民组成全体大会,掌握立法权和书记、总经理、司库、管理员等职的选举权,书记、总经理、司库、管理员组成的执行理事会掌握行政权,每周向全体大会报告工作。通过全体大会,社员的民主权利也得到实现。

\par 以上三点中,民主、平等是资本主义政治制度也具有的概念,但公有制与资本主义制度有鲜明的区别。在财产公有、人人劳动的条件下,平等不再是契约关系下的形式平等,而是实际的平等;民主也不再是资产阶级式的民主,而是社会主义的民主。

\par \textbf{3.若干问题的讨论}
\par (1)公有制下劳动的积极性
\par 我国在社会主义道路的探索中一度因急于向共产主义过渡遭遇挫折,其主要原因是平均主义的分配挫伤了劳动的积极性,“干多干少一个样”,就必然面临劳动积极性如何保证的问题。有观点认为,将来进入共产主义时,劳动成为人的第一需要,人的道德水平普遍提高,也就不存在分配平均化导致的消极怠工问题。但如何使道德水平普遍提高,在劳动风气良好的条件下如何处理少数人的消极怠工问题,依然是需要回答的。
\par 在《乌托邦》中,多数公民并不能自然地乐于从事农业劳动,依然认为农业劳动是艰苦的。但这并不妨碍农业劳动高效有序地进行,原因有每日劳动时间不长,以及农业人员更换的制度使一般人免于长期从事艰苦的工作。但我认为更重要的是深入人心的义务观念,即从事农业生产及自己的专门手艺是自己对社会承担的义务。这在基本上是道德约束,同时也有制度约束。根据规定,居民可以去其它城市或郊区旅行,但要在到访的地方参加农业生产或干本行的活,没有借口逃避工作,这可视为这种约束的一个事例。
\par 制度可以约束公民参加工作,却难以对工作的效率和质量进行精细的管控;以怎样的认真程度和投入程度对待劳动,更多地还是依靠自觉。因此公有制下劳动的实行确需要一定的道德基础。
\par (2)公有制下按需分配的条件
\par 毫无疑问,人们需求能得到完全满足的按需分配需要“物质极大丰富”作为基础。但怎样的丰富才是“极大丰富”呢?如果每个人的需求都无限制地上升到很高的水平,恐怕社会难以满足这样的需要。
\par 在乌托邦国,这种按需分配的体制得以正常运转与国民的观念和生活态度相关。比如,公民没有攀比占有物的习惯;而且,公民性情淳朴,不追求华丽的服装和饰物。的确,如果人人梦想过上奢华的生活,按需分配的制度是难以满足所有人需要的。因而按需分配的实行也需要倡导一种拒绝浮华和矫饰、追求朴素和实用的生活态度。
\par 另一方面,乌托邦的设定依然是农业为主的早期社会,其产品的丰富程度与现在无法相比。汽车、计算机、手机……如此丰富多样的产品,如果按需分配,需要是否也难以满足呢?事实上,相当多的产品可以按照共享的方式使用,如自行车完全可以采用当前共享单车的模式,只是免费使用;计算机如果做不到人手一台,可以为每个社区配备一定数目,供大家需要时使用,这也使得资源得到更充分的利用。
\par (3)莫尔与欧文构想的共同局限性
\par 恩格斯指出:“对所有这些人(空想社会主义者)来说,社会主义是绝对真理、理性和正义的表现,只要把它发现出来,他就能用自己的力量征服世界,因为绝对真理是不依赖于时间、空间和人类的历史发展的。\footnote{恩格斯《社会主义从空想到科学的发展》,《马克思恩格斯选集》第3卷,人民出版社1972年版,416页。}”莫尔和欧文也是如此,他们致力于寻找一个超越时代因素的理想的、完美的制度,但没有为制度建立坚实的现实基础。莫尔详细描述了乌托邦国的方方面面,却没有清晰地交代这一制度建立的过程,把一切归功于开国君主的伟大设计;欧文也没有找到通往共产主义的正确道路,寄希望于开明的统治阶级实行改革,忽视了无产阶级的革命力量。因而他们不乏智慧的设想只能成为空想。
\par 现在我们认识到,任何制度都与一定的时代、一定的生产力基础相联系,脱离现实、凭空设计的“理想制度”终会因缺乏进入这一制度的道路或是制度不与实际相适应而告失败。但另一方面,与当前状况真正适应的制度对于我们同样是未知的,这就需要我们进行持续的制度探索和制度创新,改善制度中不符合实际、不合理的因素。
\par \rightline{2018年5月25日}

\section*{不平凡的征程——读《变革中国:市场经济的中国之路》\footnote{本文为作者研究生一年级《中国特色社会主义实践研究》课的课程作业。}}
\addcontentsline{toc}{section}{不平凡的征程}

\par 众所周知,实行改革开放是二十世纪中国的一次重大转折,它帮助中国走出十年动乱的阴影、重新焕发生机和活力,并在几十年间创造了经济增长的中国奇迹。由R.H.科斯和王宁合著的《变革中国——市场经济的中国之路》系统阐述了上世纪七十年代以来中国探索经济体制改革、最终确立社会主义市场经济体制的历程。此书以实地调研为基础,参考大量中外文献,较为清晰、全面地梳理了中国经济体制改革的脉络,且不讳言改革中出现的失误和问题,对完整认识这段历史具有一定的参考价值。下面基于我阅读此书第三章“中国市场体制改革起源”和第四章“笼中之鸟:社会主义下的市场经济改革”的体会,谈几点认识:
\par 正确的思想路线对于制度的制定具有重要作用。我国一些重要的改革措施,如家庭联产承包责任制,开放个体经济等,其背景都与社会中的现实问题有关。人民公社的集体耕种模式缺乏激励,导致效率低下;最先自发实行包产到户的四川九龙坡村(1976年9月)和安徽小岗村(1978年底)都是有名的贫困村,包产到户使其粮食产量明显增长,并最终得到了政府认可。随着1978年城市青年上山下乡政策的废止,1980年前后我国各大城市出现了大批返城知青,部分城市的规模达到城市总人口的10\%以上。国有企业和集体企业无法容纳这样的劳动力群体;青年群体长期失业,发起了各种抗议活动。为解决就业压力和社会动荡风险,政府开放了个体经营,允许待业青年自谋出路。上述政策的实行得益于“实践是检验真理的唯一标准”这一思想路线;它们被实践证明能够解决社会中的实际问题,因而能够克服意识形态障碍推行开来。倘若停留在“两个凡是”的方针之下,被毛主席定性为资本主义模式的包产到户是绝不可能推行的。
\par 改革政策的推行往往经历了较长期的试验、摸索过程。至十一届三中全会之时,包产到户还被明确规定为违法犯罪行为。到1979年,尽管包产到户已在暗中蔓延,官方文件仍规定“不许分田单干”。1980年中央终于允许包产到户,但仅限于边远、贫困落后地区;直到1982年才正式批准包产到户。个体经济在1981年得到正式认可,但一段时间内仍受到政策限制和社会歧视。个体经济实体不能雇用超过7名员工,否则属于“走资本主义道路”的违法行为;政府奉行“三不”措施:“不推崇、不宣传、不禁止。”直到1992年市场经济体制正式确立,个体经济的地位才有所改观。改革常常伴随对传统观念的更新乃至颠覆,其被社会普遍接受需要较长期的过程,绝非轻而易举。另一方面,“摸着石头过河”也降低了失误的风险,避免疾风暴雨式改革中的错误举措带来灾难性的后果。
\par 不能因改革取得的巨大成就忽视从改革中吸取教训。经验和教训都是几十年改革历程留下的宝贵财富,我们需要反思改革中出现的问题和失误,尽可能避免在类似的情境下再犯错误。1984年开始的银行业改革中,货币政策处理不当导致了1988年两位数的通货膨胀,恐慌性购买在多个主要城市中蔓延;价格体系改革中的“双轨制”下,同官员和国营企业管理者关系密切的人可以仅凭一个皮包公司,从政府渠道获取原材料,再在市场上高价卖出,轻易牟取暴利,导致腐败行为滋长;后来国有企业改革的管理层收购中,企业管理者有机会通过“合法地做不好企业”压低收购价格,造成国有资产流失。我们不能因害怕出现问题而停止改革的步伐,但反思这些问题出现的原因、思考更优的解决方案仍是我们应该去做的。
\par 作为世界第二大经济体,我国经济市场主体众多、影响因素复杂,特别是在融入经济全球化的今天,世界市场的风险和危机更容易波及我国市场。在深化改革的今天,把握中国经济的航向无论如何都并非易事。过去几十年我国的经济改革走过了不平凡的征程,新的征程需要我们坚持实事求是的思想路线,坚持四项基本原则、改革开放;具备过硬的经济学知识和素养,深入了解我国实际情况和世界经济形势,并充分借鉴之前改革中的经验教训。

\par \rightline{2020年10月16日}

\section*{试议东西方文化差异及其根源——读《东西文化及其哲学》\footnote{本文为作者研究生一年级《中国特色社会主义实践研究》课的课程作业。}}
\addcontentsline{toc}{section}{试议东西方文化差异及其根源}

\par 我国拥有悠久的历史和绵延不断的文化,这为中国特色社会主义文化建设提供了丰厚的积淀。立足新时代的我们,仍有必要审视我国数千年来的文化脉络,并通过与其它文化的互视,在世界文明的视野下认识中华文化的特点和地位。近代著名学者梁漱溟的著作《东西文化及其哲学》作为一家之言,可供参考和借鉴。此书第二章和第三章《如何是东方化?如何是西方化?》的阅读,引发了我对这一论题新的思考。
\par \textbf{1. 梁漱溟关于东西文化比较的观点}
\par 梁漱溟认为,西方文化是由“意欲向前要求的精神”产生民主与科学两大特点的文化。
\par 书中论证科学与民主是西方文化的特点。相比于我国古代偏重于技艺经验的总结,西方把零碎不全的经验、知识进一步探讨,经营成科学。技艺可能存在主观差异,而科学追求“客观公认的确实知识”,前人所得,后人全部继承;脚踏实地、逐步前进。梁漱溟用批判的眼光指出,作为我国古代自然观根本的阴阳五行学说是一种玄学,使得其它观念也被玄学化。我国的政治传统是专制的,依靠严格的等级制度维持社会稳定;“治人者”与“治于人者”分离,且统治者有无限的权威。而在西方民主传统下,不存在统治者与被统治者的划分,每个人都有权参与公共事务,同时个人权利不容侵犯。梁漱溟将“民主”概括为“个性伸展,社会性发达”,前者指自我意识觉醒引发的平等、自由观念;后者则是民主制度下人们积极参与公共事务、协调各方意见的表现。
\par 本书进一步指出西方科学与民主产生的根源是“意欲向前要求的精神”。梁漱溟不赞同将文化的发展归因于自然环境等客观因素的观点。他将文化描述为人的生活态度,并根据佛教哲学的观点给出三种分类:向前要求,即改造局面的奋斗态度;调和自己的意欲、随遇而安;向后要求,即取消自己的要求,追求出世和解脱。科学的还原论观点、对信仰的怀疑和破除;民主对威权的反抗,都是第一种的体现。与西方文化“向前要求”相对,他认为中华文化以调和、持中为核心精神,而印度文化属于“向后要求”。事实上,西方文化的两个传统,希腊和希伯来分别代表上述第一种和第三种态度。文艺复兴结束了中世纪的“翻身向后”,回到“向前要求”的态度上来。

\par \textbf{2. 梁漱溟观点不能成立之理由}
\par 如前所述,梁漱溟对东西文化比较的观点可分为现象和根源两个层次。对于现象层面,用“民主”与“科学”概括近代西方文化是比较恰当的,但需要注意的是这一概括的时代限定。在黑暗的中世纪,这两种特征西方文化都不具备。而在当代世界,西方科学的思想和知识体系已在世界范围内得到广泛接受,民主也几乎成为国家政治中的“政治正确”,“民主”“科学”也出现在我国的社会主义荣辱观、社会主义核心价值观中,不再是西方文化的特有属性了。对于根源层面,积极奋斗、调和持中、禁欲出世的划分对于个人的生活哲学是合适的,但将其用于概括文化的“生活态度”则并不恰当。首先,一个文化群体内部每个人的生活态度有所不同,任何文化、任何时代都不免有激进的斗争者和超然物外的修行者。即使对于社会的主流趋势而言,西方“向前”,中国“调和”的观点也未必合宜。如果说西方的资产阶级民主革命是反抗威权,是“向前”的表现,那么中国历史上众多反抗封建统治的农民起义,不也是“向前”的表现吗?二者的差异更多在于中国尚无君主立宪或民主共和制度之概念,结果只是另一个封建王朝的兴起。我国从秦汉时期便开拓疆土,前有张骞通西域,后有郑和下西洋。这样的壮举难道是“调和”的态度可以概括的吗?再次,所谓生活态度在各个时期也有所不同,正如梁漱溟本人承认西方中世纪走的是“向后”的道路。而如果民主科学是“向前”的表现,那么当今大多数国家都转向“向前”的道路了。这样一种随时而变,没有定准的东西,可以成为文化特性的根源吗?
\par \textbf{3. 我的观点——文化系统的形态发生}
\par 对于文化系统的发生,一个显见的特征便是其继承性。当新个体降生到一个文化群体时,他必要首先服从这一群体的规则和秩序。这类似于树的生长,最初的走向对其后的发展影响是极为深远的;当文化系统沿一个方向演进得越久,要改变方向所付出的代价也就越大。我认为东西方的差异仍要从轴心时代奠定的哲学基础寻求根源。尽管当今世界观念在融合,东西方文化仍在一些问题上表现出程度上的差异。
\par 另一方面,文化的演进也可以刻画为问题域中的搜索过程,只不过最优化问题寻求具体函数的解,文化群体寻求生活中实际问题的解。每个人的搜索和一定的信息传递机制决定了文化的演进。而搜索本身带有随机性,例如古希腊在公元前便找到了近代科学的方向,我国则在局部较优的解(阴阳五行)停留。信息传递机制使个人寻得的解传播到整个族群,也使得优势族群将寻得的解传播到其它族群,结果是近代科学一统天下。从这个意义而言,中华人、西方人本没什么区别,只是在全人类探索生活中问题解决方案的过程中一时的随机性导致了差异。

\par \rightline{2020年11月16日}


\setlength{\parindent}{0em}
\section*{On \emph{Annabel Lee}\footnote{本文为作者大一年级《美国诗歌导读》课的课程作业。}}
\addcontentsline{toc}{section}{On \emph{Annabel Lee}}

\par I have always been interested in stories, for a story has its separate value, which can never be replaced by one or two so called main ideas. In this essay, I will focus on the story of Annabel Lee, a famous narrative poem by Edgar Allan Poe.
\par \textbf{An Abstract Story}
\par A story described a bragging competition. The person who could eat the biggest thing would be the winner. ``I can swallow two space shuttles at a time,'' one said. Another claimed that he could have galaxies for breakfast. However, a monk won at last. He said, ``The thing I eat is the very biggest of all.'' It was true that he actually said nothing, but nor could others!
\par That is just what Poe did. When describing the love between ``I'' and Annabel Lee, he wrote, 
\par \setlength{\parindent}{2em}\emph{But our love it was stronger by far than the love}
\par \emph{Of those who were older than we—}
\par \emph{Of many far wiser than we—}
\par \setlength{\parindent}{0em}He also wrote,
\par \setlength{\parindent}{2em}\emph{With a love that the winged seraphs of Heaven}
\par \emph{Coveted her and me.}
\par \setlength{\parindent}{0em}But what did they exactly do together every day? How did ``I'' and Annabel Lee show their love to each other? How can we know that their love was great? Poe did not offer any concrete description, other than repeating ``it was great.''
\par Poe also used simplification in this poem. For example, he described Annabel Lee as ``lived with no other thought than to love and be loved by me''. In real world, even a child would have thoughts toward family, nature and many other things. How could a human being be that simple? Such exaggeration could make this character too far from reality.
\par On this abstract story, some may think the absence of delicate description leaves space for us to imagine. I would say, what Poe showed us was an empty frame which you could fill  with anything, just like a mathematical formula applied to all certain numbers. It may look moving at first, but like a hanging garden high above us without anything resonating with us, it can hardly touch us deeply and constantly.
\par \textbf{Darkness in Heaven}
\par The cause of Annabel Lee's death was shocking. Angels, a symbol of beauty, kindness and other virtues, envied the love and happiness ``I'' and Annabel Lee enjoyed, and then killed her with a chilling wind! 
\par Other examples can be found that gods or other noble spirits in heaven did immoral or even evil things, which I refer to as ``darkness in heaven''. In Greek mythology, Hera, the wife of Zeus, envied the beauty of Echo, the goddess of forest, and deprived her of speaking ability, who could only repeat others' last few words after that. In another story, a woman called Arachne challenged Athena on weaving and beat the goddess. With anger, Athena changed her into a spider.
\par People worship gods, then why do they make gods do bad things at the same time? Real life involves kindness and vice, honesty and deceit, so people who live in it want heaven to be a reflection of the real world, or they did it without realizing it.
\par Thus, there are both glory and darkness in heaven.
\par Back to this poem, if she had died of hunger or diseases, she might have not taken good care of herself; but she died of ``natural disaster'', so she was no fault at all. If she had been murdered by another people, ``I'' could have tried to catch and punish him; but the murderers are seraphs, so ``I'' could only cry until the eyes were dry! 
\par Poe used this element to strengthen the despair and sorrow of ``I'': it seemed that the whole world, even the angels, were against him. He could feel no justice, no kindness, no warmth, none.
\par It is reasonable that this poem was in memory of Poe's wife, Virginia, as some suggest. The loss of dear wife could throw him into despair, which was a similar situation to that in this poem.
\par \textbf{Conclusion}
\par The poem Annabel Lee is not a vivid story, but its tragedy effect is successful. Regardless of people's comments, Annabel Lee will always be a symbol of eternal love.
\par \rightline{November 12, 2016}

\section*{Escape and Return——Reflection on Frost’s \emph{Birches}\footnote{本文为作者大一年级《美国诗歌导读》课的课程作业。}}
\addcontentsline{toc}{section}{Escape and Return}

\par Frost wrote in Birches,``I’d like to get away from earth awhile, and then come back to it and begin over.'' Obviously, his word is more than talking about swinging trees. Instead, it concerns escape from a hard life situation and return to real life afterward. I refer this process to ``escape and return''.
\par One kind of escape is to transfer your thoughts or action to other real things, usually what you like. If we think of Frost’s swinging birches ``when I’m weary of considerations, and life is too much like a pathless wood'' as a specific reference, it can be one example. When I read this part, it made me think of \emph{My Favorite Things}, a song in the movie \emph{Sound of Music}. 
\par \setlength{\parindent}{2em}\emph{When the dog bites, when the bee stings}
\par \emph{When I’m feeling sad}
\par \emph{I simply remember my favorite things}
\par \emph{And then I don’t feel so bad}
\par “My favorite things” includes snowflakes, ponies, sleigh bells and so on. These things or activities can make us forget our worries and remind us of the beauty and fun in life. The mood is cheered up and the man is refreshed. In hard situations, hold fast to merely several good things in life, and there is no reason for us to be hopeless.
\par Another kind of escape involves imagination. To seek comfort or satisfaction, we tend to use our imagination to beautify things around us. Since the real world is disappointed, why not create a better one in our mind? In Birches, ice-storms bend the trees down, but the poet would think “some boy’s been swinging them”. That is a relatively more beautiful story. 
\par In \emph{Anne of Green Gables}, little Anne Shirley often use imagination to “improve her life”. In her imagination, her plain room becomes magnificent, her poor life becomes exciting. The interesting thing is that we know it is not real yet can’t stop doing this, and it does cheer us up just like real good things. Such as the story of Santa Claus living in the Arctic, we like to be cheated by this fiction.
\par When facing difficulties and almost starting to crumble, some aren’t optimistic enough to use the power of escape, and finally fall into the abyss of depression; some find the escape so wonderful that they don’t come back to real life anymore; only the strong-minded ones can resist the temptation of escape and face their real life again.
\par With imagination, our fantasy can be as good as we want. Then why do we have to come back? Frost’s answer is ``Earth is the right place to love''. Although we sometimes feel the real world is not that real while some fantasies are moving and make us feel they are real, nothing is more real than the real world itself. You cannot ride centaurs or visit Archinland, but you can ride ponies and visit Iceland. You can see the real world with naked eyes instead of mental eyes. No need for imagination, you can perceive the real world with your own ears to hear and your own fingers to touch. You can build real friendship with real person. In this aspect, no fantasy can ever compare.
\par Back to the real world, we have to confront all the troubles. That is unavoidable. Fantasies can be free of worries and cruelty just because it can be modified by our mind, or subjective; the real world can be harsh just because it is objective. The real world should always be our main battlefield. We are to achieve the value of our life on this land, this soil. During my escape, my battery is recharged. And now, let it break in all its fury. 
\par \rightline{December 18, 2016}